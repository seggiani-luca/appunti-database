\documentclass[a4paper,12pt]{article}

\usepackage[french,italian]{babel}
\usepackage[T1]{fontenc}
\usepackage[utf8]{inputenc}
\frenchspacing 
\title{Appunti Basi di Dati (Teoria)}
\author{Luca Seggiani}
\date{13 Marzo 2024}

\begin{document}
\maketitle
\section{Linguaggi per basi di dati}
I linguaggi per basi di dati appartengono a 2 categorie distinte:
\begin{itemize}
  \item Operazioni sullo schema: \textbf{Data Definition Language (DDL)};
  \item Operazioni sui dati: \textbf{Data Manipulation Lanaguage (DML)}
  \begin{itemize}
    \item Interrogazione (query)
    \item Aggiornamento (update).
  \end{itemize}    
\end{itemize}
\par\smallskip
\textbf{Interrogazioni di basi di dati} \\
Un operazione di interrogazione è un operazione di lettura che richiede l'accessso a una o più tabelle. Per specificare
interrogazioni si possono seguire due formalismi:
\begin{itemize}
  \item \textbf{Modo dichiarativo}: si specificano le proprietà del risultato (che cosa)
  \item \textbf{Modo procedurale}: si specificano le modalità di generazione del risultato (come).
\end{itemize}
L'\textbf{algebra relazionale} permette di specificare delle interrogazioni secondo il modello procedurale: cioè elencando
i passi "primitivi" necessari alla generazione di una risposta. Il \textbf{calcolo relazionale} invece ci permette di definire
in modo dichiarativo quello che sarà il risultato dell'interrogazione. Come vedremo, sarà il calcolo relazionale a definire
la semantica del linguaggio relazionale, perchè permette di fornire un'implementazione slegata dai dettagli procedurali.

\section{Algebra relazionale}
Un'algebra è una struttura matematica dotata di un'insieme di dati e una serie di operatori che manipolano suddetti dati.
Nell'algebra relazionale abbiamo che:
\begin{itemize}
  \item \textbf{Dati}: relazioni
  \item \textbf{Operatori}:
    \begin{itemize}
      \item su relazioni
      \item che producono relazioni
      \item che possono essere composti.
    \end{itemize}
    essi sono divisi fra:
    \begin{itemize}
      \item \textbf{Operatori su insiemi}: \\
        unione, intersezione, differenza
      \item \textbf{Operatori su relazioni}: \\
        ridenominazione, selezione, proiezione, \\
        join:
        \begin{itemize}
          \item naturale, prodotto cartesiano, theta
        \end{itemize}
    \end{itemize}
\end{itemize}
\par\smallskip
\textbf{Notazione dell'algebra relazionale} \\
Specifichiamo adesso la notazione usata:
\begin{itemize}
  \item $R,R_1,...$ indicano nomi di relazionie
  \item $A,B,C,A_1...$ indicano nomi di attributo
  \item $XY$ è un'abbreviazione di $X \cup Y$
  \item Una relazione con n-uple $t_1, t_2,...$ è indicata con l'insieme $\{ t_1, t_2, ...\} $
  \item $t_j[A_j]$ indica il valore della n-upla $t_j$ sull'attributo $A_j$
  \item $t[X]$ indica l'n-upla ottenuta da considerando solo gli elementi di $X$.
\end{itemize}
\par\smallskip
\textbf{Operatori su insiemi} \\
Le relazioni sono insiemi di tuple, e non possono avere elementi duplicati (sarebbero altrimenti multiinsiemi). I risultati
di operazioni fra relazioni sono  a loro volta relazioni, ovvero insiemi di tuple. Gli operatori fra relazioni
possono applcarsi solo e soltanto fra relazioni definite sullo stesso insieme di attributi $X$, e il risultato sarà
a sua volta definito sullo stesso insieme di attributi $X$.
\par\smallskip
\textbf{Unione} \\
L'operatore di unione fra due relazioni sull'insieme di attributi $X$ comporta la formazione di una nuova 
relazione che ha tutte le n-uple delle relazioni unite. Eventuali n-uple identiche fra le relazioni appariranno nel risultato una volta sola.
Il suo simbolo è $\cup$.
\par\smallskip
\textbf{Intersezione} \\
L'operatore di intersezione fra due relazioni sull'insieme di attributi $X$ restituisce una nuova relazione che contiene
soltanto gli elementi appartenenti sia alla prima che alla seconda relazione.
Il suo simbolo è $\cap$.
\par\smallskip
\textbf{Ðifferenza} \\
L'operatore di differenza fra due relazioni sull'insieme di attributi $X$ restituisce una nuova relazione contenente tutte
le n-uple che appartengono alla prima relazione ma non alla seconda. L'operatore di differenza è l'unico fra gli operatori
insiemistici a non essere commutativo. Il suo simbolo è il meno ($-$), o più propriamente \textbackslash.
\par\smallskip
Notiamo che l'unione è l'unico operatore in grado di creare relazioni con un numero di elementi maggiore di quello degli operandi. Questa
caratteristica tornerà poi utile nel calcolo relazionale. 
\par\smallskip
Descriviamo adesso gli operatori non insiemistici (sulle relazioni):\\
\textbf{Ridenominazione} \\
L'operatore di ridenominazione è un'operatore monadico che modifica lo schema dell'operando, lasciando inalterata
l'istanza. Si scrive, data una relazione $R$, come:
$$ \rho_{B_1B_2...} \leftarrow_{A_1A_2...(R)} $$
I pedici a sinistra e a destra della freccia sono insiemi di attributi. Avremo che:
\begin{itemize}
  \item L'attributo $A_1$ viene sostituito dall'attributo $B_1$
  \item L'attributo $A_2$ viene sostituito dall'attributo $B_2$
  \item ecc...
\end{itemize}
\par\smallskip
\textbf{Selezione} \\
L'operatore di selezione è un operatore monadico che produce un un risultato con lo stesso
schema dell'operando, e un sottoinsieme di n-uple che rispettano una determinata condizione. Si scrive come:
$$ \sigma_F(R)$$
sulla relazione $R$, dove $F$ è un'espressione Booleana (predicato) ottenuta componendo con gli operatori logici AND, OR e NOT
le condizioni atomiche:
\begin{itemize}
  \item $A \star B$, dove $A$ e $B$ sono attributi di $X$ con domini compatibili e $\star$ un operatore di confronto.
  \item $A \star k$, dove $A$ è un attributo di $X$ e $k$ una costante con dominio compatibile con $A$ e $\star$ sempre un operatore di confronto. 
\end{itemize}
Notiamo che la condizione atomica è vera solo per valori non nulli in ogni attributo.
\par\smallskip
\textbf{Selezione valori NULL} \\
Per riferirsi ai valori NULL occorre usare le apposite condizioni: IS NULL e IS NOT NULL.
\par\smallskip
\textbf{Proiezione} \\
La proiezione è un'operatore monadico che produce un sottoinsieme degli attributi dell'operando contenente tutte
le sue n-uple ristrette soltanto ad alcuni attributi. Si scrive, data una relazione $R(X)$ e un insieme
di attributi $Y \subseteq X$, come:
$$ \pi_Y(R) $$
Il risultato è una relazione su $Y$ che contiene l'insieme delle n-uple di $R$ ristrette ai soli attributi di $Y$.
Notare che, di nuovo, non possono esserci righe ripetute. Ogni n-upla ripetuta nell'insieme risultato verrà inserita una volta sola.
\par
\textbf{Cardinalità di proiezioni} \\
Una proiezione contiene al massimo tante n-uple quante ne contiene l'operando. Inolre, se l'insieme di attributi $X$ è superchiave
della relazione $R$, allora la proiezione $\pi_X(R)$ avrà tante n-uple quante ne ha l'operando.
\par\smallskip
\textbf{Join} \\
Non possiamo, usando gli operatori di selezione e proiezione, estrarre e combinare informazioni da più relazioni diverse fra loro,
e non possiamo nemmeno combinare informazioni presenti in n-uple diverse di una stessa relazione. Avremo allora bisogno
di un'ulteriore serie di operazioni, le cosiddette operazioni di join. I join ci permettono di unire sulla stessa riga più righe di
relazioni diverse. Vediamo nel dettaglio:
\par\smallskip
\textbf{Join naturale} \\
Il join naturale è un'operatore con due operandi (generalizzabile), che produce come risultato l'unione degli attributi degli operandi,
contente le n-uple costruite ciascuna a partire da un n-upla di ognuno degli operandi. Formalmente:
Date due relazioni $R_1{X_1}$ e $R_2{X_2}$, il loro join naturale si scrive come:
$$ R_1 \bowtie R_2 $$
Il risultato è una relazione $R(X_1 \cup X_2)$ definita come:
$$R(X_1 \cup X_2) = R_1(X_1) \bowtie R_2(X_2) = \{ t | \exists t_1 \in R_1, \quad t_2 \in R_2 $$
$$ \mathrm{con} \quad t[X_1]=t_1, \quad t[X_2] = t_2 \}$$
N.B.: è perfettamente plausibile voler fare il join naturale tra due relazioni senza attributi in comune,
e in questo caso si ottiene il prodotto cartesiano fra le due.
\par\smallskip
\textbf{Cardinalità del join} \\
Il join di due relazioni $R_1$ e $R_2$ contiene un numero di n-uple compreso fra 0 e il prodotto di $|R_1|$ e $|R_2|$.
Se il join coinvolge una chiave di $R_2$ allora il numero di n-uple è compreso fra 0 e $|R_1|$. Se il join coinvolge una
chiave di $R_2$ e un vincolo di integrità referenziale allora il numero di n-uple è uguale a $|R_1|$. Più formalmente: \\
Il join di $R_1(A, B)$ e $R_2(B,C)$ contiene un un numero di n-uple:
$$ 0 \leq |R_1 \bowtie R_2| \leq |R_1| \times |R_2| $$
Se $B$ è una chiava di $R_2$ allora il numero di n-uple è 
$$ 0 \leq |R_1 \bowtie R_2| \leq |R_1|$$
Se $B$ è una chiave di $R_2$ ed esiste un vincolo di integrità referenziale fra $B$ (in $R_1$) e $R_2$ allora il numero delle
n-uple è:
$$ |R_1 \bowtie R_2| = |R_1| $$
Una problematica del join è il fatto che le n-uple che non contribuiscono al risultato ("che non fanno join") restano tagliate fuori.
Introduciamo allora altri tipi di operatori di join, i cosiddetti:
\par\smallskip
\textbf{Join esterno} \\
Il join esterno estende, con valori NULL, le n-uple che verrebbero altrimenti scartate dal join (interno).
Ne esistono tre versioni:
\begin{itemize}
  \item \textbf{Sinistro}: mantiene tutte le n-uple del primo operando ($\bowtie_{LEFT}$);
  \item \textbf{Destro}: mantiene tutte le n-uple del secondo operando ($\bowtie_{DESTRO}$);
  \item \textbf{Completo}: mantiene tutte le n-uple di entrambi gli operandi ($\bowtie_{FULL}$).
\end{itemize}
\textbf{Join e proiezioni} \\
Date due relazioni $R_1(X_1)$ e $R_2(X_2)$:
$$ \pi_{X_1}(R_1 \bowtie R_2) \subseteq R_1 $$
Date una relazione  $R(X)$ con $X = X_1 \cup X_2$
$$ (\pi_{X_1}(R) \bowtie \pi_{X_2}(R)) \supseteq R$$
\textbf{Prodotto cartesiano} \\
Come detto prima, date due relazioni $R_1(X_1)$ e $R_2(X_2)$, senza attributi in comune, cioè con $X_1 \cap X_2 = {0}$, la
definizione di join naturale e comunque senso e restituisce il prodotto cartesiano delle due relazioni:
$$ R = R_1 \bowtie R_2 = R_1 \times R_2 $$
la cardinalità del prodotto cartesiano è uguale al prodotto delle cardinalità delle due relazioni.
\par\smallskip
\textbf{Theta join} \\
Nella pratica, il prodotto cartesiano ha senso quasi solamente se seguito da una selezione $\sigma_F(R_1 \times R_2)$.
Questa combinazione prende il nome di theta join (è un operatore derivato) ed è indicato come:
$$ R_1 \bowtie_F R_2 $$
dove $F$ è un certo rpedicato. Spesso $F$ è una congiunzione di atomi di confronto $A_1 \sigma A_2$ dove $\sigma$ è
un operatore di confronto (<=, <, =, ecc...) e $A_1$ e $A_2$ attributi di relazioni diverse. Quando l'operatore
di confronto è l'uguaglianza parliamo di equi-join.

\end{document}
