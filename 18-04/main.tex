\documentclass[a4paper,12pt]{article}

\usepackage[french,italian]{babel}
\usepackage[T1]{fontenc}
\usepackage[utf8]{inputenc}
\frenchspacing 
\title{Appunti Basi di Dati}
\author{Luca Seggiani}
\date{18 Aprile 2024}

\usepackage{listings}
\usepackage{xcolor}

\definecolor{codegreen}{rgb}{0,0.6,0}
\definecolor{codegray}{rgb}{0.5,0.5,0.5}
\definecolor{codepurple}{rgb}{0.58,0,0.82}
\definecolor{backcolour}{rgb}{0.95,0.95,0.92}

\lstdefinestyle{code-style}{
    backgroundcolor=\color{backcolour},   
    commentstyle=\color{codegreen},
    keywordstyle=\color{magenta},
    numberstyle=\tiny\color{codegray},
    stringstyle=\color{codepurple},
    basicstyle=\ttfamily\footnotesize,
    breakatwhitespace=false,         
    breaklines=true,                 
    captionpos=b,                    
    keepspaces=true,                 
    numbers=left,                    
    numbersep=5pt,                  
    showspaces=false,                
    showstringspaces=false,
    showtabs=false,                  
    tabsize=2
}

\lstset{style=code-style}

\begin{document}
\maketitle
\section{Costrutto EXISTS}
Il costrutto EXISTS ci permette di verificare che il result set di una subquery correlata contenga almeno un
record. La sua negazione controlla che il result set sia vuoto. Poniamo ad esempio di voler trovare medico,
paziente e data delle visite di controllo del mese di gennaio 2016. Definiamo una visita di controllo come una
visita in cui un medico visita un paziente già visitato precedentemente almeno una volta.
\begin{lstlisting}[language=SQL]
SELECT V_1.Medico, V_1.Paziente, V_1.Data
FROM Visita V_1
WHERE MONTH(V_1.Data) == 1 AND YEAR(V_1.Data) = 2016
  AND EXISTS
    (
    SELECT *
    FROM Visita V_2
    WHERE V_2.Paziente = V_1.Paziente
      AND V_2.Medico = V_1.Medico
      AND V_2.Data < V_1.Data
    )
\end{lstlisting}
Diciamo invece di voler indicare le matricole dei medici che hanno visitato per la primma volta almeno un paziente
nel mese di ottobre 2013.
\begin{lstlisting}[language=SQL]
SELECT DISTINCT V_1.Medico
FROM Visita V_1
WHERE MONTH(V_1.Data) = 10 AND YEAR(V_1.Data) = 2013
  AND NOT EXISTS
  (
  SELECT *
  FROM Visita V_2
  WHERE V_2.Paziente = V_1.Paziente
    AND V_2.Medico AND V_1.Medico
    AND V_1.Data < V_2.Data
  )
\end{lstlisting}
\section{Divisione insiemistica}
L'operatore di divisionie insiemistica permette di trovare tutti i record che rispettano condizioni esaustive. Ad esempio,
possiamo usarlo per indicare i pazienti visitati da tutti i medici. L'SQL non fornisce un'operatore di divisione insiemistica
di per sé, ma possiamo crearne uno attraverso altri costrutti. Esistono due metodi:
\begin{itemize}
  \item
  Possiamo usare il costrutto EXISTS e le subquery correlate per ottenere la seguente query:
    \begin{lstlisting}[language=SQL]
    SELECT P.CodFiscale
    FROM Paziente P
    WHERE NOT EXISTS 
    (
      SELECT *
      FROM Medico M
      WHERE NOT EXISTS
      (
        SELECT *
        FROM Visita V
        WHERE V.Medico = M.Matricola
        AND V.Paziente = P.CodFiscale
      )
    )
    \end{lstlisting}
    La query si traduce pressapoco con "un paziente è stato visitato da tutti i medici se non esiste un medico per cui
    non esiste una visita di quel medico con quel paziente".
  \item Possiamo usare il raggruppamento e una subquery non correlata:
    \begin{lstlisting}[language=SQL]
    SELECT V.Paziente
    FROM Visita V
    GROUP BY V.Paziente
    HAVING COUNT(DISTINCT V.Medico) = (SELECT COUNT(*)
                                        FROM Medico)
    \end{lstlisting}
    Che è considerevolmente più semplice.
\end{itemize}
\end{document}
