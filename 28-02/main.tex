\documentclass[a4paper,12pt]{article}

\usepackage[french,italian]{babel}
\usepackage[T1]{fontenc}
\usepackage[utf8]{inputenc}
\frenchspacing 
\title{Appunti Basi di Dati}
\author{Luca Seggiani}
\date{28 Febbraio 2024}

\usepackage{listings}
\usepackage{xcolor}

\definecolor{codegreen}{rgb}{0,0.6,0}
\definecolor{codegray}{rgb}{0.5,0.5,0.5}
\definecolor{codepurple}{rgb}{0.58,0,0.82}
\definecolor{backcolour}{rgb}{0.95,0.95,0.92}

\lstdefinestyle{code-style}{
    backgroundcolor=\color{backcolour},   
    commentstyle=\color{codegreen},
    keywordstyle=\color{magenta},
    numberstyle=\tiny\color{codegray},
    stringstyle=\color{codepurple},
    basicstyle=\ttfamily\footnotesize,
    breakatwhitespace=false,         
    breaklines=true,                 
    captionpos=b,                    
    keepspaces=true,                 
    numbers=left,                    
    numbersep=5pt,                  
    showspaces=false,                
    showstringspaces=false,
    showtabs=false,                  
    tabsize=2
}

\lstset{style=code-style}

\begin{document}
\maketitle
\section{Introduzione alla base di dati relazionale}
Una base di dati (in inglese "database") è una mole di dati organizzata in modo da favorirne la consultazione
attraverso determinate interrogazioni ("query"). Formalmente, una base di datiè un archivio di dati contenente
informazioni ben strutturate organizzate secondo un modello logico (nel caso di MySQL un modello relazionale).
\par \smallskip
L'elemento fondamentale di una base di dati fondata sul modello relazionale è la \textbf{Tabella}, formata
da colonne (associate a determinati attributi), e righe (che rappresentano record dei suddetti attributi).
Una colonna particolare, che deve differire per ogni record della tabella, viene detta chiave primaria e serve
a distinguere tra di loro record (che potrebbero essere altrimente stati tra di loro identici!).
\par \smallskip
L'interrogazione della base di dati si effettua attraverso le cosiddette query, ovvero richieste compilate in
linguaggio SQL (Structured Query Language), un linguaggio dichiarativo sviluppato appositamente per questo
scopo. La query più semplice si basa su tre istruzioni: SELECT, FROM e WHERE. Prendiamo l'esempio in lingua
naturale di un'interrogazione a una base di dati contenente dati su diverse persone (nello specifico, nome,
cognome, età e codice fiscale come chiave.):

\begin{center}
  \textit{Riportare nomi e cognomi di tutte le persone con eta maggiore di 40 anni.}
\end{center}

in questo caso gli attributi di interesse sono i nomi e cognomi (in quest'ordine) delle persone valide, e la 
condizione è un età maggiore di 40 anni. Di fronte a questa richiesta, la base di dati risponderà fornendo
un insieme risultato (result set) contenente una tabella formata esattamente dalle informazioni richieste.
In SQL potremmo definire tale query come segue:

\begin{lstlisting}[language=SQL]
SELECT Cognome, Nome
FROM Persona
WHERE Eta > 40
\end{lstlisting}

in questo caso, cognome e nome sono gli attributi di interesse che verrano inseriti nel nostro insieme risultato
(attraverso un processo chiamato proiezione). La tabella persona sarà la fonte dei nostri dati, e età > 40 la 
nostra condizione.
\par \smallskip

Le condizioni in SQL supportano anche i classici operatori logici (AND, OR, ecc...). Un operatore degno di nota,
applicabile ai soli dato di tipo numerico o su cui è comunque stabilita una relazione d'ordine, è il BETWEEN,
che permette di selezionare (estremi inclusi) tutti i valori in un dato range.
\par \medskip
\textbf{Duplicati}
2 righe risultano uguali se tutti gli attributi del record hanno valori identici. Non si possono avere righe
uguali in tabelle SQL (la chiave distingue), ma è possibile avere duplicati tra gli attributi non chiave, con
quindi risultanti problemi dati dalla proiezione di tali attributi su insiemi risultato privi di attributo chiave.
Nel caso si rendesse necessario eliminare i duplicati da una data query, l'SQL offre il costrutto SELECT DISTINCT.
Si nota che il costo per l'eliminazione dei duplicaty è computazionalmente $O(n^2)$, e quindi da evitare quando
possibile.
\par \medskip
\textbf{Valori NULL}
I valori NULL corrispondono solitamente a informazioni mancanti, non disponibili o comunque sconosciute.
Qualsiasi condizione che coinvolge valori NULL è sempre falsa, compreso il confronto fra stessi NULL.
È possibile utilizzare i valori NULL assegnandoli un valore semanticamente definito (e.g. una data non definita
perchè non ancora determinata, ecc...). In questo caso, l'SQL offre i costrutti IS NOT NULL e IS NULL, che restituiscono
rispettivamente falso e vero posti accanto ad un valore NULL. Si nota inoltre che la chiave di un record non
può essere NULL.
\par \medskip
\textbf{Gestione delle date}
Come per tutti i sistemi UNIX, le date in SQL vengono rappresentate come una distanza in secondi dalla mezzanotte
del primo gennaio 1970. 

\end{document}
