\documentclass[a4paper,12pt]{article}

\usepackage[french,italian]{babel}
\usepackage[T1]{fontenc}
\usepackage[utf8]{inputenc}
\frenchspacing 
\title{Appunti Basi di Dati}
\author{Luca Seggiani}
\date{17 Aprile 2024}

\usepackage{hyperref}

\usepackage{tikz}
\usetikzlibrary{shapes.geometric, shapes.arrows}

\tikzstyle{entity} = [rectangle, text centered, minimum width=3cm, minimum height=1.5cm , draw=black, fill=white]
\tikzstyle{relationship} = [diamond, text centered, draw=black, fill=white]
\tikzstyle{arrow} = [->, >=stealth]
\tikzstyle{connector} = [-, >=stealth]
\tikzstyle{filledarrow} = [single arrow, draw=black, fill=black, rotate=90, minimum height=0.7cm]
\tikzstyle{emptyarrow} = [single arrow, draw=black, fill=white, rotate=90, minimum height=0.7cm]

\begin{document}
\maketitle
\section{Concetti inesprimibili nel modello ER}
Alcuni concetti sono inesprimibili attraverso il modello ER tradizionale, e bisognano quindi di costrutti
particolari:
\begin{itemize}
  \item \textbf{Cardinalità}
    \begin{itemize}
      \item di relationship;
      \item di attributo;
    \end{itemize}
  \item \textbf{Identificatore}
    \begin{itemize}
      \item interno;
      \item esterno;
    \end{itemize}
  \item \textbf{Generalizzazione}
\end{itemize}
\par\smallskip
\textbf{Cardinalità di relationship} \\
Una cardinalità di relationship è rappresentata da una coppia di valori associati ad ogni entità che partecipa alla
relationship. Questi specificano il numero minimo e massimo di occorrenze della relationship a cui ciascuna
occorrenza di entità può partecipare.
I simboli usati saranno:
\begin{itemize}
  \item Per la minima:
  \begin{itemize}
    \item 0 $\rightarrow$ partecipazione opzionale
    \item 1 $\rightarrow$ partecipazione obbligatoria
  \end{itemize}
  \item Per la massima:
    \begin{itemize}
      \item 1 $\rightarrow$ partecipazione opzionale
      \item N $\rightarrow$ non pone alcun limite
    \end{itemize}
\end{itemize}
Con riferimento alle cardinalità massime, abbiamo relationship:
\begin{itemize}
  \item \textbf{Uno a uno}, ad esempio:\\
    \begin{tikzpicture}[node distance=2cm]
      \node (studente) [entity] {Professore};
      \node (esame) [entity, right of=studente, xshift=8cm] {Cattedra};
      \node (presenza) [relationship, right of= studente, xshift=3cm] {Titolarità};
      \draw [arrow] (studente) -- node[anchor=south] {(0, 1)} (presenza);
      \draw [arrow] (esame) -- node[anchor=south] {(0, 1)} (presenza);
    \end{tikzpicture} \par\smallskip
    Un professore può essere tale senza essere titolare di alcuna cattedra, e una cattedra può restare
    vuota. \par\smallskip
    \begin{tikzpicture}[node distance=2cm]
      \node (studente) [entity] {Professore di ruolo};
      \node (esame) [entity, right of=studente, xshift=8cm] {Cattedra};
      \node (presenza) [relationship, right of= studente, xshift=3cm] {Titolarità};
      \draw [arrow] (studente) -- node[anchor=south] {(1, 1)} (presenza);
      \draw [arrow] (esame) -- node[anchor=south] {(0, 1)} (presenza);
    \end{tikzpicture} \par\smallskip
    Un professore di ruolo dovrà ovviamente essere titolare di almeno una cattedra, ma questo non significa
    comunque che tutte le cattedre siano occupate. \par\smallskip
    \begin{tikzpicture}[node distance=2cm]
      \node (studente) [entity] {Professore di ruolo};
      \node (esame) [entity, right of=studente, xshift=8cm] {Cattedra occupata};
      \node (presenza) [relationship, right of= studente, xshift=3cm] {Titolarità};
      \draw [arrow] (studente) -- node[anchor=south] {(1, 1)} (presenza);
      \draw [arrow] (esame) -- node[anchor=south] {(1, 1)} (presenza);
    \end{tikzpicture} \par\smallskip
    E' una forzatura? Sì. Serve a spiegare il concetto? Sempre sì. 
  \item \textbf{Uno a molti}, ad esempio:\\
    \begin{tikzpicture}[node distance=2cm]
      \node (studente) [entity] {Persona};
      \node (esame) [entity, right of=studente, xshift=8cm] {Azienda};
      \node (presenza) [relationship, right of= studente, xshift=3cm] {Impiego};
      \draw [arrow] (studente) -- node[anchor=south] {(0, 1)} (presenza);
      \draw [arrow] (esame) -- node[anchor=south] {(0, N)} (presenza);
    \end{tikzpicture} \par\smallskip
    Una persona può essere o non essere assunta, ma al massimo da una azienda. Al contrario, un'azienda avrà
    probabilmente più di un'impiegato. \par\smallskip
    \begin{tikzpicture}[node distance=2cm]
      \node (studente) [entity] {Ufficiale};
      \node (esame) [entity, right of=studente, xshift=8cm] {Nave};
      \node (presenza) [relationship, right of= studente, xshift=3cm] {Comando};
      \draw [arrow] (studente) -- node[anchor=south] {(1, N)} (presenza);
      \draw [arrow] (esame) -- node[anchor=south] {(0, 1)} (presenza);
    \end{tikzpicture} \par\smallskip
    Un ufficiale per considerarsi tale deve comandare almeno una nave, ma non è detto che tutte le navi abbiano un
    comandante. Alcune sono navi fantasma come nei Pirati dei caraibi. \par\smallskip
    \begin{tikzpicture}[node distance=2cm]
      \node (studente) [entity] {Comune};
      \node (esame) [entity, right of=studente, xshift=8cm] {Provincia};
      \node (presenza) [relationship, right of= studente, xshift=3cm] {Ubicazione};
      \draw [arrow] (studente) -- node[anchor=south] {(1, 1)} (presenza);
      \draw [arrow] (esame) -- node[anchor=south] {(1, N)} (presenza);
    \end{tikzpicture} \par\smallskip
    Ogni comune è ubicato in una e una sola provincia, mentre ogni provincia è l'ubicazione di più comnuni.
  \item \textbf{Molti a molti}, ad esempio:\\
    \begin{tikzpicture}[node distance=2cm]
      \node (studente) [entity] {Studente};
      \node (esame) [entity, right of=studente, xshift=8cm] {Esame};
      \node (presenza) [relationship, right of= studente, xshift=3cm] {Presenza};
      \draw [arrow] (studente) -- node[anchor=south] {(0, N)} (presenza);
      \draw [arrow] (esame) -- node[anchor=south] {(0, N)} (presenza);
    \end{tikzpicture} \par\smallskip
    Non è detto che ogni studente abbia sostenuto un'esame, come non è detto che ogni esame
    sia stato sostenuto da almeno uno studente. \par\smallskip
    \begin{tikzpicture}[node distance=2cm]
      \node (studente) [entity] {Montagna};
      \node (esame) [entity, right of=studente, xshift=8cm] {Alpinista};
      \node (presenza) [relationship, right of= studente, xshift=3cm] {Scalata};
      \draw [arrow] (studente) -- node[anchor=south] {(0, N)} (presenza);
      \draw [arrow] (esame) -- node[anchor=south] {(1, N)} (presenza);
    \end{tikzpicture} \par\smallskip
    Per potersi considerare alpinisti occorre aver scalato almeno una montagna, ma non è detto che tutte le
    montagne siano state scalate. \par\smallskip
    \begin{tikzpicture}[node distance=2cm]
      \node (studente) [entity] {Autore};
      \node (esame) [entity, right of=studente, xshift=8cm] {Libro};
      \node (presenza) [relationship, right of= studente, xshift=3cm] {Scrittura};
      \draw [arrow] (studente) -- node[anchor=south] {(1, N)} (presenza);
      \draw [arrow] (esame) -- node[anchor=south] {(1, N)} (presenza);
    \end{tikzpicture} \par\smallskip
    Ogni autore ha scritto almeno un libro per considerarsi tale, e ogni libro deve essere stato scritto da almeno
    un autore. Questo però non pregiudica che un autore non possa scrivere più libri o un libro non possa essere scritto
    scritto da più autori.
\end{itemize}
\par\smallskip
\textbf{Identificatore di entità} \\
L'identificatore di entità è uno strumento per l'identificazione univoca delle occorrenze di un entità.
\begin{itemize}
  \item Gli attributi dell'entità possono formare l'\textbf{identificatore interno} (o chiave). Per capirsi, un'identificatore interno è una chiave primaria o 
    comunque una chiave candidata. \par\smallskip
    \begin{tikzpicture}
    \node (entita) [entity] {Automobile};
    \fill (2,1) circle (0.1cm);
    \node[right] at (2.2,1) {Targa};
    \draw [connector] (1.9,1) -- (entita);

    \draw (2,0) circle (0.1cm);
    \node[right] at (2.2,0) {Colore};
    \draw [connector] (1.9,0) -- (entita);

    \draw (2,-1) circle (0.1cm);
    \node[right] at (2.2,-1) {Modello};
    \draw [connector] (1.9,-1) -- (entita);
  \end{tikzpicture} \par\smallskip
  Come vediamo dalla figura, targa è la chiave primaria, ovvero l'identificatore. Nel caso un'insieme di attributi sia una chiave,
  adottiamo la rappresentazione seguente: \par\smallskip
    \begin{tikzpicture}
    \node (entita) [entity] {Studente};
    \draw (2,1) circle (0.1cm);
    \node[right] at (2.2,1) {Nome};
    \draw [connector] (1.9,1) -- (entita);

    \draw (2,0) circle (0.1cm);
    \node[right] at (2.2,0) {Cognome};
    \draw [connector] (1.9,0) -- (entita);

    \draw (2,-1) circle (0.1cm);
    \node[right] at (2.2,-1) {Età};
    \draw [connector] (1.9,-1) -- (entita);

    \fill (1.7,1.1) circle (0.1cm);
    \draw[connector] (1.7, 1.1) -- (1.7, -1.1);
  \end{tikzpicture} \par\smallskip
  \item Gli attributi dell'entità e l'identificatore interno di entità esterne raggiunte attraverso relationship formano un'\textbf{identificatore esterno}.
  un'identificazione esterna può essere possibile solo nel caso esista una relationship in cui l'entità da identificare abbia cardinalità $(1,1)$. Per
  capirsi, un identificatore è una chiave contenente una chiave esterna.
    \begin{tikzpicture}[node distance=2cm]
      \node (studente) [entity] {Studente};
      \node (esame) [entity, right of=studente, xshift=8cm] {Facoltà};
      \node (presenza) [relationship, right of= studente, xshift=3cm] {Iscrizione};
      \draw [arrow] (studente) -- node[anchor=south] {(1, 1)} (presenza);
      \draw [arrow] (esame) -- node[anchor=south] {(0, N)} (presenza);

      \draw (2,1) circle (0.1cm);
      \node[right] at (2.2,1) {Nome};
      \draw [connector] (1.9,1) -- (entita);
  
      \fill (1.7,1.1) circle (0.1cm);
      \draw[connector] (1.7, 1.1) -- (1.7, -1.1);
    \end{tikzpicture} \par\smallskip
    Dalla figura si capisce che il nome dello studente e la sua iscrizione a una certa facolà formano il suo identificatore esterno.

\end{itemize}
Ogni entità deve possedere almeno un identificatore, ma può averne in generale più di uno.

\par\smallskip
\textbf{Generalizzazione} \\
La generalizzazione mette in relazione una o più entità $E_1, E_2, ... , E_n$ con una singola entità $E$ che le 
comprende come casi particolari. Si dice che $E$ è una generalizzazione (oppure entità genitrice, madre) di
$E_1, E_2, ... , E_n$ (dette specializzazioni, sottotipi o entità figlie). \par\smallskip
\begin{center}
\begin{tikzpicture}
  \node (dipendente) [entity] {Dipendente};
  \node (funzionario) [entity, right of=dipendente, yshift=-3cm, xshift=-3cm] {Funzionario};
  \node (amministratore) [entity, right of=dipendente, yshift=-3cm, xshift=1cm] {Amministratore};
  \node (filled) [filledarrow, right of=dipendente, xshift=-2.2cm] {};
  \draw [connector] (funzionario) -- (filled);
  \draw [connector] (amministratore) -- (filled);
\end{tikzpicture}
\end{center}
Le caratteristiche delle generalizzazioni sono:
\begin{itemize}
  \item \textbf{Ereditarietà}: tutte le proprietà dell'entità genitore vengono ereditate dalle figlie e non rappresentate
    esplicitamente.
  \item \textbf{Generalizzazione totale}: se ogni occorenza dell'entità genitore è rimpiazzata da almeno un occorrenza delle entità
    figlie, altrimenti si parla di \textbf{generalizzazione parziale}. Di norma le generalizzazioni parziali si indicano
    con una freccia vuota, mentre quelle totali con una freccia piena.
\end{itemize}
  \noindent 
  \begin{tikzpicture}
  \node (dipendente) [entity] {Dipendente};
  \node (funzionario) [entity, right of=dipendente, yshift=-3cm, xshift=-3cm] {Funzionario};
  \node (amministratore) [entity, right of=dipendente, yshift=-3cm, xshift=1cm] {Amministratore};
  \node (filled) [filledarrow, right of=dipendente, xshift=-2.2cm] {};
  \draw [connector] (funzionario) -- (filled);
  \draw [connector] (amministratore) -- (filled);
\end{tikzpicture}
\hspace{0.3cm}
\begin{tikzpicture}
  \node (dipendente) [entity] {Persona};
  \node (funzionario) [entity, right of=dipendente, yshift=-3cm, xshift=-3cm] {Dipendente};
  \node (amministratore) [entity, right of=dipendente, yshift=-3cm, xshift=1cm] {Studente};
  \node (filled) [emptyarrow, right of=dipendente, xshift=-2.2cm] {};
  \draw [connector] (funzionario) -- (filled);
  \draw [connector] (amministratore) -- (filled);
\end{tikzpicture}
\begin{itemize}
  \item \textbf{Generalizzazione esclusiva}: se ogni occorrenza dell'entità genitrice è occorrenza di al massimo una delle
    entità figlie, altrimenti si parla di \textbf{generalizzazione sovrapposta}.
  \item Possono esistere \textbf{gerarchie a più livelli} e multiple generalizzazioni allo stesso livello.
  \item Un'entità può essere inclusa in più gerarchie, sia come genitore che come figlia, se non entrambe.
  \item Se una generalizzazione ha solo un'entita figlia si dice \textbf{sottoinsieme}.
  \item Il genitore di una generalizzazione totale può non avere identificatore (\textbf{anonimato}) purché
    siano identificate le figlie.
\end{itemize}
\section{Progettazione concettuale}
Vediamo adesso la progettazione concettuale nel dettaglio. Il modello ER sarà lo strumento fondamentale nel corso
di questa fase: dovremo innanzitutto decidere, per una qualsiasi specifica fornitaci, quale sia il costrutto ER
più adeguato da utilizzare. Per fare questo ci basiamo sulle definizioni dei costrutti del modello ER:
\begin{itemize}
  \item \textbf{Entità}, se un oggetto ha proprietà significative e descrive oggetti con esistenza autonoma;
  \item \textbf{Attributo}, se un'oggetto è semplice e non ha proprietà specifiche;
  \item \textbf{Relationship}, se un concetto correla più oggetti;
  \item \textbf{Generalizzazione}, se un concetto è caso particolare di un altro.
\end{itemize}
\par\smallskip
\textbf{Design pattern} \\
I design pattern sono soluzioni progettuali a problemi comuni. Sono largamente usati nell'ingegneria del software,
e ne esistono alcuni comuni nella progettazione delle basi di dati.
\begin{itemize}
  \item \textbf{Reificazione di attributo di entità} \\
La reificazione di un attributo di un'entità è la promozione di tale attributo ad un'entità a sé stante, con ovvia creazione
di una relationship che esprima il rapporto fra l'entità di partenza e l'attributo che abbiamo trasformato in entità. Supponiamo
di avere il modello ER:
\begin{tikzpicture}
\node (entita) [entity] {Impiegato};
    \fill (2,1) circle (0.1cm);
    \node[right] at (2.2,1) {Codice};
    \draw [connector] (1.9,1) -- (entita);

    \draw (2,0) circle (0.1cm);
    \node[right] at (2.2,0) {Nome};
    \draw [connector] (1.9,0) -- (entita);

    \draw (2,-1) circle (0.1cm);
    \node[right] at (2.2,-1) {Azienda};
    \draw [connector] (1.9,-1) -- (entita);
\end{tikzpicture} \par\smallskip
L'attributo azienda potrebbe essere trasformato in un entità a sé, per poter includere maggiori informazioni rispetto all'azienda stessa:
\par\smallskip
\begin{tikzpicture}[node distance=2cm]
      \node (studente) [entity] {Impiegato};

    \fill (2,1) circle (0.1cm);
    \node[right] at (2.2,1) {Codice};
    \draw [connector] (1.9,1) -- (entita);

    \draw (2,-1) circle (0.1cm);
    \node[right] at (2.2,-1) {Nome};
    \draw [connector] (1.9,-1) -- (entita);

      \node (esame) [entity, right of=studente, xshift=8cm] {Azienda};
      \node (presenza) [relationship, right of= studente, xshift=3cm] {Impiego};
      \draw [arrow] (studente) -- node[anchor=south] {(1, 1)} (presenza);
      \draw [arrow] (esame) -- node[anchor=south] {(1, N)} (presenza);
    \end{tikzpicture} \par\smallskip
\item \textbf{Reificazione di relationship in entità} \\ 
  La reificazione di una relationship in entità è la promozione di una relationship in un'entità a sé stante, che sarà
  opportunamente collegata alle entità che metteva prima in relazione fra di loro attraverso altre relationship ausiliarie. Nello specifico,
  vediamo come usare la reificazione di relationship in entità per reificare relationship binarie e ternarie:
  \begin{itemize}
    \item \textbf{Reificazione di relationship binarie} \\
     Diciamo di avere un modello abbastanza semplice, contenente due entità con identificatore interno legate da una relationship:
      Magari la relationship potrebbe avere bisogno di più informazioni specifiche: potremo allora usare più approcci alternativi. 
      Il primo sarà di creare un entità corrispondente alla relationship, e di rendere poi gli attributi di quell'entità e le relazioni
      con le entità di partenza identificatore esterno. Un'altro approccio sarà quello di assegnare semplicemente alla nuova entità un identificatore
      univoco.
    \item \textbf{Reificazione di relationship ternarie} \\
      Un problema più sostanziale può essere rappresentato da una relationship ternaria. In questo caso si può adottare un approccio analogo
      al precedente, trasformando la relationship in un entità contenente tutte le informazioni necessario e legata alle entità di partenza con
      3 distinte relationship, che potranno anche formare l'identificatore esterno della nuova entità.
    \item \textbf{Reificazione di attributo di relationship}\\
      Nel modello ER, nessuno ci impedisce di assegnare attributi alle relationship. Questo però andrà molto probabilmente
      tradotto in un entità a sé stante, che sarà legata attraverso le metodologie che abbiamo appena visto alle entità che legava in partenza.
  \end{itemize}
\item \textbf{"Parte di"} \\
    Il pattern parte di permette di definire concetti che sono parte di altri concetti, come ad esempio un cinema con le sue sale.
    Viene rappresentato da una relazione "composizione", uno a molti, dove l'oggetto che è parte dell'altro ha identificatore esterno
    nel primo.
  \item \textbf{"Istanza di"} \\
    Il pattern istanza di permette di definire concetti che sono istanze di altri concetti generali, come ad esempio un torneo e le sue edizioni.
    In questo caso si usa sempre una relazione, "occorrenza", uno a molti, dove l'oggetto occorrenza ha identificatore
    esterno nell'oggetto generale.
  \item \textbf{Caso particolare di entità} \\
    Il pattern caso particolare di entità permette definire casi particolari di determinate entità, attraverso la generalizzazione totale di un entità
    specializzata su un'entita generale. L'entità specializzata sarà il caso particolare.
\end{itemize}
Grafici di modelli ER per i pattern riportati si possono trovare al link:
\url{https://github.com/Guray00/IngegneriaInformatica/blob/master/PRIMO%20ANNO/II%20SEMESTRE/Basi%20di%20dati/Diapositive/Slides%20Tonellotto/A.A.%2022-23/%5B06%5D%20Progettazione%20Concettuale%20e%20Logica.pdf}
\par\smallskip
\textbf{Documentazione associata agli schemi concettuali} \\
Il modello ER non basta mai da solo a descrivere l'interezza della realtà che vogliamo modellare. Esistono infatti, nonostante tutte
le aggiunte che abbiamo apportato al modello ER, vincoli inesprimibili, che dovranno essere espressi dalla documentazione di supporto.
Serviranno quindi:
\begin{itemize}
  \item Un \textbf{dizionario dei dati}, ovvero una tabella che contenga informazioni riguardo alle entità e alle relationship
    che dovranno essere modellizzate dal nostro database.
  \item \textbf{Regole aziendali}: ovvero informazioni riguardo ai vincoli di integrità necessarie, nonché possibili derivazioni
    specifiche ai tipi di dati che staremo trattando (formule particolari per il calcolo di indici, medie...).
\end{itemize}
\end{document}
