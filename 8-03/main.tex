\documentclass[a4paper,12pt]{article}

\usepackage[french,italian]{babel}
\usepackage[T1]{fontenc}
\usepackage[utf8]{inputenc}
\frenchspacing 
\title{Appunti Basi di Dati}
\author{Luca Seggiani}
\date{8 Marzo 2024}

\begin{document}
\maketitle
\section{Correttezza di basi di dati}
Una base di dati può essere scorretta, ovvero immagazinare dati incompatibili con la realtà dei fatti. Ad esempio,
sarebbe impensabile avere una base di dati contenente voti relativi ad esami universitari, e trovare in tale base di
dati la valutazione "27 e lode". La correttezza dei dati può essere assicurata grazie al sistema dei vincoli:
\section{Vincoli di integrità}
I vincoli di integrità sono sostanzialmente funzioni booleane (predicati) che associano alla istanza completa
di basi dati un valore vero o falso. Se il vincolo di una base di dati restituisce vero, significa che le sue
proprietà sono soddisfatte. Possiamo distinguere i vincoli in:
\begin{itemize}
  \item \textbf{Vincoli intra-relazionali}:
    i vincoli intra-relazionali sono definiti rispetto ad una singola relazione. Possiamo portare gli esempi:
    \begin{itemize}
      \item Vincolo di n-upla: può essere valutata su ciascuna n-upla indipendentemente dalle altre
      \item Vincolo di dominio: vincolo di n-upla che coinvolge un solo attributo.
    \end{itemize}
  \item \textbf{Vincoli inter-relazionali}:
    i vincoli inter-relazionali sono definiti rispetto a più relazioni diverse fra loro.
\end{itemize}

\section{Chiavi}
Una chiave è un insieme di attributi che identifica univocamente le n-uple. Formalmente:
una \textbf{superchiave} è un insieme $K$ di attributi di un insieme $r$ se $r$ non contiene due n-uple distinte
$t_1$ e $t_2$ tali che $t_1[K] = t_2[K]$. Una chiave K è tale per un insieme $r$ se è una superchiave minimale
in $R$ (cioè non contiene un'altra superchiave). N.B.: non per forza una chiave è unica.
\par\smallskip
Qualsiasi relazione ammette sempre almeno una chiave, in quanto una relazione contiene n-uple tutte diverse fra
loro, ergo tutto l'insieme degli attributi forma sempre una chiave. \\
L'esistenza di una chiave garantisce l'accessibilità a ciascun dato della base di dati, e permette di correlare
fra loro dati in relazoni diverse. Le chiavi formano inoltre un vincolo di integrità, detto appunto vincolo di chiave. \\
Nel caso una relazione abbia più di una chiave, ne scegliamo una in particolare, la chiave primaria (\textit{primary key}). Le altre
chiavi verranno allora dette chiavi candidate.
\par\smallskip
\textbf{Chiavi e valori nulli} \\
La presenza di valori nulli nelle chiavi può rappresentare un grande problema: rende impossibile il riferimento
da parte di altre relazioni e non permette l'identificazione univoca di ogni record. Va dunque evitato quando possibile.
\par\smallskip
\textbf{Dipendenze Funzionali} \\
I vincoli di chiave sono particolari tipo di vincoli, che fanno parte di una categoria puù vasta: le \textbf{dipendenze funzionali}.
Formalmente: dati due insiemi di attributi $X$ e $Y$ sulla relazione $R$, si dice che $X$ determina $Y$, e si scrive
$X \rightarrow Y$, se e solo se per ogni coppia di n-uple distinte $t_1$ e $t_2$, se $t_1[X] = t_2[X]$, allora $t_1[Y] = t_2[Y]$.

\section{Integrità referenziali}
Informazioni in relazioni diverse possono essere correlate attraverso valori comuni. Definiamo allora il concetto di chiave
esterna (\textit{foreign key}), ovvero un'insieme di attributi di una relazione che corrispondono ad una chiave di un'altra
relazione.
\par\smallskip
\textbf{Vincolo di integrità referenziale} \\
Un vincolo di integrità referenziale fra gli attributi $X$ di una relazione $R_1$ (chiave esterna di $R_1$) e
un'altra relazione $R_2$ impone ai valori su $X$ di $R_1$ di comparire come valori della chiave primaria di $R_2$. N.B.: in
questo caso l'ordine dei valori è significativo.
\par\smallskip
\textbf{Integrità referenziale e valori nulli} \\
In presenza di valori nulli i vincoli possono essere resi meno restrittivi: scegliamo da $R_1$ solamente
i valori di $X$ diversi da NULL.

\section{Operazioni di aggiornamento}
Le operazioni di aggiornamento di una relazione consistono in sostanza di:
\begin{itemize}
  \item Operazioni di inserimento, può violare:
    \begin{itemize}
      \item vincoli intra-relazionali
      \item integrità referenziale
    \end{itemize}
  \item Operazione di cancellazione, può violare:
    \begin{itemize}
      \item integrità referenziale
    \end{itemize}
  \item Operazione di modifica (cancellazione + inserimento), di norma non viola nulla.
\end{itemize}

La correttezza di queste operazioni è assicurata dal DMBS, che reagirà alle violazioni dei vincoli attraverso azioni compensative.
\par\smallskip
\textbf{Azioni compensative}
Nel caso si provi ad eliminare una n-upla causando violazionzi del vincolo di chiave, il comportamento standard potrebbe essere
quello del rifiuto dell operazione. Si potrebbero sennò adottare altre azioni compensative, come ad esempio:
\begin{itemize}
  \item Eliminazione in cascata
  \item Introduzione di valori nulli
\end{itemize}

\end{document}
