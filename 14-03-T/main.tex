\documentclass[a4paper,12pt]{article}

\usepackage[french,italian]{babel}
\usepackage[T1]{fontenc}
\usepackage[utf8]{inputenc}
\frenchspacing 
\title{Appunti Basi di Dati}
\author{Luca Seggiani}
\date{14 Marzo 2024}

\usepackage{forest}

\begin{document}
\maketitle
Tutti gli operatori visti finora, ovvero gli operatori insiemistici e quelli relazionali, bastano a realizzare
qualsiasi possibile interrogazione. Ogni altro operatore sarà un' operatore derivato dei 5 operatori relazionali
e i 3 poeratori insiemistici. Un particolare operatore derivato è:
\section{Divisione}
Dati due insiemi di attributi disgiunti $X_1$ e $X_2$, una relazione $r$ sulla loro unione e una relazione
$r_2$ su $X_2$, la divisione $r \div r_2$ è una relazione su $X_1$ che contiene le n-uple ottenute 
come "proiezione" di n-uple di $r$ che si combinano con tutte le n-uple di $r_2$ per formare n-uple di $r$, in una 
sorta di prodotto cartesiano inverso. In simboli:
$$ r \div r_2 = \{ t_1[X_1] \quad | \quad \forall t_2 \in r_2 \quad \exists t \in r : t[X_1] = t_1, \quad t[X_2] = t_2 \}$$
possiamo dimostrare che è un operatore derivato definendolo come composizione di operatori fondamentali, in questo modo:
$$ r \div r_2 = \pi_{X_1}(r) - \pi_{X_1}((\pi_{X_1}(r)\times r_2)-r)$$
Ovvero:
\begin{itemize}
  \item $ \pi_{X_1}(r) \times r_2 $ contiene tutte le n-uple di $\pi_{X_1}(r)$ "estese" con tutti i possibili valori
    di $r_2$.
  \item $(\pi_{X_1}(r)\times r_2)-r$ contiene le "estensioni" di $\pi_{X_1}(r)$ che non compaiono in $r$.
  \item $\pi_{X_1}((\pi_{X_1}(r)\times r_2)-r)$ contiene le n-uple di $\pi_{X_1}(r)$ per le quali un qualche
    completamento con $r_2$ non compare in $r$.
  \item Togliendo queste ultime n-uple a $\pi_{X_1}(r)$  otteniamo tutte le n-uple di $\pi_{X_1}(r)$ che si combinano con tutte
    le n-uple di $r_2$.
\end{itemize}
\section{Chiusura transitiva}
Poniamoci il problema di dover trovare, in un'opportuna tabella supervisione formata da matricole di impiegati e supervisori di tali impiegati,
le matricole di tutti i supervisori di un dato impiegato (ammettendo che i supervisori siano impiegati e possano a loro volta
avere supervisori). Tale richiesta sarebbe perfettamente valida, ma impossibile da esprimere attraverso gli operatori dell'algebra relazionale. \\
Nell'algebra relazionale non esiste la possibilità di esprimere interrogazioni che calcolino la chiusura transitiva di una relazione
arbitraria. Tale operazione potrebbe infatti richiedere un numero infinito di di join (join illimitato). 
\section{Espressioni equivalenti} 
Due espressioni sono equivalenti se producono lo stesso risultato qualunque sia l'istanza fornitagli. In questo caso, sarà opportuno
scegliere espressioni di costo minore, dove il loro "costo" è determinato dalle dimensioni delle istanze intermedie
che la loro esecuzione genera. Vediamo alcune equivalenze fondamentali:
\begin{itemize}
  \item \textbf{Atomizzazione delle selezioni}: una congiunzione di selezioni può essere sostituita da una sequenza
    di selezioni atomiche:
    $$ \sigma_{F_1 \land F_2}(E) = \sigma_{F_1}(\sigma{F_2}(E))$$
  \item \textbf{Idempotenza delle proiezoni}: una proiezione può essere trasformata in una sequenza di proiezioni:
    $$ \pi_{X}(E) = \pi_{X}(\pi_{XY}(E))$$
  \item \textbf{Push selections down}: se una condizione $F$ coinvolge solo attributi dell'espressione $E_2$:
    $$ \sigma_{F}(E_1 \bowtie E_2) = E_1 \bowtie \sigma_{F}(E_2) $$
  \item \textbf{Push projections down}: se un'espressione $E_1$ ha attributi $X_1$, un'espressione $E_2$ ha attributi
    $X_2$, $Y_2 \subseteq X_2$ e gli attributi $X_2 - Y_2$ non sono coinvolti nel join ($X_1 \cap X_2 \subseteq Y_2$):
    $$ \pi_{X_1Y_2}(E_1 \bowtie E_2) = E_1 \bowtie \pi_{Y_2}(E_2) $$
\end{itemize}
\section{Ottimizzazione delle interrogazioni}
Un modulo presente nel DBMS è il \textbf{query processor} (od ottimizzatore). L'ottimizzatore si occupa di scegliere
la strategia realizzativa a partire dall'istruzione in linguaggio dichiarativo di alto livello, tenendo conto
del costo di implementazioni diverse. Le fasi di eseecuzione di una query saranno:
\begin{itemize}
  \item Analisi lessicale, sintattica e semantica della query in linguaggio di alto livello (SQL). Al termine
    di questa analisi, la query verrà tradotta in un'espressione dell'algebra relazionale.
  \item Ottimizzazione algebraica: a questo punto verranno calcolate una o più nuove espressioni algebriche
    equivalenti a quella di partenza, sfruttando le equivalenza sopra descritte.
  \item Ottimizzazoine basata sui costi: fra le alternative calcolate prima, viene selezionata la più efficiente,
    che viene poi utilizzata per effettuare l'interrogazione effettiva sulla base di dati.
\end{itemize}
Nella prima e l'ultima di queste fasi, il DMBS interagisce con un componente detto catalogo, che contiene informazioni
sugli schemi contenuti nel database, la cardinalità delle loro istanze, ecc..
\par\medskip
\textbf{Profili delle relazioni} \\
Tra le informazioni quantitative memorizzate nel catalogo troviamo:
\begin{itemize}
  \item Cardinalità di ciascuna relazione;
  \item Dimensioni delle n-uple;
  \item Dimensioni dei valori;
  \item Numero di valori distinti degli attributi;
  \item Valore minimo e massimo degli attributi.
\end{itemize}
Queste informazioni vengono usate nella fase di ottimizzazione basata sui costi per stimare le dimensioni dei risultati
intermedi di più espressioni algebriche alternative generate dall'ottimizzazione algebrica.
\par\smallskip
\textbf{Ottimizzazione algebrica} \\
In verità, il termine ottimizzazione non è completamente accurato: il processo utilizza infatti delle euristiche per
trovare risultati migliori. Si basa sul concetto di equivalenza per trovare query che restituiscono lo stesso
risultato generando dimensioni d'istanza intermedie minori. Ad esempio, un'ottimizzazione tipica è quella
di eseguire selezioni e proiezioni il più presto possibile, ovvero le cosiddette "push selections down" e "push
projections down".
\section{Grafi}
Un grafo $G=(V,E)$ consiste in un insieme $V$ di vertici (o nodi) e un insieme $E$ di coppie di vertici, detti archi.
Ogni arco ovviamente connetter fra loro due vertici. Possiamo allora distinguere:
\begin{itemize}
  \item \textbf{Grafi orientati}: detti anche grafi diretti, dove ogni arco è orientato è rappresenta relazioni ordinate fra oggetti;
  \item \textbf{Grafi non orientati}: detti anche grafi indiretti, dove ogni arco non è orientato e rappresenta relazioni simmetriche fra oggetti.
\end{itemize}
Sui grafi si possono definire \textbf{cammini} da un vertice $x$ ad un vertice $y$. Formalmente, un cammino è:
$$ (v_0, ... , v_k) \quad \mathrm{di} \quad V, \quad v_0 = x, \quad v_k = y \quad | \quad 1 \leq i \leq k : (v_{i-1}, v_i) \in E$$
Un cammino che torna da dove parte, ovvero dove $(v_0, ... , v_k) : v_0 = v_k$, è detto ciclico. Si dice che un grafo
diretto privo di cicli è \textbf{aciclico}.
\par\smallskip
\textbf{Alberi} \\
Un grafo non orientato si dice connesso se esiste un cammino fra ogni coppia di vertici. Un'albero è un grafo non orientato
nel quale due vertici qualsiasi sono connessi da uno e un solo cammino.
\par\medskip
Un'interrogazione può essere rappresentata da un'albero, dove le foglie (i nodi finali) sono dati (relazioni, file, ecc...),
e i nodi intermedi sono operatori (operatori algebrici, poi effettivi operatori di accesso ai dati). Ad esempio,
l'albero corrispondente all'espressione:
$$ \sigma_{A=10}(R_1 \bowtie R_2) $$
sarà:
\begin{center}
\begin{forest}
  [$\sigma_{A=10}$
    [
      $\bowtie$
      [$R_1$]
      [$R_2$]
    ]
  ]
\end{forest}
\end{center}
La stessa interrogazione dopo il push down della selezione $\sigma_{A=10}$ sarà allora:
$$ R_1 \bowtie \sigma{A = 10}(R_2)$$
con relativo albero:
\begin{center}
\begin{forest}
    [
      $\bowtie$
      [$R_1$]
      [$\sigma_{A=10}$
      [$R_2$]
      ]
    ]
\end{forest}
\end{center}

\par\smallskip
\textbf{Procedura euristica di ottimizzazione} \\
La procedura di ottimizzazione consiste quindi nel:
\begin{itemize}
  \item Decomporre le selezioni congiuntive in successive selezioni atomiche;
  \item Anticipare il più possibile le selezioni;
  \item In una sequenza di selezioni, anticipare le più selettive;
  \item Combinare prodotti cartesiani e selezioni per formare join;
  \item Anticipare il più possibile le proiezioni.
\end{itemize} 

\end{document}
