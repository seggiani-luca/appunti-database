\documentclass[a4paper,12pt]{article}

\usepackage[french,italian]{babel}
\usepackage[T1]{fontenc}
\usepackage[utf8]{inputenc}
\frenchspacing 
\title{Appunti Basi di Dati}
\author{Luca Seggiani}
\date{8 Maggio 2024}

\usepackage{listings}
\usepackage{xcolor}

\definecolor{codegreen}{rgb}{0,0.6,0}
\definecolor{codegray}{rgb}{0.5,0.5,0.5}
\definecolor{codepurple}{rgb}{0.58,0,0.82}
\definecolor{backcolour}{rgb}{0.95,0.95,0.92}

\lstdefinestyle{code-style}{
    backgroundcolor=\color{backcolour},   
    commentstyle=\color{codegreen},
    keywordstyle=\color{magenta},
    numberstyle=\tiny\color{codegray},
    stringstyle=\color{codepurple},
    basicstyle=\ttfamily\footnotesize,
    breakatwhitespace=false,         
    breaklines=true,                 
    captionpos=b,                    
    keepspaces=true,                 
    numbers=left,                    
    numbersep=5pt,                  
    showspaces=false,                
    showstringspaces=false,
    showtabs=false,                  
    tabsize=2
}

\lstset{style=code-style}

\begin{document}
\maketitle
Calcolare $F^+$ è molto costoso: l'algoritmo è a complessità esponenziale. Spesso però, quello che ci
interessa beramente e verificare se $F^+$ contiene una certa dipendenza e non generare l'intera lista
di chiusura. Per fare ciò basta calcolare $X^+$ per il teorema di chisura degli attributi:
$$ F \vdash X \rightarrow Y \Leftrightarrow Y \subseteq X^+ $$
\par\smallskip
\textbf{Algoritmo per il calcolo di $\mathbf{X}^+$} \\
Vediamo quindi come calcolare la chisurua $X^+$:
\begin{itemize}
  \item Input: $R(T,F), \quad X \subseteq T$;
  \item Output: $X^+$ \textit{chiusura}.
\end{itemize}
\begin{lstlisting}[language=perl]
metti X in X+
while (X+ non cambia) do
  foreach W -> V in F do
    if W in X+ and V not in X+ then
      metti V in X+
return X+
\end{lstlisting}
\par\smallskip
\textbf{Chiavi} \\
Dato uno schema $R(T,F)$, un'insieme di attributi $K \subseteq T$ si dice \textbf{superchiave} di $R$ se la dipendenza funzionale
$K \rightarrow T$ è implicata da $F$, ovvero se $K \rightarrow T \in F^+$. Un'insieme di attributi $K \subseteq T$ si dice a questo punto \textbf{chiave}
di $R$ se $K$ è una superchiave di $R$ e non esiste alcun sottoinsieme proprio di $K$ che sia superchiave di $R$.
Dato che in uno schema possono esserci più chiavi, di solito si identifica una chiave primaria che possa fare da identificatore
per tutte le n-uple dello schema. Tutte le altre chiavi si dicono chiavi candidate.
\par\smallskip
\textbf{Algoritmo per tutte le chavi} \\
Il problema di trovare tutte le chiavi di una relazione $R(Z)$ ha complessità esponenziale nel caso peggiore:
\begin{itemize}
  \item Gli attributi che stanno solo a sinistra sono in tutte le chiavi. Si chiami $N$ questo insieme;
  \item Gli attributi che stanno solo a destra non sono in nessuna chiave. Si chiami $M$ questo insieme;
  \item Si aggiunge a $N$ un'attributo alla volta tra quelli che non sono né in $N$ nè in $M$, poi una coppia di attributi,
    e così via. Chiamiamo $X_i$ questo sottoinsieme di attributi: ad ogni aggiunta controlleremo se la dipendenza $N \cup X_i \rightarrow Z$ esiste.
\end{itemize}
\par\smallskip
\textbf{Verifica di una chiave} \\
Spesso è molto più conveniente verificare se una chiave è tale, piuttosto che trovare ogni possibile chiave. Per fare ciò,
possiamo usare l'algoritmo per il calcolo della chiusura di un'insieme di attributi:
\begin{itemize}
  \item $X\subseteq T$ è superchiave di $R(T,F)$ se e solo se $X \rightarrow T \in F^+$, ovvero $T\subseteq X^+$
  \item $X\subseteq T$ è chiave di $R(T, F)$ se e solo se $T \subseteq X^+$, e non esiste $Y \subset X$ tale che
    $T \subseteq Y^+$.
\end{itemize}
\par\smallskip
\textbf{Equivalenza} \\
Due insiemi di dipendenze funzionali $F$ e $G$ sugli attributi $T$ di una relazione $R(T)$ sono equivalenti, in simboli $T \equiv G$, se e solo
se $F^+ = G^+$. La relazione di equivalenza permette di stabilire se due insiemi di dipendenza rappresentano gli stessi fatti. Per verificare
l'equivalenza è sufficiente che:
\begin{itemize}
  \item Tutte le dipendenze di $F$ appartengano a $G^+$;
  \item Tutte le dipendenze di $G$ appartengano a $F^+$.
\end{itemize}
\par\smallskip
\textbf{Ridondanza} \\
Sia $F$ un insieme di dipendenze funzionali. Data $X\rightarrow Y \in F$, $X$ contiene un \textbf{attributo estraneo} se e solo se:
$$ (F - \{X\rightarrow Y\}) \cup (X - \{A\rightarrow\} Y) \equiv F$$
ovvero $X - \{A\} \rightarrow Y \in F^+$ \\ 
$X\rightarrow Y$ è una \textbf{dipendenza ridondante} se e solo se $(F - \{X \rightarrow Y\}) \equiv F$, ovvero $X\rightarrow Y \in (F - \{X\rightarrow Y\})^+$.
Le dipendenze che non contengono attributi estranei e la cui parte destra è un unico attributo sono dette \textbf{dipendenze elementari}.
\par\smallskip
\textbf{Copertura minimale} \\
Sia $F$ un'insieme di dipendenze funzionali. $F$ è una \textbf{copertura minimale} (detta anche \textit{insieme minimale} e \textit{copertura canonica}) se e solo se:
\begin{itemize}
  \item Ogni parte destra di una dipendenza ha un unico attributo;
  \item Le dipendenze non contengono attributi estranei;
  \item Non esistono dipendenze ridondanti;
\end{itemize}
In soldoni, la copertura minimale rappresenta l'insieme di dipendenze funzionali $F'$ più piccolo che implica tutte le dipendenze di $F$.
\par\smallskip
\textbf{Algoritmo per il calcolo della copertura minimale}
\begin{itemize}
  \item Input: F;
  \item Output G \textit{copertura minimale}:
\end{itemize}
\begin{lstlisting}[language=perl]
metti F in G
for each X -> G do
  metti X in Z
  for each A in X Do
    if Y in (Z - {A})+_F then
      metti Z - {A} in Z
    metti (G - {X -> Y}) U (Z -> Y) in G
  foreach F in G do
    if f in (G-{f})+ then
      metti G - {f} in G
return G
\end{lstlisting}
\par\smallskip
\textbf{Teorema della copertura minimale} \\
Il precedente algoritmo dimostra il fatto che per ogni insieme di dipendenze funzionalil $F$ esiste una copertura minimale.
Si noti che questo non afferma che la copertura minimale è unica: possono tranquillamente esistere coperture minimali equivalenti.
\section{Normalizzazione}
\par\smallskip
\textbf{Eliminare le anomalie} \\
La teoria che abbiamo sviluppato finora viene usata per identificare le anomalie in uno schema mal definito. Definiamo quindi il concetto di \textbf{forma normale}:
una forma normale è una proprietà che deve essere soddisfatta dalla dipendenza fra attibuti di schemi "ben fatti". Noi vedremo due esempi:
la \textbf{forma normale di Boyce-Codd}, e un suo miglioramento, la \textbf{terza forma normale}. 
\par\smallskip
\textbf{Forma normale di Boyce-Codd} \\
Uno schema $R(T,F)$ è in forma normale di Boyce-Codd (BCNF) se e solo se per ogni dipendenza funzionale non banale
$ X \rightarrow Y \in F^+ $, $X$ è superchiave di $R$. Per definizione, il fatto che uno schema sia o meno in BCNF dipende
dalla chiusura $F^+$, non dalla specifica copertura $F$. Calcolare $F^+$, come abbiamo visto, ha complessità esponenziale:
fortunatamente esiste un'algoritmo di complessità polinomiale per valutare se uno schema è in forma BCNF. Si usa il corollario:
uno schema $R(T,F)$ con $F$ copertura minimale è in BCNF se e solo se per ogni dipendenza funzionale elementare $X \rightarrow A \in F$, 
$X$ è superchiave.
\begin{itemize}
  \item Input: $R(T,F)$;
  \item Output true se $R$ è in BCNF, false altrimenti. 
\end{itemize}
\begin{lstlisting}[language=perl]
for each X -> Y in F do
  if Y not in X and T not in X+ then
    return false
return true
\end{lstlisting}
\end{document}
