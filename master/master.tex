\documentclass[a4paper,12pt]{article}

\usepackage[french,italian]{babel}
\usepackage[T1]{fontenc}
\usepackage[utf8]{inputenc}
\frenchspacing 
\title{Basi di Dati}
\author{Luca Seggiani}
\date{\today}

\usepackage{amsmath}
\usepackage{gensymb}
\usepackage{forest}
\usepackage{hyperref}

\usepackage{listings}
\usepackage{xcolor}

\definecolor{codegreen}{rgb}{0,0.6,0}
\definecolor{codegray}{rgb}{0.5,0.5,0.5}
\definecolor{codepurple}{rgb}{0.58,0,0.82}
\definecolor{backcolour}{rgb}{0.95,0.95,0.92}

\usepackage{tikz}
\usetikzlibrary{shapes.geometric, shapes.arrows}

\tikzstyle{entity} = [rectangle, text centered, minimum width=3cm, minimum height=1.5cm , draw=black, fill=white]
\tikzstyle{relationship} = [diamond, text centered, draw=black, fill=white]
\tikzstyle{arrow} = [->, >=stealth]
\tikzstyle{connector} = [-, >=stealth]
\tikzstyle{filledarrow} = [single arrow, draw=black, fill=black, rotate=90, minimum height=0.7cm]
\tikzstyle{emptyarrow} = [single arrow, draw=black, fill=white, rotate=90, minimum height=0.7cm]

\lstdefinestyle{code-style}{
    backgroundcolor=\color{backcolour},   
    commentstyle=\color{codegreen},
    keywordstyle=\color{magenta},
    numberstyle=\tiny\color{codegray},
    stringstyle=\color{codepurple},
    basicstyle=\ttfamily\footnotesize,
    breakatwhitespace=false,         
    breaklines=true,                 
    captionpos=b,                    
    keepspaces=true,                 
    numbers=left,                    
    numbersep=5pt,                  
    showspaces=false,                
    showstringspaces=false,
    showtabs=false,                  
    tabsize=2
}

\lstset{style=code-style}

\usepackage{standalone}

\begin{document}
\maketitle
\documentclass[a4paper,12pt]{article}

\usepackage[french,italian]{babel}
\usepackage[T1]{fontenc}
\usepackage[utf8]{inputenc}
\frenchspacing 
\title{Appunti Basi di Dati}
\author{Luca Seggiani}
\date{28 Febbraio 2024}

\usepackage{listings}
\usepackage{xcolor}

\definecolor{codegreen}{rgb}{0,0.6,0}
\definecolor{codegray}{rgb}{0.5,0.5,0.5}
\definecolor{codepurple}{rgb}{0.58,0,0.82}
\definecolor{backcolour}{rgb}{0.95,0.95,0.92}

\lstdefinestyle{code-style}{
    backgroundcolor=\color{backcolour},   
    commentstyle=\color{codegreen},
    keywordstyle=\color{magenta},
    numberstyle=\tiny\color{codegray},
    stringstyle=\color{codepurple},
    basicstyle=\ttfamily\footnotesize,
    breakatwhitespace=false,         
    breaklines=true,                 
    captionpos=b,                    
    keepspaces=true,                 
    numbers=left,                    
    numbersep=5pt,                  
    showspaces=false,                
    showstringspaces=false,
    showtabs=false,                  
    tabsize=2
}

\lstset{style=code-style}

\begin{document}
\maketitle
\section{Introduzione alla base di dati relazionale}
Una base di dati (in inglese "database") è una mole di dati organizzata in modo da favorirne la consultazione
attraverso determinate interrogazioni ("query"). Formalmente, una base di datiè un archivio di dati contenente
informazioni ben strutturate organizzate secondo un modello logico (nel caso di MySQL un modello relazionale).
\par \smallskip
L'elemento fondamentale di una base di dati fondata sul modello relazionale è la \textbf{Tabella}, formata
da colonne (associate a determinati attributi), e righe (che rappresentano record dei suddetti attributi).
Una colonna particolare, che deve differire per ogni record della tabella, viene detta chiave primaria e serve
a distinguere tra di loro record (che potrebbero essere altrimente stati tra di loro identici!).
\par \smallskip
L'interrogazione della base di dati si effettua attraverso le cosiddette query, ovvero richieste compilate in
linguaggio SQL (Structured Query Language), un linguaggio dichiarativo sviluppato appositamente per questo
scopo. La query più semplice si basa su tre istruzioni: SELECT, FROM e WHERE. Prendiamo l'esempio in lingua
naturale di un'interrogazione a una base di dati contenente dati su diverse persone (nello specifico, nome,
cognome, età e codice fiscale come chiave.):

\begin{center}
  \textit{Riportare nomi e cognomi di tutte le persone con eta maggiore di 40 anni.}
\end{center}

in questo caso gli attributi di interesse sono i nomi e cognomi (in quest'ordine) delle persone valide, e la 
condizione è un età maggiore di 40 anni. Di fronte a questa richiesta, la base di dati risponderà fornendo
un insieme risultato (result set) contenente una tabella formata esattamente dalle informazioni richieste.
In SQL potremmo definire tale query come segue:

\begin{lstlisting}[language=SQL]
SELECT Cognome, Nome
FROM Persona
WHERE Eta > 40
\end{lstlisting}

in questo caso, cognome e nome sono gli attributi di interesse che verrano inseriti nel nostro insieme risultato
(attraverso un processo chiamato proiezione). La tabella persona sarà la fonte dei nostri dati, e età > 40 la 
nostra condizione.
\par \smallskip

Le condizioni in SQL supportano anche i classici operatori logici (AND, OR, ecc...). Un operatore degno di nota,
applicabile ai soli dato di tipo numerico o su cui è comunque stabilita una relazione d'ordine, è il BETWEEN,
che permette di selezionare (estremi inclusi) tutti i valori in un dato range.
\par \medskip
\textbf{Duplicati}
2 righe risultano uguali se tutti gli attributi del record hanno valori identici. Non si possono avere righe
uguali in tabelle SQL (la chiave distingue), ma è possibile avere duplicati tra gli attributi non chiave, con
quindi risultanti problemi dati dalla proiezione di tali attributi su insiemi risultato privi di attributo chiave.
Nel caso si rendesse necessario eliminare i duplicati da una data query, l'SQL offre il costrutto SELECT DISTINCT.
Si nota che il costo per l'eliminazione dei duplicaty è computazionalmente $O(n^2)$, e quindi da evitare quando
possibile.
\par \medskip
\textbf{Valori NULL}
I valori NULL corrispondono solitamente a informazioni mancanti, non disponibili o comunque sconosciute.
Qualsiasi condizione che coinvolge valori NULL è sempre falsa, compreso il confronto fra stessi NULL.
È possibile utilizzare i valori NULL assegnandoli un valore semanticamente definito (e.g. una data non definita
perchè non ancora determinata, ecc...). In questo caso, l'SQL offre i costrutti IS NOT NULL e IS NULL, che restituiscono
rispettivamente falso e vero posti accanto ad un valore NULL. Si nota inoltre che la chiave di un record non
può essere NULL.
\par \medskip
\textbf{Gestione delle date}
Come per tutti i sistemi UNIX, le date in SQL vengono rappresentate come una distanza in secondi dalla mezzanotte
del primo gennaio 1970. 

\end{document}

\documentclass[a4paper,12pt]{article}

\usepackage[french,italian]{babel}
\usepackage[T1]{fontenc}
\usepackage[utf8]{inputenc}
\frenchspacing 
\title{Appunti Basi di Dati}
\author{Luca Seggiani}
\date{29 Febbraio 2024}

\usepackage{listings}
\usepackage{xcolor}
\usepackage{hyperref}

\definecolor{codegreen}{rgb}{0,0.6,0}
\definecolor{codegray}{rgb}{0.5,0.5,0.5}
\definecolor{codepurple}{rgb}{0.58,0,0.82}
\definecolor{backcolour}{rgb}{0.95,0.95,0.92}

\lstdefinestyle{code-style}{
    backgroundcolor=\color{backcolour},   
    commentstyle=\color{codegreen},
    keywordstyle=\color{magenta},
    numberstyle=\tiny\color{codegray},
    stringstyle=\color{codepurple},
    basicstyle=\ttfamily\footnotesize,
    breakatwhitespace=false,         
    breaklines=true,                 
    captionpos=b,                    
    keepspaces=true,                 
    numbers=left,                    
    numbersep=5pt,                  
    showspaces=false,                
    showstringspaces=false,
    showtabs=false,                  
    tabsize=2
}

\lstset{style=code-style}

\begin{document}
\maketitle
\textbf{Gestione delle date in SQL} \\
Come per tutti i sistemi UNIX, le date in SQL vengono rappresentate come una distanza in secondi dalla mezzanotte
del primo gennaio 1970 (il cosiddetto tempo di riferimento EPOCH, \textit{EPOCH reference}). 
\par\smallskip
\textbf{Formato di data}
Esistono 2 tipi di dato data in SQL: DATE (formato YYYY-MM-DD) e TIMESTAMP (formato YYYY-MM-DD HH:MM:SS). La
manipolazione di date è possibile attraverso la funzione DATE\_FORMAT, che consente di cambiare il formato o 
effettuare manipolazioni d'utilità. Ad esempio potrò avere:
\begin{lstlisting}[language=SQL]
SELECT Matricola, DATE_FORMAT(DataLaurea, '%d|%m|%Y, %w')
FROM Studente
WHERE DataLaurea > 12-7-2004
\end{lstlisting}
che prende tutti gli studenti laureati dopo il 12 luglio 2004 e ne riporta il nome di matricola e la data formattata come
DD-MM-YYYY + settimana. ancora, per includere ad esempio gli studenti laureati di mercoledì:
\begin{lstlisting}[language=SQL]
SELECT Matricola
FROM Studente
WHERE DATE_FORMAT(DataLaurea, '%w') = 3
\end{lstlisting}
dove il '\%w' restituisce il giorno della settimana di DataLaurea. Qualsiasi ulteriore informazione sui formati
  delle date può essere trovata al link: \\
  \url{https://www.w3schools.com/sql/func_mysql_date_format.asp} \\

Poniamoci adesso il problema di restituire i cognomi di tutti gli studenti laureati 5 anni fa, ovvero con
un offset di 5 anni rispetto alla data attuale. Servirà chiaramente un qualche riferimento alla data di oggi,
che l'SQL fornisce con la parola chiave CURRENT\_DATE. Potrò quindi avere:
\begin{lstlisting}[language=SQL]
SELECT DISTINCT Cognome
FROM Studente
WHERE DataLaurea IS NOT NULL
  AND YEAR(DataLaurea) = YEAR(CURRENT_DATE) - 5
\end{lstlisting}
si noti che abbiamo usato l'operatore "-": l'SQL fornisce anche la funzione DATEDIFF(dataRecente, dataRemota), 
che restituisce il numero di giorni che separano due date. Nota bene: somme e sottrazioni di date non hanno senso: tutto quello
che possiamo fare è sommare i singoli valori (anno, giorno, ecc...) di date diverse. Per qualsiasi altro caso
occorre usare DATEDIFF(). Possiamo avere ad esempio:
\begin{lstlisting}[language=SQL]
SELECT Matricola, DATEDIFF('2005-07-15', DataIscrizione)
FROM Studente
WHERE DataIscrizione < '2005-07-15'
  AND DataLaurea > '2005-07-15'
\end{lstlisting}
che restituirà la matricola e giorni da quando si erano iscritti degli studenti ad oggi laureati e che non si erano ancora laureati
il 15 Luglio 2005. \\
Inoltre, per sommare e sottrarre lassi di tempo a date possiamo usare le funzioni DATE\_ADD e DATE\_SUB. I suddetti
lassi andranno espressi con la parola chiave INTERVAL:
\begin{lstlisting}[language=SQL]
INTERVAL NumeroIntero [YEAR|MONTH|DAY]
\end{lstlisting}
riportiamo ad esempio matricola e mese di iscrizione degli studenti che si sono laureati dopo cinque anni esatti dal giorno
dell'iscrizione:
\begin{lstlisting}[language=SQL]
SELECT Matricola, MONTH(DataIscrizione)
FROM Studente
WHERE DataLaurea = DATE_ADD(DataIscrizione, INTERVAL 5 YEAR)
\end{lstlisting}
quando la somma è definita fra date e intervalli, possiamo usare anche solo l'operatore "+":
\begin{lstlisting}[language=SQL]
  AND DataLaurea = DataIscrizione + INTERVAL 5 YEAR
\end{lstlisting}
esistono poi diverse altre funzioni di utilità sulle date, che possono essere trovate nel sito precedentemente citato.
\par\medskip
\textbf{Operatori di aggregazione} \\
Gli operatori di aggregazione permettono di fare determinati calcoli i cui operandi sono i valori assunti
da un attributo in un insieme di record (ovvero conteggio, somma, minimo / massimo, media), e di collassarli in
un unico attributo numerico. Gli operatori di aggregazione sono disponibili solamente nel SELECT, in quanto il
WHERE non ha alcuna visibilità a livello globale della tabella su cui lavora, ma solo a livello di record.
\par\smallskip
\textbf{Conteggio} \\
Iniziamo col contare il numero di righe di una tabella o di un suo sottoinsieme. Definita
la tabella di una semplice realtà medica, abbiamo:
\begin{lstlisting}[language=SQL]
SELECT COUNT(*) AS VisitePrimoMarzo
FROM Visita
WHERE Data = '2013-03-01'
\end{lstlisting}
per trovare il numero di visite effettuate in data 1 marzo 2013. La parola chiave AS serve solamente per rinominare
la tabella generata dalla funzione COUNT(). Possiamo effettuare anche, anzichè il conteggio delle righe, il conteggio
dei valori diversi assunti da un attributo. Ad esempio, se cercassimo il numero di pazienti visitati nel mese di marzo
2013, visto che lo stesso paziente potrà essere stato visitato più volte in un solo mese, dovremmo fare:
\begin{lstlisting}[language=SQL]
SELECT COUNT(DISTINCT Paziente) as PazientiMarzo
FROM Visita
WHERE MONTH(Data) = '03'
  AND YEAR(Data) = '2013'
\end{lstlisting}
\par\medskip
\textbf{Somma} \\
Posso sommare i valori numerici di degli attributi di più record usando SUM(). Supponiamo di voler calcolare,
data una tabella del reddito di più persone, il reddito totale di una sola famiglia:
\begin{lstlisting}[language=SQL]
SELECT SUM(Reddito) AS RedditoTotale
FROM Paziente
WHERE Cognome = 'Lepre'
\end{lstlisting}
\par\medskip
\textbf{Media} \\
Allo stesso modo, potrò calcora il reddito medio:
\begin{lstlisting}[language=SQL]
SELECT AVG(Reddito) AS RedditoMedio
FROM Paziente
WHERE Cognome = 'Lepre'
\end{lstlisting}
\par\medskip
\textbf{Minimo / Massimo} \\
Potrò inoltre calcolare valori massimi e minimi:
\begin{lstlisting}[language=SQL]
SELECT MIN(Reddito)
FROM Paziente

SELECT MAX(Reddito)
FROM Paziente
\end{lstlisting}

A questo punto, se volessimo cercare il reddito massimo, e il nome e cognome di chi lo detiene, incontreremmo
un'ostacolo: non si può infatti ottenere un qualsiasi altro attributo da un insieme risultato ormai collassato
ad un solo valore numerico. Non otterremmo nulla dal codice:
\begin{lstlisting}[language=SQL]
SELECT MAX(Reddito), Nome, Cognome
FROM Paziente
\end{lstlisting}
Riassumendo: non si possono affiancare agli operatori di aggregazione i nomi di attributi
ormai collassati.
\par\smallskip
\textbf{Query su più tabelle} \\
In un database le informazioni sono spesso frammentate su più tabelle. Questo aiuta a evitare ridondanze, anomalie,
ed a vere la possibilità di distribuire i dati. Ad esempio, immaginiamo il database di una clinica medica, che
dovrà quindi immagazinare i dati di pazienti, dottori, visite mediche ecc... Potremo allora definire più tabelle separate
per ciascuna di queste categorie di dati, ognuna con i propri specifici attributi. A questo punto, ogni tabella
potrà contenere come attributi altre tabelle, o meglio record provienienti da altre tabelle in qualche modo
"collassati" nel singolo attributo della tabella. Questo meccanismo si concrettizza immagazzinando nella tabella
l'informazione minimale per ritrovare il record desiderato nella tabella d'appartenenza, ovvero riportandone soltanto
la chiave (che sappiamo essere diversa per ogni record archviato). La successiva esplosione della chiave fino al record
completo nella tabella d'appartenenza viene effettutato in SQL attraverso le operazioni di unione (join).
\par\medskip
\textbf{Inner join} \\
L'inner join trova il record nella tabella d'appartenenza che corrisponde alla chiave nella tabella su cui lavoriamo,
e semplicemente lo affianca. Ad esempio, magari vogliamo indicare nome e cognome dei medici che hanno effettuato,
nella nostra clinica, almeno una visita. Dovremmmo allora prendere la nostra tabella visita, che conterrà la chiave
di un medico nella tabella medico, che andremo quindi a sostituire con il record completo del medico corrispondente.
In codice:
\begin{lstlisting}[language=SQL]
SELECT DISTINCT M.Nome, M.Cognome
FROM Visita V
  INNER JOIN
  Medico M ON V.MEDICO = M.Matricola
\end{lstlisting}


\end{document}

\documentclass[a4paper,12pt]{article}

\usepackage[french,italian]{babel}
\usepackage[T1]{fontenc}
\usepackage[utf8]{inputenc}
\frenchspacing 
\title{Appunti Basi di Dati}
\author{Luca Seggiani}
\date{1 Marzo 2024}

\usepackage{listings}
\usepackage{xcolor}

\definecolor{codegreen}{rgb}{0,0.6,0}
\definecolor{codegray}{rgb}{0.5,0.5,0.5}
\definecolor{codepurple}{rgb}{0.58,0,0.82}
\definecolor{backcolour}{rgb}{0.95,0.95,0.92}

\lstdefinestyle{code-style}{
    backgroundcolor=\color{backcolour},   
    commentstyle=\color{codegreen},
    keywordstyle=\color{magenta},
    numberstyle=\tiny\color{codegray},
    stringstyle=\color{codepurple},
    basicstyle=\ttfamily\footnotesize,
    breakatwhitespace=false,         
    breaklines=true,                 
    captionpos=b,                    
    keepspaces=true,                 
    numbers=left,                    
    numbersep=5pt,                  
    showspaces=false,                
    showstringspaces=false,
    showtabs=false,                  
    tabsize=2
}

\lstset{style=code-style}

\begin{document}
\maketitle
Riprendiamo la trattazione delle operazioni che permettono l'unione (join) di più tabelle.
\par\smallskip
\textbf{Natural join} \\
Il join naturale combina i record della prima tabella con i record della seconda tabella aventi valori uguali 
su tutti gli attributi omonimi. Ad esempio, in pseudocodice, ammettendo che la tabella visita abbia un attributo
ominimo ad un'altro attributo sulla tabella medico:
\begin{lstlisting}[language=SQL]
  SELECT M.Nome, M.Cognome
  FROM Visita V
    NATURAL JOIN
    Medico M
\end{lstlisting}
Nota bene: \textbf{tutti} gli attributi omonimi dovranno avere valori identici, ed a quel punto il record
della tabella di provenienza verrà unito alla tabella su cui lavoriamo una sola volta.
\par\smallskip
\textbf{External join} \\
Il join esterno combina tutti i record della prima tabella con tutti i record della seconda che soddisfano
una condizione, mantenendo tutti i record in una delle tabelle, a differenza del join esterno che invece
scarterebbe i record non appaiati con nessuno dei record della tabella madre. Il join esterno può essere
pensato come una sorta di prodotto cartesiano fra tabelle. I record spaiati che verrano uniti avranno tutti
valori NULL sugli attributi della tabella di provenienza (inesistenti). Poniamo ad esempio di voler indicare le
visite effettuate da medici che non lavorano più presso la clinica. Supponiamo quindi che loro anagrafica
sia stata quindi eliminata dal database.
\begin{lstlisting}[language=SQL]
SELECT V.*
FROM Visita V
  LEFT OUTER JOIN
  Medico M ON V.Medico = M.Matricola
WHERE M.Matricola IS NULL;
\end{lstlisting}
Così facendo potremo unire tutti i record di tutti i medici, anche quelli che non lavorano più nella mia clinica.
Controllando a questo punto chi di questi medici ha elemento chiave (si controlla, per convenzione, proprio 
l'elemento chiave) NULL, otterremo le visite svolte dai suddetti. Notiamo infine la differenza fra join esterno
destro e join esterno sinistro: il join esterno sinistro mantiene tutti i record della tabella di sinistra mettendo
NULL a destra per i record non appaiati. Il join esterno destro fa la cosa opposta, quindi mantenendo i record a destra
e inserendo NULL a sinistra.
\par\smallskip
\textbf{Query con join e condizioni sui record} \\
Possiamo a questo punto combinare il FROM con instruzione di join allo WHERE con condizioni di selezione dei record
d'interesse applicate alla tabella \textit{dopo} il join. Ad esempio, se voglio ottenere matricola, cognome
e specializzazione dei medici che hanno visitato almeno un paziente il giorno 1 Marzo 2013
\begin{lstlisting}[language=SQL]
SELECT DISTINCT M.Matricola, M.Cognome, M.Specializzazione
FROM Medico M
  INNER JOIN
  Visita V ON M.Matricola = V.Medico
WHERE V.Data = '2013-01-03'
\end{lstlisting}

\textbf{Unioni multiple} \\
Il join multiplo può essere effettuato su più di 2 tabelle. Diciamo che oltre al medico, la nostra tabella visita
contiene un'attributo corrispondente al paziente visitato. Vogliamo allora ottenere nomi e cognomi di tutti i pazienti
visitati da un certo medico. Notiamo che il nome del medico sta nella sottotabella corrispondente al medico, e
il nome del paziente sta nella sottotabella corrispondente al paziente, ovvero dovremmo unire non una ma
due tabelle alla nostra tabella madre. In codice:
\begin{lstlisting}[language=SQL]
SELECT DISTINCT P.Nome, P.Cognome
FROM Paziente P
  INNER JOIN
  Visita V ON P.CodFiscale = V.Paziente
  INNER JOIN
  Medico M ON V.Medico = M.Matricola
WHERE M.Nome = 'Rino'
  AND M.Cognome = 'Neri';
\end{lstlisting}
notiamo che in questo caso, per assicurare la sicurezza dell'operazione, nome e cognome del medico dovrebbero
essere una cosiddetta \textit{chiave candidata}, ovvero un insieme di attributi sempre diversi in tutti i record
(come la chiave primaria ma senza lo status di chiave primaria).
\par\smallskip
Si noti inoltre di come gli alias (P V e M) nell'esempio precedente hanno il compito di evitare ambiguità su attributi
con lo stesso nome ma appartenenti a tabelle diverse. Ulteriore esempio, supponiamo di voler indicare nome e cognome
dei pazienti visitati  nel mese di dicembre 2013, e il nome e cognome dei medici che li hanno visitati:
\begin{lstlisting}[language=SQL]
SELECT DISTINCT P.Nome AS NomePaziente, P.Cognome AS CognomePaziente
                M.Nome AS NomeMedico, M.Cognome AS CognomeMedico
FROM Paziente P
  INNER JOIN
  Visita V ON P.CodFiscale = V.Paziente
  INNER JOIN
  Medico M ON V.Medico = M.Matricola
WHERE YEAR(V.Data) = 2013
  AND MONTH(V.Data) = 12
\end{lstlisting}

\textbf{Self join} \\
Il self join combina le righe di una tabella con righe della stessa tabella. Più propriamente, non esiste una 
keyword SELF JOIN, ma possiamo sfruttare le possibilità dell'INNER JOIN. Diciamo ad esempio di volere il codice fiscale dei
pazienti che sono stati visitati più di una volta da uno stesso medico della clinica, nel mese corrente:
\begin{lstlisting}[language=SQL]
SELECT DISTINCT V1.Paziente
FROM Visita V1
  INNER JOIN
  Visita v2 ON (
                V2.Medico = V1.Medico
                AND V2.Paziente = V1.Paziente
                AND V2.Data<> V1.Data
                )
WHERE MONTH(V1.Data) = MONTH(CURRENT_DATE)
  AND YEAR(V1.Data) = MONTH(CURRENT_DATE)
  AND MONTH(V2.Data) = MONTH(CURRENT_DATE)
  AND YEAR(V2.Date) = YEAR(CURRENT_DATE);
\end{lstlisting}
qui unisco alla mia tabella visita un record proveniente dalla stessa tabella quando medico, paziente corrispondono
e la data differisce. A questo punto seleziono soltanto i record della nuova tabella creata che hanno date diverse fra 
le due visite. Si noti che per n visite, il paziente comparirà nel result set n - 1 volte (da cui il DISTINCT), ovvero il
join accade in una sola direzione.
\par\medskip

\textbf{Common Table Expression}
Le common table expression (CTE) sono result set dotati di identificatori che possono essere usati prima di
una query per costruire risultati intermedi. Si scrivono, separate da virgole, prima della query che li usa,
tramite la parola chiave WITH e separati da virgole. Sono parti di codice i cui risultati vengono stoccati e poi restituiti alle
query che nli usano. Ad esempio:
\begin{lstlisting}[language=SQL]
WITH
  name1 AS (query1)
  name2 AS (query2)
  ...
  nameN AS (queryN)
query finale;
\end{lstlisting}
Diciamo per esepmio, di volere indicare il numero di pazienti di Siena mai visitati da ortopedici. Avrò
allora la CTE che trova tutti gli ortopedici:
\begin{lstlisting}[language=SQL]
WITH ortopedici AS
  (
  SELECT M.Matricola AS Medico
  FROM Medico M
  WHERE M.Specializzazione = 'Ortopedia'
  )
\end{lstlisting}
poi la CTE che trova tutti i pazienti visitati da tali ortpedici:
\begin{lstlisting}[language=SQL]
paz_visitati_ortopedici AS
  (
  SELECT V.Paziente AS CodFiscale
  FROM Visita V NATURAL JOIN Ortopedici O
  )
\end{lstlisting}
infine ottengo, usando le CTE precedenti, tutti i pazienti senesi:
\begin{lstlisting}[language=SQL]
SELECT COUNT(*)
FROM Paziente P
  NATURAL LEFT OUTER JOIN
  paz_visitati_ortopedici PVO
WHERE Citta = 'Siena'
  AND PVO.Matricola IS NULL
\end{lstlisting}
\end{document}

\documentclass[a4paper,12pt]{article}

\usepackage[french,italian]{babel}
\usepackage[T1]{fontenc}
\usepackage[utf8]{inputenc}
\frenchspacing 
\title{Appunti Basi di Dati}
\author{Luca Seggiani}
\date{27 Marzo 2024}

\usepackage{listings}
\usepackage{xcolor}

\definecolor{codegreen}{rgb}{0,0.6,0}
\definecolor{codegray}{rgb}{0.5,0.5,0.5}
\definecolor{codepurple}{rgb}{0.58,0,0.82}
\definecolor{backcolour}{rgb}{0.95,0.95,0.92}

\lstdefinestyle{code-style}{
    backgroundcolor=\color{backcolour},   
    commentstyle=\color{codegreen},
    keywordstyle=\color{magenta},
    numberstyle=\tiny\color{codegray},
    stringstyle=\color{codepurple},
    basicstyle=\ttfamily\footnotesize,
    breakatwhitespace=false,         
    breaklines=true,                 
    captionpos=b,                    
    keepspaces=true,                 
    numbers=left,                    
    numbersep=5pt,                  
    showspaces=false,                
    showstringspaces=false,
    showtabs=false,                  
    tabsize=2
}

\lstset{style=code-style}

\begin{document}
\maketitle
\section{Subquery}
Le subquery (query annidate) permettono di incapsulare query in altre query, in modo correlato o non correlato.
Presentano un modo alternativo di risolvere i problemi su cui normalmente si userebbe il join: infatti per ogni
problema che si può risolvere usando il join esiste sempre un corrispettivo che usa le subquery. Chiamiamo
outer query la query più esterna, cioè quella che incapsula, e inner query la subquery incapsulata.
\par\smallskip
\textbf{Subquery non correlate} \\
Le subquery non correlate (\textit{noncorrelated}) sono incapsulate nel WHERE, e permettono di ottenere un certo
result set intermedio, che verrà poi usato dall'outer query per calcolare il result set finale. Il result set
della subquery non correlata non dipende dall'outer query. Poniamo ad esempio di voler indicare nome, cognome 
e parcella degli ortopedici che hanno effettuato almeno una visita nell'anno 2013. Definiremo allora la subquery:
\begin{lstlisting}[language=SQL]
SELECT V.Medico
FROM Visita V
WHERE YEAR(V.Data) = 2013
\end{lstlisting}
che innesteremo poi in:
\begin{lstlisting}[language=SQL]
SELECT M.Nome, M.Cognome, M.Parcella
FROM Medico M
WHERE M.Specializzazione = 'Ortopedia'
  AND M.Matricola IN  (
                      SELECT V.Medico
                      FROM VIsita V
                      WHERE YEAR(V.Data) = 2013
                      );
\end{lstlisting}
La parola chiave IN permette di controllare la presenza dell'attributo M.Matricola nel result set della subquery.
La subquery calcolerà quindi un result set composto da tutte le visite fatte nel 2013. L'outer query userà la subquery,
confrontandola con tutti i medici specializzati in ortopedia la cui matricola trova un riscontro nel result set
calcolato dalla subquery. A questo punto si restituiscono gli attributi richiesti. Il meccanismo è del tutto analogo
a quello di un join naturale fra le tabelle "Medico" e "Visita". Notiamo che non è necessario alcuna istruzione DISTINCT, 
in quanto ogni record della tabella Medico viene considerato comunque una volta sola, e cercato nel result set delle visite. \\
Avremmo potuto porre la stessa query usando un comune join, ad esempio come:
\begin{lstlisting}[language=SQL]
SELECT M.Nome, M.Cognome, M.Parcella
FROM Medico M 
  INNER JOIN Visita V ON V.Medico = M.Matricola
WHERE YEAR(V.Data) = 2013 
  AND Medico.Specializzazione = 'Ortopedico';
\end{lstlisting}
\par\medskip
Si può anche usare la negazione di quanto appena fatto, usando la parola chiave NOT IN. Ad esempio, se vogliamo indicare i cognomi
dei pazienti che non appartengono anche a un medico, abbiamo la subquery:
\begin{lstlisting}[language=SQL]
SELECT M.Cognome
FROM Medico M
\end{lstlisting}
innestata in:
\begin{lstlisting}[language=SQL]
SELECT DISTINCT P.Cognome
FROM Paziente P
WHERE P.Cognome NOT IN  (
                        SELECT M.Cognome
                        FROM Medico M
                        );
\end{lstlisting}
\par\smallskip
Analogamente a prima, esisterà una versione che usa il join:
\begin{lstlisting}[language=SQL]
SELECT DISTINCT P.Cognome
FROM Paziente P
  LEFT OUTER JOIN Medico M ON
    M.Cognome = P.Cognome
WHERE M.Cognome IS NULL;
\end{lstlisting}
\textbf{Annidamento multiplo} \\
Non c'è teoricamente limite al numero di subquery che si possono innestare l'una dentro l'altra. A volte può infatti
tornare utile avere innesti multipli (e frontali!). Vediamo come ottenere il numero di tutti i pazienti di Siena mai visitati
da pazienti. Innanzitutto si trovano i medici specializzati in ortopedia:
\begin{lstlisting}[language=SQL]
SELECT M.Matricola
FROM Medico M
WHERE M.Specializzazione = 'Ortopedia'
\end{lstlisting}
\newpage
poi i pazienti visitati da suddetti medici:
\begin{lstlisting}[language=SQL]
SELECT V.Paziente
FROM Visita V
WHERE V.Medico IN (...)
\end{lstlisting}
e si rimuovono da una tabella dei soli pazienti senesi:
\begin{lstlisting}[language=SQL]
SELECT COUNT(*)
FROM Paziente P
WHERE P.Citta = 'Siena'
  AND P.CodFiscale NOT IN (
    SELECT V.Paziente
    FROM Visita V
    WHERE V.Medico IN (
      SELECT M.Matricola
      FROM Medico M
      WHERE M.Specializzazione = 'Ortopedia'
    )
  );
\end{lstlisting}
\par\smallskip
\textbf{Subquery scalari} \\
Una subquery scalare produce, invece che un insieme di record che poi verrà controllato con IN e NOT IN,
un singolo record di un singolo attributo, come ad esempio farebbe la funzione COUNT(). A questo punto, il valore
scalare prodotto può essere controllato con i vari operatori di confronto. Vogliamo ad esempio indicare il numero
degli otorini aventi parcella più alta della media delle parcelle della loro specializzazione. Troviamo innanzitutto
il la parcella media degli otorini (un valore scalare):
\begin{lstlisting}[language=SQL]
SELECT AVG(Medico.Parcella)
FROM Medico M
WHERE M.Specializzazione = 'Otorinolaringoiatria'
\end{lstlisting}
e la useremo in un'outer query:
\begin{lstlisting}[language=SQL]
SELECT COUNT(*)
FROM Medico M1
WHERE M1.Parcella >  (
                      SELECT AVG(Medico.Parcella)
                      FROM Medico M
          WHERE M.Specializzazione = 'Otorinolaringoiatria'
                      );

\end{lstlisting}
\par\medskip
Un caso interessante: mettiamo di voler trovare le entrate complessive generate dai cardiologi della clinica
negli ultimi 2 anni. Ciò risulta semplice con un join:
\begin{lstlisting}[language=SQL]
SELECT SUM(M.Parcella) AS IncassoTotale
FROM Visita V
INNER JOIN Medico M ON V.Medico = M.Matricola
WHERE V.Data > CURRENT_DATE - INTERVAL 12 YEAR
  AND M.Specializzazione = 'Cardiologia'
\end{lstlisting}
La versione con subquery è invece più complessa, richiedendo una subquery nel SELECT:
\begin{lstlisting}[language=SQL]
SELECT SUM((
  SELECT M.Parcella
  FROM Medico M
  WHERE M.Matricola = V.Medico
     AND M.Specializzazione = 'Cardiologia'
)) AS IncassoTotale
FROM Visita V
WHERE V.Data > CURRENT_DATE - INTERVAL 12 YEAR
\end{lstlisting}
Notiamo che la subquery è correlata, come vedremo nella prossima lezione.
\end{document}


\documentclass[a4paper,12pt]{article}

\usepackage[french,italian]{babel}
\usepackage[T1]{fontenc}
\usepackage[utf8]{inputenc}
\frenchspacing 
\title{Appunti Basi di Dati}
\author{Luca Seggiani}
\date{4 Aprile 2024}

\usepackage{listings}
\usepackage{xcolor}

\definecolor{codegreen}{rgb}{0,0.6,0}
\definecolor{codegray}{rgb}{0.5,0.5,0.5}
\definecolor{codepurple}{rgb}{0.58,0,0.82}
\definecolor{backcolour}{rgb}{0.95,0.95,0.92}

\lstdefinestyle{code-style}{
    backgroundcolor=\color{backcolour},   
    commentstyle=\color{codegreen},
    keywordstyle=\color{magenta},
    numberstyle=\tiny\color{codegray},
    stringstyle=\color{codepurple},
    basicstyle=\ttfamily\footnotesize,
    breakatwhitespace=false,         
    breaklines=true,                 
    captionpos=b,                    
    keepspaces=true,                 
    numbers=left,                    
    numbersep=5pt,                  
    showspaces=false,                
    showstringspaces=false,
    showtabs=false,                  
    tabsize=2
}

\lstset{style=code-style}

\begin{document}
\maketitle
\section{Raggruppamento}
Il raggruppamento (o aggregazione) divide in gruppi i record risultanti dalle clausole FROM e WHERE (detti record target),
sulla base di un certo attributo particolare in comune, che resta quindi costante in un determinato gruppo.
Poniamo di voler calcolare, nello schema clinica, per ogni specializzazione, la parcella media dei medici:
\begin{lstlisting}[language=SQL]
SELECT Specializzazione, AVG(Parcella) AS ParcellaMedia
FROM Medico
GROUP BY Specializzazione;
\end{lstlisting}
Dove il select di Specializzazione e AVG(Parcella) ha senso solo a seguito del GROUP BY, con Specializzazione
proprio come attributo di raggruppamento (!).
\par\smallskip
\textbf{Condizioni sui gruppi} \\
Le condizioni sui gruppi sono controllate gruppo per gruppo (non record per record) e permettono di scartarne alcuni. 
Si applicano attraverso l'operatore HAVING.
Poniamo di voler indicare le specializzazioni della clinica con più di due medici:
\begin{lstlisting}[language=SQL]
SELECT Specializzazione
FROM Medico 
GROUP BY Specializzazione
HAVING COUNT(*) > 2
\end{lstlisting}
Si nota che il DISTINCT non è necessario: ogni specializzazione viene presa comunque una volta sola.
Vediamo adesso un caso un attimo più complesso: ottenere le specializzazioni con la più alta parcella media. Dovremo
allora calcolare tutte le medie delle parcelle dei medici di ciascuna specializzazione, calcolare tra queste
la più alta, e selezionare quindi le specializzazioni la cui media corrisponde a questa media massima. \newpage
Il massimo si troverà con:
\begin{lstlisting}[language=SQL]
SELECT MAX(D.MediaParcelle)
FROM (
  SELECT M2.Specializzazione, AVG(M2.Parcella) AS MediaParcella
  FROM Medico M2
  GROUP BY M2.Specializzazione
)
AS D;
\end{lstlisting}
e la query completa sarà:
\begin{lstlisting}[language=SQL]
SELECT M.Specializzazione
FROM Medico M
GROUP BY Specializzazione
HAVING AVG(Parcella) = 
(
  SELECT MAX(D.MediaParcelle)
  FROM (
    SELECT M2.Specializzazione, AVG(M2.Parcella) AS MediaParcella
    FROM Medico M2
    GROUP BY M2.Specializzazione
  )
  AS D;
)
\end{lstlisting}
\section{Subquery correlate}
Abbiamo già visto le subquery non correlate (\textit{non-correlated}): il result set di tali subquery viene calcolato
una sola volta ed è indipendente dalle specifiche della query esterna (\textit{outer query}). Nelle subquery correlate
invece il result set dipende da ciascuna tupla della tupla esterna. L'ordine di esecuzione è: prima si esegue
il WHERE della query esterna, e poi la subquery. Poniamo ad esempio di voler indicare matricola
e parcella dei medici che hanno visitato per la prima volta almeno un paziente nel mese di ottobre 2013. Avremo
bisogno di selezionare pazienti che non erano mai stati visitati da un determinato medico prima di ottobre:
\begin{lstlisting}[language=SQL]
SELECT DISTINCT V1.Medico
FROM Visita V1
WHERE YEAR(V1.Data) = 2013
  AND MONTH (V1.Data) = 10
  AND V1.Paziente NOT IN ( 
                            SELECT V2.Paziente
                            FROM Visita V2
                            WHERE V2.Medico = V1.Medico
                              AND V2.Data < V1.Data
                          )
\end{lstlisting}
Notiamo la particolarità della subquery correlata: la relazione V1, appartenente alla query esterna, viene usata
nella clausola WHERE della correlata. Ricordiamo che una subquery correlata viene eseguita nuovamente per ogni tupla,
ed è quindi particolarmente inefficiente dal punto di vista della complessità. Notiamo inoltre che una subquery può
essere usata anche nella clausola SELECT della query, per calcolare velocemnte un valore da inserire nel risultato. Ad esempio,
vogliamo trovare tutti i pazienti di sesso maschile, indicandone nome e numero di visite effettuate. Quest'ultima informazione
non sarà un dato da proiettare (nel WHERE), ma da calcolare nel SELECT.
\begin{lstlisting}[language=SQL]
SELECT Nome, (
  SELECT COUNT(*)
  FROM Visite V
  WHERE V.Paziente = P.CodFiscale
) AS NumeroVisite
FROM Paziente P
WHERE P.Sesso = 'M'
\end{lstlisting}
\end{document}


\documentclass[a4paper,12pt]{article}

\usepackage[french,italian]{babel}
\usepackage[T1]{fontenc}
\usepackage[utf8]{inputenc}
\frenchspacing 
\title{Appunti Basi di Dati}
\author{Luca Seggiani}
\date{11 Aprile 2024}

\usepackage{listings}
\usepackage{xcolor}

\definecolor{codegreen}{rgb}{0,0.6,0}
\definecolor{codegray}{rgb}{0.5,0.5,0.5}
\definecolor{codepurple}{rgb}{0.58,0,0.82}
\definecolor{backcolour}{rgb}{0.95,0.95,0.92}

\lstdefinestyle{code-style}{
    backgroundcolor=\color{backcolour},   
    commentstyle=\color{codegreen},
    keywordstyle=\color{magenta},
    numberstyle=\tiny\color{codegray},
    stringstyle=\color{codepurple},
    basicstyle=\ttfamily\footnotesize,
    breakatwhitespace=false,         
    breaklines=true,                 
    captionpos=b,                    
    keepspaces=true,                 
    numbers=left,                    
    numbersep=5pt,                  
    showspaces=false,                
    showstringspaces=false,
    showtabs=false,                  
    tabsize=2
}

\lstset{style=code-style}

\begin{document}
\maketitle
\par\smallskip
\textbf{Correlated subquery nel SELECT} \\
L'inserimento di subquery correlate all'interno del SELECT permette di calcolare valori da inserire in un attributo
dell'insieme risultato. Proprio per questo motivo le subquery di questo tipo dovranno obbligatoriamente essere scalari.
Poniamo ad esempio di voler considerare il nome e il numero di visite effettuate da tutti i pazienti di sesso maschile.
Attraverso il raggruppamento, si potrà semplicemente dire:
\begin{lstlisting}[language=SQL]
SELECT P.Nome, COUNT(*) AS NumeroVisite
FROM Visita V INNER JOIN Paziente P ON V.Paziente = P.CodFiscale
WHERE P.Sesso = "M"
GROUP BY P.CodFiscale
\end{lstlisting}
Lo stesso comportamento può essere ottenuto attraverso le subquery correlate, ad esempio con:
\begin{lstlisting}[language=SQL]
SELECT P.Nome, (
  SELECT COUNT(*)
  FROM Visite V
  WHERE V.Paziente = P.CodFiscale
) AS NumeroVisite
FROM Paziente P
WHERE P.Sesso = "M"
\end{lstlisting}
Si può assumere che il modulo del DBMS detto \textit{query optimizer}, che ha il compito di ottimizzare le query,
tradurrà una query simile nell'equivalente che utilizza il raggruppamento.
\end{document}

\documentclass[a4paper,12pt]{article}

\usepackage[french,italian]{babel}
\usepackage[T1]{fontenc}
\usepackage[utf8]{inputenc}
\frenchspacing 
\title{Appunti Basi di Dati}
\author{Luca Seggiani}
\date{18 Aprile 2024}

\usepackage{listings}
\usepackage{xcolor}

\definecolor{codegreen}{rgb}{0,0.6,0}
\definecolor{codegray}{rgb}{0.5,0.5,0.5}
\definecolor{codepurple}{rgb}{0.58,0,0.82}
\definecolor{backcolour}{rgb}{0.95,0.95,0.92}

\lstdefinestyle{code-style}{
    backgroundcolor=\color{backcolour},   
    commentstyle=\color{codegreen},
    keywordstyle=\color{magenta},
    numberstyle=\tiny\color{codegray},
    stringstyle=\color{codepurple},
    basicstyle=\ttfamily\footnotesize,
    breakatwhitespace=false,         
    breaklines=true,                 
    captionpos=b,                    
    keepspaces=true,                 
    numbers=left,                    
    numbersep=5pt,                  
    showspaces=false,                
    showstringspaces=false,
    showtabs=false,                  
    tabsize=2
}

\lstset{style=code-style}

\begin{document}
\maketitle
\section{Costrutto EXISTS}
Il costrutto EXISTS ci permette di verificare che il result set di una subquery correlata contenga almeno un
record. La sua negazione controlla che il result set sia vuoto. Poniamo ad esempio di voler trovare medico,
paziente e data delle visite di controllo del mese di gennaio 2016. Definiamo una visita di controllo come una
visita in cui un medico visita un paziente già visitato precedentemente almeno una volta.
\begin{lstlisting}[language=SQL]
SELECT V_1.Medico, V_1.Paziente, V_1.Data
FROM Visita V_1
WHERE MONTH(V_1.Data) == 1 AND YEAR(V_1.Data) = 2016
  AND EXISTS
    (
    SELECT *
    FROM Visita V_2
    WHERE V_2.Paziente = V_1.Paziente
      AND V_2.Medico = V_1.Medico
      AND V_2.Data < V_1.Data
    )
\end{lstlisting}
Diciamo invece di voler indicare le matricole dei medici che hanno visitato per la primma volta almeno un paziente
nel mese di ottobre 2013.
\begin{lstlisting}[language=SQL]
SELECT DISTINCT V_1.Medico
FROM Visita V_1
WHERE MONTH(V_1.Data) = 10 AND YEAR(V_1.Data) = 2013
  AND NOT EXISTS
  (
  SELECT *
  FROM Visita V_2
  WHERE V_2.Paziente = V_1.Paziente
    AND V_2.Medico AND V_1.Medico
    AND V_1.Data < V_2.Data
  )
\end{lstlisting}
\section{Divisione insiemistica}
L'operatore di divisionie insiemistica permette di trovare tutti i record che rispettano condizioni esaustive. Ad esempio,
possiamo usarlo per indicare i pazienti visitati da tutti i medici. L'SQL non fornisce un'operatore di divisione insiemistica
di per sé, ma possiamo crearne uno attraverso altri costrutti. Esistono due metodi:
\begin{itemize}
  \item
  Possiamo usare il costrutto EXISTS e le subquery correlate per ottenere la seguente query:
    \begin{lstlisting}[language=SQL]
    SELECT P.CodFiscale
    FROM Paziente P
    WHERE NOT EXISTS 
    (
      SELECT *
      FROM Medico M
      WHERE NOT EXISTS
      (
        SELECT *
        FROM Visita V
        WHERE V.Medico = M.Matricola
        AND V.Paziente = P.CodFiscale
      )
    )
    \end{lstlisting}
    La query si traduce pressapoco con "un paziente è stato visitato da tutti i medici se non esiste un medico per cui
    non esiste una visita di quel medico con quel paziente".
  \item Possiamo usare il raggruppamento e una subquery non correlata:
    \begin{lstlisting}[language=SQL]
    SELECT V.Paziente
    FROM Visita V
    GROUP BY V.Paziente
    HAVING COUNT(DISTINCT V.Medico) = (SELECT COUNT(*)
                                        FROM Medico)
    \end{lstlisting}
    Che è considerevolmente più semplice.
\end{itemize}
\end{document}

\documentclass[a4paper,12pt]{article}

\usepackage[french,italian]{babel}
\usepackage[T1]{fontenc}
\usepackage[utf8]{inputenc}
\frenchspacing 
\title{Appunti Basi di Dati}
\author{Luca Seggiani}
\date{2 Maggio 2024}

\usepackage{listings}
\usepackage{xcolor}

\definecolor{codegreen}{rgb}{0,0.6,0}
\definecolor{codegray}{rgb}{0.5,0.5,0.5}
\definecolor{codepurple}{rgb}{0.58,0,0.82}
\definecolor{backcolour}{rgb}{0.95,0.95,0.92}

\lstdefinestyle{code-style}{
    backgroundcolor=\color{backcolour},   
    commentstyle=\color{codegreen},
    keywordstyle=\color{magenta},
    numberstyle=\tiny\color{codegray},
    stringstyle=\color{codepurple},
    basicstyle=\ttfamily\footnotesize,
    breakatwhitespace=false,         
    breaklines=true,                 
    captionpos=b,                    
    keepspaces=true,                 
    numbers=left,                    
    numbersep=5pt,                  
    showspaces=false,                
    showstringspaces=false,
    showtabs=false,                  
    tabsize=2
}

\lstset{style=code-style}

\begin{document}
\maketitle
\section{SQL Procedurale}
Introduciamo adesso il PL/SQL, un linguaggio di programmazione procedurale che estende l'SQL.
\par\smallskip
\textbf{Stored procedure} \\
Le stored procedure (procedure "stoccate") sono dei programmi dichiarativo-procedurali che possono essere memorizzati nel DBMS
e invocati proprio come funzioni nei linguaggi di programmazione tradizionali, restituendo valori. Procedure possono essere dichiarate
e poi chiamate all'interno di query, o altre procedure.
\par\smallskip
\textbf{Interfacce con stored procedure} \\
Il meccanismo delle stored procedure può essere usato per implementare delle interfacce: all'utente non si mette a disposizione l'SQL, ma soltanto
le procedure definite su quel DBMS. In questo modo l'utente non ha mai effettivamente accesso al database stesso, ma può comunque interagirci con procedure già definite.
Questo sistema assicura il mascheramento dei dati e del codice, aumentando la sicurezza e la protezione da eventuali attacchi.
Meccanismi di questo tipo sono alla base di architetture \textbf{multilivello} (\textit{multi-tier}) usati ad esempio nei siti internet:
l'uso di procedure predefinite per l'interazione fra server e database, e quindi fra utente e server, permette di aumentare la sicurezza del sistema.
L'invocazione di procedure richiede, da parte del chiamante, il possesso di un particolare \textbf{grant} ("permesso"), che garantisce l'accesso
del suddetto a tali procedure. E' ad esempio possibile che un'applicazione abbia accesso alle procedure ma non alle tabelle del database. Vediamo un'esempio.
Vogliamo scrivere una stored procedure che restituisca tuttte le specializzazioni mediche offerte dalla clinica. Dovremo usare la seguente sintassi:

\begin{lstlisting}[language=SQL]
DROP PROCEDURE IF EXISTS mostra_specializzazioni
DELIMITER $$
CREATE PROCEDURE mostra_specializzazioni()
  BEGIN
    SELECT DISTINCT Specializzazione
    FROM Medico
  END $$
DELIMITER;
\end{lstlisting}
Il pasticcio di delimitatori serve a far ignorare all'SQL l'uso del ; all'interno della procedura, che altrimenti significherebbe
"compila quanto hai letto fino ad ora". Alla fine della dichiarazione reimpostiamo il delimitatore a quello di default. \\
Potremo adesso chiamare la procedure con:
\begin{lstlisting}[language=SQL]
CALL mostra_specializzazioni()
\end{lstlisting}
\par\smallskip
\textbf{Variabili locali} \\
All'interne delle stored procedure è possibile memorizzante informazioni intermedie di ausilio, attraverso variabili locali. La dichiarazione
di variabili locali va fatta contestualmente alla dichiarazione della procedura (cioè subito) con la sintassi:
\begin{lstlisting}[language=SQL]
DECLARE nome_variabile tipo(size) DEFAULT valore_default;
\end{lstlisting}
La tipizzazione è forte, e sono presenti i classici tipi (int, double, char, date...). Il valore size rappresenta la dimensione della variabile:
ad esempio, per il tipo char, potremo specificare char(50) per indicare una stringa di 50 caratteri. Il tipo varchar è un tipo particolare che può
variare la sua dimensione, occupando \textit{al massimo} il valore fornito alla dichiarazione. Si può fare assegnamento sulle variabili attraverso la parola
chiave SET:
\begin{lstlisting}[language=SQL]
DECLARE min_visite_mensili INT DEFAULT 0;

% il corpo della procedura...

SET min_visite_mensili = 20;
\end{lstlisting}
Facciamo un'esempio: supponiamo di essere nel body di una stored procedure e creare una variabile contenente il numero minimo di visite effettuate questo mese, avremo:
\begin{lstlisting}[language=SQL]
DECLARE visite_mese_attuale INT DEFAULT 0;

% il corpo della procedura...

SET visite_mese_attuale = (
  SELECT COUNT(*)
  FROM Visita V
  WHERE MONTH(V.Data) = MONTH(CURRENT_DATE)
    AND YEAR(V.Data) = YEAR(CURRENT_DATE)
)
\end{lstlisting}
oppure, analogamente:
\begin{lstlisting}[language=SQL]
DECLARE visite_mese_attuale INT DEFAULT 0;

% il corpo della procedura...

SELECT COUNT(*)
FROM Visita V
WHERE MONTH(V.Data) = MONTH(CURRENT_DATE)
  AND YEAR(V.Data) = YEAR(CURRENT_DATE)
INTO visite_mese_attuale
\end{lstlisting}
Notiamo l'errore (di impedenza) in cui potremmo incorrere facendo qualcosa del tipo:
\begin{lstlisting}[language=SQL]
DECLARE visite_mese_attuale VARCHAR(255)
SELECT * INTO visite_mese_attuale
FROM Visita V
WHERE MONTH(V.Data) = MONTH(CURRENT_DATE)
  AND YEAR(V.Data) = YEAR(CURRENT_DATE)
\end{lstlisting}
Non possiamo infatti immagazzinare (anzi, "collassare") un'intero record in una sola variabile di tipo VARCHAR.
\par\smallskip
\textbf{Variabili user-defined} \\
Le variabili user-defined sono variabili definite dall'utente senza necessità di dichiarazione, che hanno tempo di vita 
uguale alla durata della connessione dell'utente al server. A differenza delle variabili locali, le variabili user-defined hanno
tipizzazione debole e dinamica: potremo memorizzarvici qualsiasi tipo di valore, e anche valori di tipi diversi in momenti diversi.
Non possono contenere result set, ma solo valori scalari. Si dichiarano con @.
\par\smallskip
\textbf{Parametri di una stored procedure} \\
Una stored procedure MySQL accetta parametri di tipo:
\begin{itemize}
  \item \textbf{Ingresso} \\
    Un parametro di ingresso può essere letto, ma non modificato. Equivale al passaggio per valore del C++. Si dichiara
    con la parola chiave IN. Ad esempio, poniamo di voler scrivere una stored procedure che stampi la parcella media di una specializzazione
    specificata come parametro:
    \begin{lstlisting}[language=SQL]
DROP PROCEDURE IF EXISTS parcella_media_spec;  
DELIMITER $$
CREATE PROCEDURE parcella_media_spec(IN _specializzazione VARCHAR(100))
% resto del codice...
DELIMITER;
\end{lstlisting}
    a questo punto potremo chiamare la procedura con:
    \begin{lstlisting}[language=SQL] 
CALL parcella_media_spec('Ortopedia');
\end{lstlisting}
  \item \textbf{Uscita} \\
Un parametro di uscita può essere modificato dalla procedura, e viene fornito in fase di chiamata dal chiamante. Nella chiamata si possono usare,
nei parametri di uscita, variabili user defined (@variabile). \\
Facciamo un'esempio: vogliamo scrivere una stored porcedure che restituisca \textit{come parametro di uscita} il numero di pazienti visitati da medici di una data
specializzazione, ricevuta come parametro di ingresso:
\begin{lstlisting}[language=SQL]
DROP PROCEDURE IF EXISTS parcella_media_spec;  
DELIMITER $$
CREATE PROCEDURE parcella_media_spec(IN _specializzazione VARCHAR(100), OUT totale_pazienti INT)
% resto del codice...
DELIMITER;

\end{lstlisting}
come prima, potremo adesso chiamare la procedura con la sintassi:
\begin{lstlisting}[language=SQL]
CALL tot_pazienti_visitati_spec('Neurologia', @quantiPazienti);
SELECT @quantiPazienti
\end{lstlisting}
dove @quantiPazienti è una variabile user-defined. \\
Si noti che si possono avere più parametri di uscita in una procedura SQL.
  \item \textbf{Ingresso-uscita} \\
    Non vengono trattate da questo corso. In altre parole t'attacchi.
\end{itemize}
\par\smallskip
\textbf{Istruzioni condizionali} \\
Le espressioni condizionali permettono di esprimere condizioni, modificando il flusso di esecuzioni. Possono
contenere letterali, variabili e funzioni. Nell'SQL si riducono alle due parole chiave: IF e CASE.
\begin{itemize}
  \item \textbf{Istruzione IF} \\
    L'IF è analogo a quello di altri linguaggi, con l'inclusione del THEN:
    \begin{lstlisting}[language=SQL]
IF if_condition THEN
  -- blocco di istruzioni 
ELSEIF elseif_1_condition THEN
 ...
ELSEIF elseif_N_condition THEN
ELSE
  -- blocco di else
END IF;
    \end{lstlisting}
  \item \textbf{Istruzione CASE} \\
    Vediamo il CASE:
    \begin{lstlisting}[language=SQL]
CASE 
WHEN condition_1 THEN
  -- blocco 1
 ...
WHEN condition_n THEN
  --blocco n
END CASE
\end{lstlisting}
\end{itemize}
\par\smallskip
\textbf{Istruzioni iterative} \\
Le struzioni iterative permettono di ripetere blocchi di codice. L'SQL fornisce il WHILE, il REPAT, e il LOOP.
\begin{itemize}
  \item \textbf{Istruzione WHILE} \\
    Corrisponde al semplice while.
  \begin{lstlisting}[language=SQL]
WHILE condition DO
  -- blocco istruzioni
END WHIILE
\end{lstlisting}
\item \textbf{Istruzione REPEAT} \\
  \begin{lstlisting}[language=SQL]
REPEAT
  -- blocco istruzioni
UNTIL condition
END REPEAT
\end{lstlisting}
\item \textbf{Istruzione LOOP} \\
Definisce un LOOP, che andrà chiuso con un'istruzione LEAVE.
\begin{lstlisting}[language=SQL]
loop_label: LOOP
  -- blocco di istruzioni, check di condizioni
END LOOP
\end{lstlisting}
\end{itemize}
\par\smallskip
\textbf{Istruzioni di salto} \\
Le istruzioni di salto permettono di interrompere cicli o passare a iterazioni successive. Nell'SQL sono rispettivamente
la LEAVE e ITERATE.
\par\medskip
\textbf{Cursore} \\
Un cursore scorre i record su un result set, solo in avanti, per effettuare delle azioni all'interno di istruzioni iterative.
Vediamo la sintessi:
\begin{lstlisting}[language=SQL]
DECLARE NomeCursore CURSOR FOR
SQL query;
\end{lstlisting}
I cursori si possono dichiarare immediatamente dopo la dichiarazione di tutte le variabili, contestualmente alla dichiarazione
della procedura. Su un cursore valgono le seguenti operazioni:
\begin{lstlisting}[language=SQL]
OPEN NomeCursore;
FETCH NomeCursore INTO ListaVariabili;
CLOSE NomeCursore;
\end{lstlisting}
La prima e l'ultima operazione sono banali. Il FETCH si limita a restituire il prossimo record e avanzare il cursore.
\par\smallskip
\textbf{Handler} \\
Gli handler sono gestori di situazioni, utili per eseguire codice quando l'istruzione fetch porta il cursore in fondo alla tabella.
Possono essere definiti dopo le dichiarazioni di variabili e cursori, sempre contestualmente alla dichiarazione della procedura.
Ad esempio, esiste il NOT FOUND HANDLER:
\begin{lstlisting}[language=SQL]
DECLARE CONTINUE HANDLER FOR NOT FOUND
SET finito = 1;
\end{lstlisting}
Questo handler viene eseguito quando si arriva a "fine corsa", ovvero in fondo alla tabella che si stava scorrendo.
La parola chiave CONTINUE rappresenta il fatto che l'handler non interrompe l'esecuzione, a differenza di un handler EXIT.
\par\smallskip
\textbf{Ciclo di fetch} \\
Il meccanismo del FETCH e degli HANDLER ci permette di stabilire un ciclo di fetch:
\begin{lstlisting}[language=SQL]
DECLARE cur CURSOR FOR tabella
DECLARE CONTINUE HANDLER FOR NOT FOUND
SET finito = 1;

scan: LOOP
FETCH cur -- prelieva il record
IF finito = 0
  -- processo il record
ELSE LEAVE scan
END LOOP scan;
\end{lstlisting}
\end{document}

\documentclass[a4paper,12pt]{article}

\usepackage[french,italian]{babel}
\usepackage[T1]{fontenc}
\usepackage[utf8]{inputenc}
\frenchspacing 
\title{Appunti Basi di Dati}
\author{Luca Seggiani}
\date{9 Maggio 2024}

\usepackage{listings}
\usepackage{xcolor}

\definecolor{codegreen}{rgb}{0,0.6,0}
\definecolor{codegray}{rgb}{0.5,0.5,0.5}
\definecolor{codepurple}{rgb}{0.58,0,0.82}
\definecolor{backcolour}{rgb}{0.95,0.95,0.92}

\lstdefinestyle{code-style}{
    backgroundcolor=\color{backcolour},   
    commentstyle=\color{codegreen},
    keywordstyle=\color{magenta},
    numberstyle=\tiny\color{codegray},
    stringstyle=\color{codepurple},
    basicstyle=\ttfamily\footnotesize,
    breakatwhitespace=false,         
    breaklines=true,                 
    captionpos=b,                    
    keepspaces=true,                 
    numbers=left,                    
    numbersep=5pt,                  
    showspaces=false,                
    showstringspaces=false,
    showtabs=false,                  
    tabsize=2
}

\lstset{style=code-style}

\begin{document}
\maketitle
\section{Manipolazione dei dati}
Vediamo adesso i costrutti della parte \textbf{DML} (\textit{Data Manipulation Language}), dell'SQL.
\par\smallskip
\textbf{Inserimento} \\
L'inserimento consiste nel, considerata una nuova tabella, inserire un nuovo record i cui valori degli attributi
possono essere sia statici che ricavati. Inseriamo un nuovo paziente nel database, notiamo i dati (statici):
\begin{itemize}
  \item Nome: Elvira
  \item Cognome: Passerotti
  \item Sesso: F
  \item Data di nascita: 27 Ottobre 1964
  \item Città: Pisa
  \item Reddito: 1500 Euro
  \item Codice Fiscale: PSSLVR65R67G702U
\end{itemize}
\begin{lstlisting}[language=SQL]
INSERT INTO Paziente
VALUES ('PSSLVR65R67G702U', 'Passerotti', 'Elvira', 'F', '1965-10-27', 'Pisa', 1500);
\end{lstlisting}
Dove è importante rispettare l'ordine di definizione degli attributi della tabella. Facciamo un'altro esempio:
vogliamo inserire il paziente Edoardo Lepre, visitato tre giorni fa, di codice fiscale "slq6". Poniamo però
di non sapere se la visita era stata mutuata o meno:
\begin{lstlisting}[language=SQL]
INSERT INTO Visita(Medico, Paziente, Data)
VALUES(010, 'slq_6', CURRENT_DATE - INTERVAL 3 DAY);
\end{lstlisting}
Questo mi permette di specificare solo alcuni valori, lasciando gli altri a NULL o al loro valore di default.
Notiamo che, in generale, la sintassi:
\begin{lstlisting}[language=SQL]
INSERT INTO Tabella [(Attributo_1, Attributo_2, ...)]
VALUES (Valore_1, Valore_2, ...);
\end{lstlisting}
Ci permette di definire qualsiasi attributo in qualsiasi ordine, anche diverso da quello di definizione.
\par\smallskip
\textbf{Inserimento con valori ricavati} \\
E' possibile definire un'inserimento del tipo:
\begin{lstlisting}[language=SQL]
INSERT INTO Tabella [(Attributo_1, Attributo_2, ..., Attributo_N)]
Query_di_selezione;
\end{lstlisting}
Attraverso una query "Query\_di\_selezione" arbitraria che restituisca valori per tutti gli attributi. Poniamo
di voler archiviare tutte i record delle visite effettuate prima di due anni fa in una nuova tabella, VisitaVecchia(CognomePaziente, CognomeMedico, DataVisita, Specializzazione):
\begin{lstlisting}[language=SQL]
INSERT INTO VisitaVecchia
SELECT P.Cognome AS CognomePaziente,
  M.Cognome AS CognomeMedico,
  V.Data AS DataVisita,
  M.Specializzazione
FROM Visita V INNER JOIN Medico M ON M.Matricola = V.Medico
  INNER JOIN Paziente P ON P.CodFiscale = V.Paziente
WHERE YEAR(V.Data) <= YAER(CURRENT_DATE) - 2;
\end{lstlisting}
Come vediamo, basterà definire una query che restituisce tutti i record da inserire nella tabella nell'ordine di definizione (dato nel SELECT).
\par\smallskip
\textbf{Aggiornamento} \\
Possiamo anche aggiornare record già definiti. Dovremo però usare una clausola WHERE per specificare quali record aggiornare:
\begin{lstlisting}[language=SQL]
UPDATE Tabella
SET Attributo_i = Valore_i, ecc...
WHERE Condizione
\end{lstlisting}
Mutuiamo tutte le visite del mese corrente effettuate da pazienti nati prima del 1925:
\begin{lstlisting}[language=SQL]
UPDATE Visita
SET Mutuata = 1
WHERE MONTH(Data) = MONTH(CURRENT_DATE)
  AND YEAR(Data) = YEAR(CURRENT_DATE)
  AND Paziente IN (
    SELECT CodFiscale
    FROM Paziente
    WHERE YEAR(DataNascita) < 1925
  );
\end{lstlisting}
\par\smallskip
\textbf{Cancellazione} \\
Vediamo come cancellare record dalla tabella. La sintassi è analoga a quella dell'aggiornamento:
\begin{lstlisting}[language=SQL]
DELETE FROM Tabella
WHERE Condizione
\end{lstlisting}
Poniamo di voler licenziare tutti i medici di Pisa che non hanno effettuato visite mutate il mese scorso.
Scriviamo la query:
\begin{lstlisting}[language=SQL]
DELETE FROM Medico
WHERE Matricola IN (
  SELECT M_1.Matricola
  FROM Medico M_1 LEFT OUTER JOIN (
    SELECT V_2.Medico AS MedicoMutuato
    FROM Visita V_2 INNER JOIN Medico M_2 ON M_2.Matricola = V_2.Medico
    WHERE M_2.Citta = 'Pisa'
      AND V_2.Mutuata = TRUE
      AND V_2.Data > CURRENT_DATE - INTERVAL 1 MONTH
    ) AS D
    ON M_1.Matricola = D.MedicoMutuato
    WHERE D.MedicoMutuato IS NULL
);
\end{lstlisting}
Il DBMS riceve la query e restituisce un errore. Questo perchè, nel momento in cui dichiariamo l'operazione di cancellazione con DELETE FROM Medico;
stiamo effettivamente bloccando la tabella. Non possiamo quindi definire una query che legga tale tabella: il DBMS ce lo impedirà in quanto potremmo
generare loop infiniti. Possiamo risolvere questa situazione in più modi.
\begin{itemize}
  \item Trasformare la query in una derived table:
\begin{lstlisting}[language=SQL]
DELETE FROM Medico
WHERE Matricola IN (
  SELECT * FROM (
    SELECT M_1.Matricola
    FROM Medico M_1 LEFT OUTER JOIN (
     SELECT V_2.Medico AS MedicoMutuato
     FROM Visita V_2 INNER JOIN Medico M_2 ON M_2.Matricola = V_2.Medico
     WHERE M_2.Citta = 'Pisa'
       AND V_2.Mutuata = TRUE
       AND V_2.Data > CURRENT_DATE - INTERVAL 1 MONTH
     ) AS D
     ON M_1.Matricola = D.MedicoMutuato
     WHERE D.MedicoMutuato IS NULL
  ) AS D_2
);
\end{lstlisting}
Questo assicura che il valore della derived table venga calcolato prima dell'operazione di cancellazione, eludendo quindi il blocco imposto dal DBMS. La stessa
cosa potrà essere fatta, anziché con una derived table, usando una CTE.
\item Un'altro approccio, meno intuitivo, è quello di usare un join:
\begin{lstlisting}[language=SQL]
DELETE M_1.*
FROM Medico M_1 LEFT OUTER JOIN (
  SELECT V_2.Medico AS MedicoMutuato
  FROM Visita V_2 INNER JOIN Medico M_2 ON M_2.Matricola = V_2.Medico
  WHERE M_2.Citta = 'Pisa'
    AND V_2.Mutuata = TRUE
    AND V_2.Data > CURRENT_DATE - INTERVAL 1 MONTH
 ) AS D
ON M_1.Matricola = D.MedicoMutuato
WHERE D.MedicoMutuato IS NULL;
\end{lstlisting}
\end{itemize}
\end{document}

\documentclass[a4paper,12pt]{article}

\usepackage[french,italian]{babel}
\usepackage[T1]{fontenc}
\usepackage[utf8]{inputenc}
\frenchspacing 
\title{Appunti Basi di Dati}
\author{Luca Seggiani}
\date{15 Maggio 2024}

\usepackage{listings}
\usepackage{xcolor}

\definecolor{codegreen}{rgb}{0,0.6,0}
\definecolor{codegray}{rgb}{0.5,0.5,0.5}
\definecolor{codepurple}{rgb}{0.58,0,0.82}
\definecolor{backcolour}{rgb}{0.95,0.95,0.92}

\lstdefinestyle{code-style}{
    backgroundcolor=\color{backcolour},   
    commentstyle=\color{codegreen},
    keywordstyle=\color{magenta},
    numberstyle=\tiny\color{codegray},
    stringstyle=\color{codepurple},
    basicstyle=\ttfamily\footnotesize,
    breakatwhitespace=false,         
    breaklines=true,                 
    captionpos=b,                    
    keepspaces=true,                 
    numbers=left,                    
    numbersep=5pt,                  
    showspaces=false,                
    showstringspaces=false,
    showtabs=false,                  
    tabsize=2
}

\lstset{style=code-style}

\begin{document}
\maketitle
\section{Database attivi}
Abbiamo visto finora le metodistiche dall'SQL procedurale, che permette di implementare funzionalità in linguaggio, appunto, procedurale, sebbene il paradigma dell'SQL sia in teoria dichiarativo. Adesso vedremo
alcune caratteristiche dell'SQL che permettono di implementare comportamenti "attivi" del database: per la precisione, \textbf{trigger} e \textbf{event}.
\par\smallskip
\textbf{Trigger} \\
Un trigger è una procedura che viene eseguita sulla base di eventi di istruzione DML (\textit{data manipulation language}).
Un trigger è fornito di una \textbf{parte reattiva} che reagisce a eventi DML, o a determinati istanti temporali, che causa l'esecuzione
di un'azione (operazione) sui dati. Gli eventi DML scatenanti possono ad esempio essere modifiche, cancellazioni, inserzioni, timer ecc... \\
La \textbf{condizione} è un predicato booleano che viene valutato dopo l'evento. Un risultato positivo della condizione comporta
l'esecuzione dell'azione. Abbiamo quindi:
\begin{enumerate}
  \item Evento;
  \item Condizione;
  \item Azione.
\end{enumerate}
Facciamo un'esempio: voglliamo gestire un'attributo ridondante nella tabella Paziente contenente la data nella quale un paziente è stato visitato l'ultima volta.
Sarà un'attributo ridondante perché lo potremo già tovare nella tabella Visita. D'altronde, ad ogni successiva visita, l'attributo sulla tabella paziente
andrà aggiornato. In ogni caso, il comportamento che vogliamo è che, all'inserzione dell'ultima visita nella tabella visita da parte di un'operatore, il database
possa automaticamente (\textit{database attivo}) inserire l'informazione necessaria nella tabella paziente attraverso un trigger. \\
Individuiamo evento, condizione ed azione:
\begin{itemize}
  \item \textbf{Evento:} l'evento sarà l'inserimento di una nuova visita nella tabella Visita;
  \item \textbf{Condizione:} qui la condizione non c'é: ogni nuova visita è un'ultima visita.
  \item \textbf{Azione:} rintraccia il paziente della visita, e aggiorna la sua ultima visita.
\end{itemize}
in SQL, la sintassi di un trigger sarà:
\begin{lstlisting}[language=SQL]
DROP TRIGGER IF EXISTS nome_trigger
CREATE TRIGGER nome_trigger
[BEFORE | AFTER][INSERT |UPDATE|DELETE] ON target
FOR EACH ROW blocco_istruzioni
\end{lstlisting}
Vediamo i dettagli. Con BEFORE si indica un'azione di \textit{preprocessing} quindi di esecuzione prima dell'evento, mentre con AFTER
si indica un'azione \textit{a posteriori}, o "collaterale", che viene eseguita dopo l'evento. La sintassi che vorremo nell'esempio appena sopra
sarà quindi:
\begin{lstlisting}[language=SQL]
DROP TRIGGER IF EXISTS aggiorna_ultima_visita;
CREATE TRIGGER aggiorna_ultima_visita
AFTER INSERT ON Visita FOR EACH ROW
  UPDATE Paziente
  SET UltimaVisita = CURRENT_DATE
  WHERE CodFiscale = NEW.Paziente
\end{lstlisting}
dove NEW indica il record che è stato già inserito (in un trigger AFTER; in un trigger BEFORE sarebbe stato quello che sta per essere inserito)..
\par\smallskip
\textbf{Trigger multi-statement} \\
Un trigger \textit{multi-statement} ("multi-blocco") è formato da più blocchi separati da punti e virgola. Si rende quindi necessario,
come avevamo già visto, l'uso di un delimitatore ausiliario:
\begin{lstlisting}[language=SQL]
DROP TRIGGER IF EXISTS nome_trigger;
DELIMITER $$
CREATE TRIGGER nome_trigger
BEFORE ... ON ...
FOR EACH ROW

BEGIN
-- blocchi di istruzioni
END $$
DELIMITER ;
\end{lstlisting}

\end{document}

\documentclass[a4paper,12pt]{article}

\usepackage[french,italian]{babel}
\usepackage[T1]{fontenc}
\usepackage[utf8]{inputenc}
\frenchspacing 
\title{Appunti Basi di Dati}
\author{Luca Seggiani}
\date{16 Maggio 2024}

\usepackage{listings}
\usepackage{xcolor}

\definecolor{codegreen}{rgb}{0,0.6,0}
\definecolor{codegray}{rgb}{0.5,0.5,0.5}
\definecolor{codepurple}{rgb}{0.58,0,0.82}
\definecolor{backcolour}{rgb}{0.95,0.95,0.92}

\lstdefinestyle{code-style}{
    backgroundcolor=\color{backcolour},   
    commentstyle=\color{codegreen},
    keywordstyle=\color{magenta},
    numberstyle=\tiny\color{codegray},
    stringstyle=\color{codepurple},
    basicstyle=\ttfamily\footnotesize,
    breakatwhitespace=false,         
    breaklines=true,                 
    captionpos=b,                    
    keepspaces=true,                 
    numbers=left,                    
    numbersep=5pt,                  
    showspaces=false,                
    showstringspaces=false,
    showtabs=false,                  
    tabsize=2
}

\lstset{style=code-style}

\begin{document}
\maketitle
\par\smallskip
\textbf{Regole aziendali} \\
Le regole aziendali (\textit{business rule}) sono determinate regole della realtà d'interesse da modelizzare nel DBMS. Il meccanismo
dei trigger fornisce un modo per assicurarsi che tali regole vengano rispettate. Poniamo ad esempio la regola aziendale:
"Ogni mese, le visite mutuate di un medico non possono essere più di 10".
\begin{lstlisting}[language=SQL]
DROP TRIGGER IF EXISTS blocca_non_mutuate;
DELIMITER $$
CREATE TRIGGER blocca_non_mutuate
BEFORE INSERT ON Visite
FOR EACH ROW
BEGIN

  DECLARE non_mutuate_mese INTEGER DEFAULT 0;

  SET non_mutuate_mese = (
    SELECT COUNT(*)
    FROM Visita V
    WHERE M.Medico = NEW.Medico
      AND MONTH(V.Data) = MONTH(CURRENT_DATE)
      AND YEAR(V.Data) = YEAR(CURRENT_DATE)
      AND V.Mutuata = 1
  );
  IF non_mutuate_mese = 10 THEN
    SIGNAL SQLSTATE '45000'
    SET MESSAGE_TEXT = 'Visita non inseribile';
  END IF;

END $$
DELIMITER ;
\end{lstlisting}
Il comando "SIGNAL SQLSTATE '45000'" serve a lanciare un'errore (segnale), di cui impostiamo subito dopo il messaggio d'errore,
e impedire l'inserimento. Nello specifico, il codice 45000 sta per "errore generico". In ogni caso, l'\textit{exception handling} è al di fuori
degli argomenti del corso.
\par\smallskip
\textbf{Event} \\
Gli \textbf{event} sono procedure simili ai trigger, ma la loro causa scatenante è il raggiungimento di uno 
specifico istante temporale. Ad esempio possono essere usati per eseguire azioni periodice (giornalmente, mensilmente, ecc...).
Poniamo per esempio di voler creare e mantenere giornalmente aggiornata una ridondanza nella tabella Medico contenente il totale
delle visite effettuate. Potremo dire:
\begin{lstlisting}[language=SQL]
CREATE EVENT AggiornaTotaliVisite
ON SCHEDULE EVERY 1 DAY
STARTS '2023-05-25 23:55:00'
DO
  UPDATE Medico
  SET TotaleVisite = TotaleVisite + 
                      (SELECT COUNT(*)
                       FROM Visita V
                       WHERE V.Medico = Matricola
                        AND V.Data = CURRENT_DATE);
\end{lstlisting}
Notiamo la parola chiave STARTS per impostare la data di inizio. Esiste inoltre la variante STARTS - ENDS  per impostare
anche la data di fine della schedulazione.
\par\smallskip
\textbf{Attivazione dello scheduler} \\
L'attivazione dello scheduler si ottiene impostando la varibile di sistema event\_scheduler, come:
\begin{lstlisting}[language=SQL]
SET GLOBAL event_scheduler = ON;
\end{lstlisting}
\section{Materialized view}
Le materialized view sono viste ridondanti, cioè ricavabili dai dati nel database (detti \textit{raw data}), che scegliamo di precalcolare.
Sostanzialmente, una materialized view è il risultato precalcolato di una query (spesso formata da un join molto corposo).
A differenza delle normali view (viste), la materialized view non viene calcolata ad ogni accesso, ma precalcolata periodicamente,
accumulando inevitabilmente un certo scarto temporale. Si usano quando il risultato deve essere ottenuto velocemente,
ma la query necessaria a calcolarla richiede molte risorse (e tempo). \\
La dichiarazione di una materialized view corrisponde a quella di una tabella:
\begin{lstlisting}[language=SQL]
CREATE TABLE MATERIALIZED_VIEW(
  Paziente CHAR(100) NOT NULL,
  NumVisite INT(11) NOT NULL DEFAULT 0,
  UltimaVisita DATE,
  PRIMARY KEY(Paziente)
) ENGINE = InnoDB DEFAULT CHARSET=latin1;
\end{lstlisting}

\par\smallskip
\textbf{Politiche di refresh} \\
Esistono più politiche di refresh (ricalcolo) delle materialized view:
\begin{itemize}
  \item \textbf{Immediate:} dopo ogni evento, quindi equiparabili ad un \textbf{trigger};
  \item \textbf{Deferred:} su base temporale, quindi equiparabili ad un \textbf{event};
  \item \textbf{On demand:} avviate manualmente, quindi equiparabili ad una \textbf{stored procedure}.
\end{itemize}
Esistono inoltre due modalità di aggiornamento della materialized view:
\begin{itemize}
  \item \textbf{Full refresh:} si aggiorna tutta la vista in blocco, da zero.
  \item \textbf{Incremental refresh:} si aggiornano solamente le componenti non più aggiornate. Può quindi essere
    un'aggiornamento sia totale che parziale.
\end{itemize}
Vediamo alcuni esempi:
\begin{itemize}
  \item \textbf{Immediate refresh} (sync): \\
    Una vista materializzata ad aggiornamento immediato dopo ogni aggiornamento si può implementare come:
    \begin{lstlisting}[language=SQL]
DELIMITER $$
DROP TRIGGER IF EXISTS immediate_refresh_visita $$
CREATE TRIGGER immediate_refresh_visita
AFTER INSERT ON Visita
FOR EACH ROW
BEGIN
  UPDATE MATERIALIZED_VIEW
  SET NumVisite = NumVisite + 1
    UltimaVisita = CURRENT_DATE
  WHERE Paziente = NEW.Paziente
END $$
\end{lstlisting}
questo assumendo che ad ogni paziente corrisponderà un record in MATERIALIZED\_VIEW, creato con il trigger:
\begin{lstlisting}[language=SQL]
DROP TRIGGER IF EXISTS immediate_refresh_paziente $$
CREATE TRIGGER immediate_refresh_paziente
AFTER INSERT ON Paziente
FOR EACH ROW
BEGIN
  INSERT INTO MATERIALIZED_VIEW(Paziente)
  VALUES(NEW.CodFiscale)
END $$
DELMITER ;
\end{lstlisting}
\item \textbf{On demand refresh} (full): \\
  Vediamo una vista materializzata on demand (ad aggiornamento manuale, con procedura) di tipo full:
\begin{lstlisting}[language=SQL]
DELIMITER $$
DROP PROCEDURE IF EXISTS on_demand_refresh $$
CREATE PROCEDURE on_demand_refresh
BEGIN 
  TRUNCATE MATERIALIZED_VIEW;
  INSERT INTO MATERIALIZED_VIEW
  SELECT P.CodFiscale,
    IF(V.Paziente IS NOT NULL, COUNT(*), 0),
    IF(V.Paziente IS NOT NULL, MAX(V.Data), NULL)
  FROM Visita V
    RIGHT OUTER JOIN
    Paziente P ON P.CodFiscale = V.Paziente
  GROUP BY P.CodFiscale
END $$
DELMITER ;
\end{lstlisting}
\item \textbf{Deferred refresh} (full): \\
  Vediamo infine come implementare il refresh in differita, appoggiandoci alla procedura on demand appena creata:
\begin{lstlisting}[language=SQL]
DELIMITER $$
DROP EVENT IF EXISTS deferred_refresh $$
CREATE EVENT deferred_refresh
ON SCHEDULE EVERY 1 WEEK
BEGIN
  CALL on_demand_refresh;
END $$
DELIMITER ;
\end{lstlisting}
\end{itemize}
\end{document}

\documentclass[a4paper,12pt]{article}

\usepackage[french,italian]{babel}
\usepackage[T1]{fontenc}
\usepackage[utf8]{inputenc}
\frenchspacing 
\title{Appunti Basi di Dati (Teoria)}
\author{Luca Seggiani}
\date{6 Marzo 2024}

\begin{document}
\maketitle
\section{Introduzione alla teoria delle basi di dati}
\textit{Che cos'è l'informatica?}
\begin{itemize}
  \item L'informatica è la scienza del trattamento razionale, spesso attraverso macchine automatiche,
    dell'informazione.
\end{itemize}
possiamo fare una distinzione fra:
\begin{itemize}
  \item metodologica
  \item tecnologica
\end{itemize}

\section{Sistema informativo}
\begin{itemize}
  \item Il sistema organizzativo è costituito da risorse e regole per lo svolgimento coordinato di attività
    di una certa organizzazione (azienda, ente, ecc...) Noi ci concentreremo sul
  \item Sistema informativo, ovvero la parte del sistema organizzativo che acquisice, conserva, elabora e produce
    informazioni d'interesse per l'organizzazione. Inoltre esegue e gestisce i processi informativi (che
    coinvolgono informazioni).
\end{itemize}

possiamo analizzare più nel dettaglio i tipo di attività svolte dal sistema informativo:
\begin{itemize}
  \item Raccolta e acquisizione di informazioni
  \item Archiviazione e conservazione di suddette informazioni
  \item Elaborazione, trasformazione e ancora produzione di nuove informazioni sulla base di quelle ottenute
  \item Distribuzione e comunicazione delle informazioni così elaborate.
\end{itemize}

Occorre fare attenzione: il sistema informativo non è di per se in alcun modo legato all'informatica (esistono
sistemi informativi, si pensi ai servizi anagrifici o alle banche, che non usano alcuna automatizzazione). \\
La parte del sistema informativo che usa la tecnologia informatica è il sistema informativo automatizzato, 
o semplicemente \textbf{sistema informatico}. \\
Ricapitolando, possiamo stabilire la seguente relazione:
\begin{center}
  Azienda $\rightarrow$ Sistema organizzativo $\rightarrow$ Sistema informativo $\rightarrow$ sistema informatico 
\end{center}
\section{Gestione delle informazioni}
L'informazione può essere gestita secondo modalità diverse, e su supporti diversi. Ad esempio, abbiamo:
\begin{itemize}
  \item Idee informali
  \item Linguaggio naturale
  \item Disegni, schemi, grafici
  \item Numeri e codici
\end{itemize}
su vari supporti:
\begin{itemize}
  \item Mente umana
  \item Carta
  \item Dispositivi elettronici (e.g. hard disk, ecc...)
\end{itemize}

Definiamo ora la differenza fra informazioni e dati:
\begin{itemize}
  \item \textbf{Informazione}\\
    Notizia, dato o elemento che consente di avere una conoscenza dei fatti, situazioni o modi di essere.
  \item \textbf{Dato} \\
    Ciò che è immediatamente presente alla conoscenza prima dell'elaborazione. In informatica, elementi di informazione
    costituiti da \textit{simboli} che devono essere elaborati.
\end{itemize}

I dati sono spesso codifiche particolari di informazioni, che vanno quindi da essi estrapolate. Ad esempio,
il codice fiscale, i cartelli stradali, ecc...

\section{La base di dati}
Il cuore di un sistema informativo automatizzato è la base di dati (database), cioè un insieme organizzato di
dati rappresentanti informazioni di interesse. Nelle due accezioni (metodologica e tecnologica), possiamo dire:
\begin{itemize}
  \item Metodologica: insieme organizzato di dati utilizzati come supporto per lo svolgimento di attività
  \item Tecnologica: insieme di dati gestito da un DBMS (Database Management System)
\end{itemize}

Le basi di dati hanno solitamente:
\begin{itemize}
  \item Dimensioni molto maggiori della memoria centrale dei sistemi di calcolo utilizzati
  \item Tempo di vita indipendente dalle signole istanze dei programmi che li utilizzano (persistenza dei dati)
  \item Supporto per gestione di collezioni di dati condivise fra più dispositivi
  \item Capacità di garantire privacy, affidabilità, efficienza ed efficacia
\end{itemize}

Vediamo nel dettaglio dell'aspetto di condivisione:
\begin{itemize}
  \item Ogni organizzazione è divisa in settori o almeno svolge disparate attività
  \item Ciascun settore potrebbe essere fornito di un sottosistema informativo, magari disgiunto a quello principale.
\end{itemize}
Questo può chiaramente portare a problemi di:
\begin{itemize}
  \item \textbf{Ridondanza}: ripetizione dell'informazione
  \item \textbf{Incoerenza}: più versioni dell'informazione che non coincidono.
\end{itemize}

Nota: perchè non usare un semplice archivio di file invece che di un database? \\
Un archivio non fornisce alcuna gestione dell'interdipendenza fra informazioni, ed è quindi poco portato
agli eventuali controlli sulla coerenza e la correttezza dell'informazione. Inoltre, in un comune filesystem
abbiamo a disposizione solamente le operazioni rudimentali di scrittura/lettura, senza particolari controlli
su concorrenza o funzionalità particolari.

\par\smallskip
Per ovviare a tutta questa serie di problemi, introduciamo:
\begin{itemize}
  \item \textbf{Autorizzazione}: gestione dell'accesso a date risorse
  \item \textbf{Concorrenza}: gestione dell'accesso \textit{simultaneo} (!) a date risorse
  \item \textbf{Affidabilità}: resistenza a malfunzionamenti hardware e software. Una tecnica fondamentale
    in questo campo è la corretta gestione delle \textbf{transazioni}
  \item \textbf{Efficienza}: gestione ottimale delle risorse in termini di memoria e tempo
  \item \textbf{Efficacia}: offerta di funzionalità articolate, potenti e flessibili.
\end{itemize}

Vediamo un'attimo nel dettaglio l'aspetto delle transazioni:
\par\smallskip
\textbf{Transazioni}\\
Una transazione è un insieme di operazioni da considerare indivisibili (\textit{"atomiche"}), corretto
anche in presenza di concorrenza, e con effetti definitivi a fine esecuzione. Una transazione deve essere
eseguita \textit{per intero} o \textit{per niente}. La corretta gestione della concorrenza deve invece assicurarsi
che transazioni concorrenti vengano gestite correttamente (o in serie, e. g. transazioni bancarie, o in maniera
mutualmente esclusiva, e. g. prenotazione biglietto aereo). La transazione dovrà poi essere permanente, ovvero la conclusione
positiva di una transazione corrisponde ad un impegno (commit) a mantenere traccia della versione ormai aggiornata
dei dati a seguito della stessa.
\par\smallskip
\textbf{Descrizione dei dati} \\
Nei programmi tradizionali che accedono a file, ogni programma contiene una descrizione della struttura del
file stesso, con i conseguenti rischi di incoerenza fra informazioni e file stessa. Ecco perchè nei DBMS è
opportuno dedicare una porzione della base di dati alla descrizione centralizzata dei dati. Introduciamo il 
concetto di \textbf{modello dei dati}: un insieme di costrutti utilizzati per organizzare i dati di interesse
e descriverne le varie dinamiche, fornendoci quindi una vista astratta dei suddetti. Occorre quindi definire
la differenza fra:
\begin{itemize}
  \item \textbf{Schema}: descrizione della struttura della dei dati, solitamente invariante nel tempo (attributi)
  \item \textbf{Istanza}: i valori attuali immagazzinati nella base di dati, che variano nel tempo (record).
\end{itemize}

introdiciamo l'elemento fondamentale della base di dati: la \textbf{tabella}. Una comune tabella di una base di dati
contiene come intestazioni delle sue colonne gli attributi (schema) dei dati immagazinati, mentre le successive righe
presentano vari record (istanze) dei dati definiti dalle colonne.

\end{document}

\documentclass[a4paper,12pt]{article}

\usepackage[french,italian]{babel}
\usepackage[T1]{fontenc}
\usepackage[utf8]{inputenc}
\frenchspacing 
\title{Appunti Basi di Dati (Teoria)}
\author{Luca Seggiani}
\date{7 Marzo 2024}

\begin{document}
\maketitle
\section{Modelli dei Dati}
\textbf{Modelli logici} \\
Adottati nei DBMS esistenti per l'organizzazione dei dati, indipendenti dalle strutture fisiche.
Esempi: relazionale, reticolare, gerarchico, a oggetti, basato su XML.
\par\smallskip
\textbf{Modelli concettuali} \\
Permettono di rappresentare i dati in modo indipendente da ogni sistema, cercando di descrivere i concetti
del mondo reale. Sono utilizzati nelle fasi preliminari di progettazione. \\
Esempio: Entity-Relationship (ER).
\par\smallskip
Possiamo dire che l'utente si interfaccia con lo schema logico di un database, dettato dal modello logico adottato.
A livello fisico accade quanto definito nello schema interno, ovvero la parte di effettiva implementazione del DBMS.
\par\smallskip
\textbf{Schema logico} \\
Lo schema logico è la descrizione della base di dati nel modello logico, quindi la struttura di tabelle, ecc...
L'utente che sviluppa il database si interfaccia con lo schema logico, usufruendo delle sue astrazioni.
\par\smallskip
\textbf{Schema interno} \\
Lo schema interno è l'implementazione dello schema logico, ed è noto solo a chi implementa il DBMS.

\section{Linguaggi per Basi di Dati}
I DBMS dispongono di vari linguaggi e interfacce per la definizione di schemi, modifica e lettura dei dati, e formmulazione
di query. Possiamo avere:
\begin{itemize}
  \item Linguaggi testuali interattivi (SQL)
  \item Comandi (SQL) immersi in un linguaggio ospite (Java, C++, ...)
  \item Interfaccie grafiche (Access)
\end{itemize}
Si può fare l'ulteriore distinzione:
\begin{itemize}
  \item \textbf{Data definition language} (DDL): per la definizione di schemi (logici e fisici)
  \item \textbf{Data manipulation language} (DML): per l'interrogazione e l'aggiornamento di istanze di basi di dati.
\end{itemize}

\section{Architettura a tre livelli per DBMS}
Possiamo complicare le cose introducendo, oltre allo schema interno e lo schema logico, un'ulteriore schema, lo schema esterno,
attraverso cui permetteremo agli utenti di interfacciarsi con la base di dati a livello ancora più alto (astratto). \\
Lo schema esterno immplementa fondamentalmente viste parziali (magari solo alcune tabelle) di database. Sarà in ogni caso importante
stabilire:

\begin{itemize}
  \item Indipendenza fisica: il livello logico e quello esterno sono indipendenti da quello fisico, come nel
    paradigma dell'astrazione procedurale, la realizzazione dello schema fisico può variare senza che debbano
    essere attuate modifiche dello schema logico.
  \item Indipendenza logica: il livello esterno è poi indipendente da quello logico, aggiunte o modifiche alle viste
    non richiedono modifiche al livello logico, e viceversa le modifiche dello schema logico lasciano inalterato lo schema esterno.
\end{itemize}

\section{Attori}
Possiamo adesso stabilire quali sono gli "attori" che implementeranno, lavoreranno su ed interagiranno con il DBMS:
\begin{itemize}
  \item Progettisti e realizzatori di DBMS
  \item Progettisti della base di dati e suoi amministratori
  \item Progettisti e programmatori di applicazioni
  \item Utenti:
    \begin{itemize}
      \item Utenti finali: eseguono applicazioni predefinite:
      \item Utenti casuali: eseguono operazioni noon previste a priori, usando linguaggi interattivi (SQL)
    \end{itemize}
\end{itemize}

\section{Modello relazionale}
Vale la pena riportare i 3 modelli logici storici, tra cui figura il modello relazionale che andremo a studiare:
\begin{center}
gerarchico, reticolare, relazionale
\end{center}
Più recentemente si è poi diffuso il paradigma ad oggetti, basato su XML, anche detto NoSQL (viene principalmente
usato su database di dimensioni particolarmente grandi).

I modelli gerarchici e reticolari non sono molto utilizzati per un particolare difetto: la gestione di relazioni
fra dati non viene gestita in maniera efficiente (si usa il meccanismo dei riferimenti, che sono però
fin troppo dipendenti dalla struttura fisica adottata). Nel modello relazionale, invece, tutto è basato sui valori,
completamente slegati alle specificità della struttura fisica. Gli stessi riferimenti, o più propriamente relazioni,
sono basati su valori condivisi fra più tabelle.
\par\smallskip
\textbf{Tabelle} \\
La tabella è l'entità fondamentale di un database relazionale. Come già visto prima, una tabella contiene righe (record)
e colonne (attributi), che rappresentano in un qualsiasi momento un istanza dello schema logico adottato.
\textbf{Relazioni} \\
  Le relazioni del database relazionale si basano sul concetto matematico di relazione. Poniamo ad esempio
due insiemi:
$$ D_1 = (a, b) $$
$$ D_2 = (x, y, z) $$
e il loro prodotto cartesiano:
$$ D_1 \times D_2 = ((a, x), (a, y), (a, z), (b, x), ...) $$
una certa relazione $R$ è:
$$ R \subseteq D_1 \times D_2 $$
sottoinsieme del loro prodotto cartesiano. Formalmente: \\
Dati n insiemi (anche non distinti) $D_1, D_2, ... , D_n$, e definito su di essi un prodotto cartesiano $D_1 \times D_2 \times ... \times D_n$,
ovvero l'insieme di n-uple ordinate $(d_1, d_2, ..., d_n)$ con $d_1 \in D_1, d_2 \in D_2, ... , d_n \in D_n$, una relazione
è un sottoinsieme del prodotto cartesiano fra gli insiemi. \\
Non esiste ordinamento fra le n-uple, e non esistono n-uple uguali. Gli insiemi $D_1, D_2, ... , D_n$ sono detti domini della relazione. 
\par\smallskip
\textbf{Struttura non posizionale} \\
Dando un certo nome ai domini della relazione, possiamo passare da quella che è effettivamente una struttura posizionale (le n-uple sono ordinate),
ad una struttura dove le posizioni dei domini (attributi) non sono più rilevanti.
\par\smallskip
\textbf{Riferimenti fra relazioni} \\
I riferimenti fra relazioni diverse sono rappresentati da valori dei domini che compaiono nelle n-uple.
\par\medskip
\textbf{Definizioni formali} \\
Abbiamo adesso tutti gli strumenti necessari a definire formalmente il modello relazionale:
\begin{itemize}
  \item \textbf{Schema di relazione}: 
    simbolo $R$ detto nome della relazione e un insieme di attributi $X = (A_1, ..., A_2)$ solitamente
    indicato con $R(X)$. A ciascun attributo $X$ è associato un dominio. Gli attributi non possono essere relazioni.
  \item \textbf{Schema di base di dati}: 
    un insieme di schemi di relazione con nomi diversi, solitamente indicato come $R = (R_1(X_1), ... , R_n(X_m))$.
\end{itemize}

\textbf{N-uple}\\
Una n-upla o tupla su insieme di attributi $X$ è una funzione che associa a ciascun attributo $A \in X$ un
elemento, o valore, nel dominio di A. Il simbolo $t[A]$ denota il valore della n-upla $t$ sull'attributo $A$.
Il simbolo $t[Y]$, con $Y \subseteq X$, denota il valore della n-upla $t$ sull'insieme di attributi $Y$.
N.B.: le n-uple non sono ordinate!

Definiamo allora le istanze di relazioni (e quindi di database):
\begin{itemize}
  \item \textbf{Istanza di relazione}: su uno schema $R(X)$, un insieme $r$ di n-uple su $X$
  \item \textbf{Istanza di base di dati}: su uno schema $R = (R_1(X_1), ... , R_n(X_m))$, un insieme di relazioni
    $r = (r_1, ... , r_2)$ dove ogni $r_i$ è una relazione sullo schema $R_i(X_i)$
\end{itemize}

Graficamente, una n-upla è una riga della tabella, e un istanza l'insieme di tutte le sue righe.
\par\medskip
\textbf{Informazione incompleta} \\
Il modello relazionale impone ai dati una struttura rigida, ma non è detto che i dati del mondo reale aderiscano
sempre a questa struttura. Si introduce allora la parola chiave NULL, per evitare di usare valori particolari
del dominio (0, ecc...), che potrebbero diventare significativi, o la cui gestione è comunque abbastanza
complicata. Il valore di un certo attributo $A$, ovvero $t[A]$,
potrà quindi appartenere al dominio $D_A$ oppure essere il valore NULL. Si possono inoltre imporre determinate restrizioni
sulla presenza dei valori nulli in una data relazione. Un valore NULL può modellizzare almeno 3 casi differenti:
\begin{itemize}
  \item valore sconosciuto
  \item valore inesistente
  \item valore senza informazione
\end{itemize}
N.B.: Il DBMS non fa alcuna distinzione fra valori NULL! \\
La limitazione sui valori che possono o non possono essere NULL torna utile ad esempio nel caso della definizione
di chiavi primarie di record, che devono essere fra di loro diverse ma comunque mai nulle.

\end{document}


\documentclass[a4paper,12pt]{article}

\usepackage[french,italian]{babel}
\usepackage[T1]{fontenc}
\usepackage[utf8]{inputenc}
\frenchspacing 
\title{Appunti Basi di Dati}
\author{Luca Seggiani}
\date{10 Aprile 2024}

\begin{document}
\maketitle
\section{Modellazione e progettazione concettuale}
La definizione di schemi adeguati per le basi di dati richiede metodologie precise per la modellazione accurata
della realtà che ci interessa. Bisogna tenere a mente che:
\begin{itemize}
  \item Non conviene concetrarsi subito sui dettagli;
  \item Conviene invece stabilire subito interdipendenze fra relazioni;
  \item Il modello relazionale sarà rigido una volta ultimato.
\end{itemize}
In generale si può dire che esiste sempre (e spesso è unico) uno schema che modella accuratamente e nel modo
più semplice possibile una certa realtà di interesse. La progettazione di basi di dati è solo una fase delloo
sviluppo di un sistema informativo (vedere a riguardo le prime lezioni di teoria). Bisogna quindi tenere conto 
di:
\begin{itemize}
  \item \textbf{Ciclo di vita} \textit{(lifecycle)} del sistema informativo, ovvero l'insieme
    delle attivita svolte da analisti, progettisti e utenti nello sviluppo e nell'uso del sistema informativo.
    Questa attività è iterativa, e quindi ciclica.
\end{itemize}
I passi ciclo di vita dovranno essere ben definiti attraverso linguaggi e modelli prestabiliti. Per le basi di
dati, in particolare, conviene adottare modelli di facile utilizzo, che consentano la decomposizione delle attività
in fasi (e/o livelli) distinti, e di utilizzare strategie e criteri di scelta nei vari passaggi.
\par\smallskip
\textbf{Modello a cascata} \\
Il modello a cascata (\textit{waterfall model}) le fasi sono sequenzialmente ordinate, etichettate, e non ripetibili. In ordine,
esse sono le seguenti:
\begin{enumerate}
  \item \textbf{Studio di fattibilità}:
    definizione di costi e priorità della produzione;
  \item \textbf{Raccolta e analisi dei requisiti}:
    studio delle proprietà del sistema che andranno implementate;
  \item \textbf{Progettazione}:
    progettazione di strutture dati e funzioni;
  \item \textbf{Realizzazione}:
    implementazione effettiva del codice;
  \item \textbf{Validazione e collaudo}:
    sperimentazione del prodotto;
  \item \textbf{Funzionamento}:
    il sistema diventa opertaivo in produzione (\textit{shipping}).
\end{enumerate}

Vediamo alcune fasi nel dettaglio.
\section{Raccolta e analisi dei requisti}
Questa fase può essere, a sua volta, divisa in due sotto-fasi:
\begin{itemize}
  \item \textbf{Acquisizione dei requisiti}: il reperimento dei requisiti è un'attività non standardizzata. Esistono
    più modalità:
    \begin{itemize}
      \item Direttamente dagli utenti:
      \begin{itemize}
        \item Interviste, focus group, recensioni, ecc...
        \item Documentazioni apposite;
      \end{itemize}
    \item Attraverso documentazioni preesistenti:
      \begin{itemize}
        \item Normative (legislazioni, regolamenti di settore);
        \item Regolamenti interni, procedure aziendali;
        \item Realizzazioni preesistenti.
      \end{itemize}
    \end{itemize}
    Esistono linguaggi per definire requisiti (\textit{UML}).
  \item \textbf{Analisi dei requisiti}: si analizzano i requisiti raccolti, spesso nella prospettiva di successive
    acquisizioni.
\end{itemize}

\par\smallskip
\textbf{Interazione con gli utenti} \\
Le problematiche dell'interazione con gli utenti possono essere:
\begin{itemize}
  \item Utenti diversi danno risposte diverse;
  \item Utenti a livello più alto hanno spesso una visione più ampia ma meno dettagliata;
  \item Spesso l'acquisizione di requisiti avviene per raffinazione.
\end{itemize}
Conviene quindi:
\begin{itemize}
  \item Effettuare spesso verifiche di comprensione e coerenza;
  \item Verificare anche attraverso esempi (sopratutto nei casi limite);
  \item Richiedere definizioni e classificazioni chiare e specifiche;
  \item Separare gli aspetti essenziali da quelli marginali (\textit{ranking}).
\end{itemize}

\par\smallskip
\textbf{Interazione con gli utenti tramite documentazione} \\
E' opportuno seguire alcune linee guide generali, assicurando:
\begin{itemize}
  \item Standardizzazione della struttura delle frasi;
  \item Separazione delle frasi riguardanti dati da quelle riguardanti funzioni;
  \item Organizzazione di termini e concetti:
    \begin{itemize}
      \item Unificazione di termini (eliminazione di sinonimi);
      \item Esplicitazione del riferimento fra termini;
    \end{itemize}
  \item Organizzazione delle frasi per concetti.
\end{itemize}
\par\smallskip
\section{Progettazione}
La progettazione è una fase del ciclo di vita. Per un sistema software la progettazione si divide effettivamente in:
\begin{itemize}
  \item Progettazione dei dati;
  \item Progettazione delle applicazioni.
\end{itemize}
\textbf{Progettazione per astrazione} \\
Come in tutte le applicazioni informatiche, abbiamo visto che è necessario progettare per livelli successivi di astrazione:
\begin{itemize}
  \item \textbf{Livello concettuale}: esprime i requisiti di un sistema in una descrizione adatta all'analisi
    da punti di vista esterni
  \item \textbf{Livello logico}: evidenzia l'organizzazione dei dati dal punto di vista del loro contenuto informativo,
    descrivendo la struttura dei record e le loro interdipendenze.
  \item \textbf{Livello fisico}: si concentra sulla base di dati vista come un insieme di blocchi fisici sul disco,
    e riguarda quindi l'allocazione dei dati e le modalità di memorizzazione.
\end{itemize}
Con riferimento a quanto detto sugli schemi logici avremo quindi che la progettazione si divide in:
\begin{itemize}
  \item \textbf{Progettazione concettuale}, parte dai requisiti individuati della base e produce uno schema concettuale;
  \item \textbf{Progettazione logica}, parte dallo schema concettuale e produce uno schema logico;
  \item \textbf{Progettazione fisica}, parta dallo schema logico e produce lo schema fisico finale.
\end{itemize}
Come lo era stato il modello a cascata, la progettazione per astrazione è composta da fasi sequenzialmente ordinate che
vanno eseguite strettamente in ordine.
\par\smallskip
\textbf{Modello dei dati} \\
Il modello dei dati è l'insieme dei costrutti utilizzati per organizzare i dati di interesse e definirne
la dinamica. Componente fondamentale del modello sono i meccanismi di strutturazione (costruttori di tipi).
Come per i linguaggi di programmazione comuni, esistono meccanismi che permettono di definire nuovi tipi. Ogni 
modello dei dati prevede alcuni costruttori: ad esempio il modello relazionale prevede un costruttore relazionale che
permette di definire insiemi di record omogenei.
Per riassumere in breve, in ogni base di dati si ha:
\begin{itemize}
  \item Lo \textbf{schema}, invariante nel tempo, che ne descrive la struttura. Si notino inoltre le
    \textbf{intestazioni} delle tabelle (previste nel modello relazionale).
  \item L'\textbf{istanza}, i valori effettivi assunti dalla base di dati in un dato momento. Nel modello
    relazionale rappresenta il \textbf{corpo} di ciascuna tabella.
\end{itemize}
\par\smallskip
\textbf{Modello concettuale Entity-Relationship} \\
Il modello concettuale che utilizzeremo sarà quello \textbf{Entity-Relationship (ER)}. Si noti che la parola \textit{relationship},
sebbene abbia lo stesso significato in lingua inglese, non si riferisce al concetto matematico di relazione che sta alla base del 
modello relazionale. Per questo motivo useremo sempre il termine inglese per descrivere le relazioni del modello
entity-relationship, in modo da distinguerle dalle relazioni del modello relazionale. Il modello entity-relationship
viene sviluppato da P.P. Chen nel 1976, ed è oggi una delle metodologie più affermate nel campo della progettazione dei sistemi informatici,
anche se in un'accezione leggermente diversa da quella in cui era stato concepito inizialmmente. \\ I suoi costrutti base sono:
\begin{itemize}
  \item Entità
  \item Relationship
  \item Attributi
\end{itemize}
\par\smallskip
\textbf{Entità} \\
Un'entità è una classe di oggetti dell'applicazione d'interesse con proprietà comuni e esistenza autonoma.
Un'occorrenza (o istanza) di entità è un elemento della classe (un'elemento, non i dati ad esso legati!). Ogni
entità deve avere un nome che la identifica univocamente nello schema. Graficamente è rappresentato da una scatola.
\par\smallskip
\textbf{Relationship} \\
Una relationship è un legame logico fra due o più entita, rilevante nell'applicazione d'interesse. Può essere chiamata
anche relazione (vedi sopra), correlazione o associazione. Ogni relationship, come per le entità, ha un nome
che la identifica univocamente nello schema. Graficamente è rappresentata da una losanga. \\
Vediamo allora di definire il concetto di occorrenza di relationship, più complesso di quello di occorrenza di entità:
\begin{itemize}
  \item Un'occorrenza di \textbf{relationship binaria} è una coppia di occorrenza di entità, una per ciascuna entità coinvolta.
  \item Una occorrenza di una \textbf{relationship n-aria} è una n-upla di occorenze di entità, una per ciascuna delle $n$ entità coinvolte.
\end{itemize}
Nell'ambito di una relationship non ci possono essere occorrenze (né coppie né n-uple) ripetute.
\par\smallskip
\textbf{Attributo} \\
L'attributo è una proprietà elementare di un'entità o di una relationship, che ci interessa ai fini dell'applicazione
d'interesse. Associa a ogni occorrenza di entità o relationship un valore appartenente ad un dominio, il cosiddetto
dominio dell'attributo.
\par\smallskip
\textbf{Attributo composito} \\
Gli attributi commpositi raggruppano attributi di una medesima entità o relatioship che presentano affinità nel loro
significato o uso (e.g. giorno, mese, anno si compone in data, via, numero civico, CAP in indirizzo, ecc...).
\end{document}

\documentclass[a4paper,12pt]{article}

\usepackage[french,italian]{babel}
\usepackage[T1]{fontenc}
\usepackage[utf8]{inputenc}
\frenchspacing 
\title{Appunti Basi di Dati (Teoria)}
\author{Luca Seggiani}
\date{13 Marzo 2024}

\begin{document}
\maketitle
\section{Linguaggi per basi di dati}
I linguaggi per basi di dati appartengono a 2 categorie distinte:
\begin{itemize}
  \item Operazioni sullo schema: \textbf{Data Definition Language (DDL)};
  \item Operazioni sui dati: \textbf{Data Manipulation Lanaguage (DML)}
  \begin{itemize}
    \item Interrogazione (query)
    \item Aggiornamento (update).
  \end{itemize}    
\end{itemize}
\par\smallskip
\textbf{Interrogazioni di basi di dati} \\
Un operazione di interrogazione è un operazione di lettura che richiede l'accessso a una o più tabelle. Per specificare
interrogazioni si possono seguire due formalismi:
\begin{itemize}
  \item \textbf{Modo dichiarativo}: si specificano le proprietà del risultato (che cosa)
  \item \textbf{Modo procedurale}: si specificano le modalità di generazione del risultato (come).
\end{itemize}
L'\textbf{algebra relazionale} permette di specificare delle interrogazioni secondo il modello procedurale: cioè elencando
i passi "primitivi" necessari alla generazione di una risposta. Il \textbf{calcolo relazionale} invece ci permette di definire
in modo dichiarativo quello che sarà il risultato dell'interrogazione. Come vedremo, sarà il calcolo relazionale a definire
la semantica del linguaggio relazionale, perchè permette di fornire un'implementazione slegata dai dettagli procedurali.

\section{Algebra relazionale}
Un'algebra è una struttura matematica dotata di un'insieme di dati e una serie di operatori che manipolano suddetti dati.
Nell'algebra relazionale abbiamo che:
\begin{itemize}
  \item \textbf{Dati}: relazioni
  \item \textbf{Operatori}:
    \begin{itemize}
      \item su relazioni
      \item che producono relazioni
      \item che possono essere composti.
    \end{itemize}
    essi sono divisi fra:
    \begin{itemize}
      \item \textbf{Operatori su insiemi}: \\
        unione, intersezione, differenza
      \item \textbf{Operatori su relazioni}: \\
        ridenominazione, selezione, proiezione, \\
        join:
        \begin{itemize}
          \item naturale, prodotto cartesiano, theta
        \end{itemize}
    \end{itemize}
\end{itemize}
\par\smallskip
\textbf{Notazione dell'algebra relazionale} \\
Specifichiamo adesso la notazione usata:
\begin{itemize}
  \item $R,R_1,...$ indicano nomi di relazionie
  \item $A,B,C,A_1...$ indicano nomi di attributo
  \item $XY$ è un'abbreviazione di $X \cup Y$
  \item Una relazione con n-uple $t_1, t_2,...$ è indicata con l'insieme $\{ t_1, t_2, ...\} $
  \item $t_j[A_j]$ indica il valore della n-upla $t_j$ sull'attributo $A_j$
  \item $t[X]$ indica l'n-upla ottenuta da considerando solo gli elementi di $X$.
\end{itemize}
\par\smallskip
\textbf{Operatori su insiemi} \\
Le relazioni sono insiemi di tuple, e non possono avere elementi duplicati (sarebbero altrimenti multiinsiemi). I risultati
di operazioni fra relazioni sono  a loro volta relazioni, ovvero insiemi di tuple. Gli operatori fra relazioni
possono applcarsi solo e soltanto fra relazioni definite sullo stesso insieme di attributi $X$, e il risultato sarà
a sua volta definito sullo stesso insieme di attributi $X$.
\par\smallskip
\textbf{Unione} \\
L'operatore di unione fra due relazioni sull'insieme di attributi $X$ comporta la formazione di una nuova 
relazione che ha tutte le n-uple delle relazioni unite. Eventuali n-uple identiche fra le relazioni appariranno nel risultato una volta sola.
Il suo simbolo è $\cup$.
\par\smallskip
\textbf{Intersezione} \\
L'operatore di intersezione fra due relazioni sull'insieme di attributi $X$ restituisce una nuova relazione che contiene
soltanto gli elementi appartenenti sia alla prima che alla seconda relazione.
Il suo simbolo è $\cap$.
\par\smallskip
\textbf{Ðifferenza} \\
L'operatore di differenza fra due relazioni sull'insieme di attributi $X$ restituisce una nuova relazione contenente tutte
le n-uple che appartengono alla prima relazione ma non alla seconda. L'operatore di differenza è l'unico fra gli operatori
insiemistici a non essere commutativo. Il suo simbolo è il meno ($-$), o più propriamente \textbackslash.
\par\smallskip
Notiamo che l'unione è l'unico operatore in grado di creare relazioni con un numero di elementi maggiore di quello degli operandi. Questa
caratteristica tornerà poi utile nel calcolo relazionale. 
\par\smallskip
Descriviamo adesso gli operatori non insiemistici (sulle relazioni):\\
\textbf{Ridenominazione} \\
L'operatore di ridenominazione è un'operatore monadico che modifica lo schema dell'operando, lasciando inalterata
l'istanza. Si scrive, data una relazione $R$, come:
$$ \rho_{B_1B_2...} \leftarrow_{A_1A_2...(R)} $$
I pedici a sinistra e a destra della freccia sono insiemi di attributi. Avremo che:
\begin{itemize}
  \item L'attributo $A_1$ viene sostituito dall'attributo $B_1$
  \item L'attributo $A_2$ viene sostituito dall'attributo $B_2$
  \item ecc...
\end{itemize}
\par\smallskip
\textbf{Selezione} \\
L'operatore di selezione è un operatore monadico che produce un un risultato con lo stesso
schema dell'operando, e un sottoinsieme di n-uple che rispettano una determinata condizione. Si scrive come:
$$ \sigma_F(R)$$
sulla relazione $R$, dove $F$ è un'espressione Booleana (predicato) ottenuta componendo con gli operatori logici AND, OR e NOT
le condizioni atomiche:
\begin{itemize}
  \item $A \star B$, dove $A$ e $B$ sono attributi di $X$ con domini compatibili e $\star$ un operatore di confronto.
  \item $A \star k$, dove $A$ è un attributo di $X$ e $k$ una costante con dominio compatibile con $A$ e $\star$ sempre un operatore di confronto. 
\end{itemize}
Notiamo che la condizione atomica è vera solo per valori non nulli in ogni attributo.
\par\smallskip
\textbf{Selezione valori NULL} \\
Per riferirsi ai valori NULL occorre usare le apposite condizioni: IS NULL e IS NOT NULL.
\par\smallskip
\textbf{Proiezione} \\
La proiezione è un'operatore monadico che produce un sottoinsieme degli attributi dell'operando contenente tutte
le sue n-uple ristrette soltanto ad alcuni attributi. Si scrive, data una relazione $R(X)$ e un insieme
di attributi $Y \subseteq X$, come:
$$ \pi_Y(R) $$
Il risultato è una relazione su $Y$ che contiene l'insieme delle n-uple di $R$ ristrette ai soli attributi di $Y$.
Notare che, di nuovo, non possono esserci righe ripetute. Ogni n-upla ripetuta nell'insieme risultato verrà inserita una volta sola.
\par
\textbf{Cardinalità di proiezioni} \\
Una proiezione contiene al massimo tante n-uple quante ne contiene l'operando. Inolre, se l'insieme di attributi $X$ è superchiave
della relazione $R$, allora la proiezione $\pi_X(R)$ avrà tante n-uple quante ne ha l'operando.
\par\smallskip
\textbf{Join} \\
Non possiamo, usando gli operatori di selezione e proiezione, estrarre e combinare informazioni da più relazioni diverse fra loro,
e non possiamo nemmeno combinare informazioni presenti in n-uple diverse di una stessa relazione. Avremo allora bisogno
di un'ulteriore serie di operazioni, le cosiddette operazioni di join. I join ci permettono di unire sulla stessa riga più righe di
relazioni diverse. Vediamo nel dettaglio:
\par\smallskip
\textbf{Join naturale} \\
Il join naturale è un'operatore con due operandi (generalizzabile), che produce come risultato l'unione degli attributi degli operandi,
contente le n-uple costruite ciascuna a partire da un n-upla di ognuno degli operandi. Formalmente:
Date due relazioni $R_1{X_1}$ e $R_2{X_2}$, il loro join naturale si scrive come:
$$ R_1 \bowtie R_2 $$
Il risultato è una relazione $R(X_1 \cup X_2)$ definita come:
$$R(X_1 \cup X_2) = R_1(X_1) \bowtie R_2(X_2) = \{ t | \exists t_1 \in R_1, \quad t_2 \in R_2 $$
$$ \mathrm{con} \quad t[X_1]=t_1, \quad t[X_2] = t_2 \}$$
N.B.: è perfettamente plausibile voler fare il join naturale tra due relazioni senza attributi in comune,
e in questo caso si ottiene il prodotto cartesiano fra le due.
\par\smallskip
\textbf{Cardinalità del join} \\
Il join di due relazioni $R_1$ e $R_2$ contiene un numero di n-uple compreso fra 0 e il prodotto di $|R_1|$ e $|R_2|$.
Se il join coinvolge una chiave di $R_2$ allora il numero di n-uple è compreso fra 0 e $|R_1|$. Se il join coinvolge una
chiave di $R_2$ e un vincolo di integrità referenziale allora il numero di n-uple è uguale a $|R_1|$. Più formalmente: \\
Il join di $R_1(A, B)$ e $R_2(B,C)$ contiene un un numero di n-uple:
$$ 0 \leq |R_1 \bowtie R_2| \leq |R_1| \times |R_2| $$
Se $B$ è una chiava di $R_2$ allora il numero di n-uple è 
$$ 0 \leq |R_1 \bowtie R_2| \leq |R_1|$$
Se $B$ è una chiave di $R_2$ ed esiste un vincolo di integrità referenziale fra $B$ (in $R_1$) e $R_2$ allora il numero delle
n-uple è:
$$ |R_1 \bowtie R_2| = |R_1| $$
Una problematica del join è il fatto che le n-uple che non contribuiscono al risultato ("che non fanno join") restano tagliate fuori.
Introduciamo allora altri tipi di operatori di join, i cosiddetti:
\par\smallskip
\textbf{Join esterno} \\
Il join esterno estende, con valori NULL, le n-uple che verrebbero altrimenti scartate dal join (interno).
Ne esistono tre versioni:
\begin{itemize}
  \item \textbf{Sinistro}: mantiene tutte le n-uple del primo operando ($\bowtie_{LEFT}$);
  \item \textbf{Destro}: mantiene tutte le n-uple del secondo operando ($\bowtie_{DESTRO}$);
  \item \textbf{Completo}: mantiene tutte le n-uple di entrambi gli operandi ($\bowtie_{FULL}$).
\end{itemize}
\textbf{Join e proiezioni} \\
Date due relazioni $R_1(X_1)$ e $R_2(X_2)$:
$$ \pi_{X_1}(R_1 \bowtie R_2) \subseteq R_1 $$
Date una relazione  $R(X)$ con $X = X_1 \cup X_2$
$$ (\pi_{X_1}(R) \bowtie \pi_{X_2}(R)) \supseteq R$$
\textbf{Prodotto cartesiano} \\
Come detto prima, date due relazioni $R_1(X_1)$ e $R_2(X_2)$, senza attributi in comune, cioè con $X_1 \cap X_2 = {0}$, la
definizione di join naturale e comunque senso e restituisce il prodotto cartesiano delle due relazioni:
$$ R = R_1 \bowtie R_2 = R_1 \times R_2 $$
la cardinalità del prodotto cartesiano è uguale al prodotto delle cardinalità delle due relazioni.
\par\smallskip
\textbf{Theta join} \\
Nella pratica, il prodotto cartesiano ha senso quasi solamente se seguito da una selezione $\sigma_F(R_1 \times R_2)$.
Questa combinazione prende il nome di theta join (è un operatore derivato) ed è indicato come:
$$ R_1 \bowtie_F R_2 $$
dove $F$ è un certo rpedicato. Spesso $F$ è una congiunzione di atomi di confronto $A_1 \sigma A_2$ dove $\sigma$ è
un operatore di confronto (<=, <, =, ecc...) e $A_1$ e $A_2$ attributi di relazioni diverse. Quando l'operatore
di confronto è l'uguaglianza parliamo di equi-join.

\end{document}

\documentclass[a4paper,12pt]{article}

\usepackage[french,italian]{babel}
\usepackage[T1]{fontenc}
\usepackage[utf8]{inputenc}
\frenchspacing 
\title{Appunti Basi di Dati}
\author{Luca Seggiani}
\date{14 Marzo 2024}

\usepackage{forest}

\begin{document}
\maketitle
Tutti gli operatori visti finora, ovvero gli operatori insiemistici e quelli relazionali, bastano a realizzare
qualsiasi possibile interrogazione. Ogni altro operatore sarà un' operatore derivato dei 5 operatori relazionali
e i 3 poeratori insiemistici. Un particolare operatore derivato è:
\section{Divisione}
Dati due insiemi di attributi disgiunti $X_1$ e $X_2$, una relazione $r$ sulla loro unione e una relazione
$r_2$ su $X_2$, la divisione $r \div r_2$ è una relazione su $X_1$ che contiene le n-uple ottenute 
come "proiezione" di n-uple di $r$ che si combinano con tutte le n-uple di $r_2$ per formare n-uple di $r$, in una 
sorta di prodotto cartesiano inverso. In simboli:
$$ r \div r_2 = \{ t_1[X_1] \quad | \quad \forall t_2 \in r_2 \quad \exists t \in r : t[X_1] = t_1, \quad t[X_2] = t_2 \}$$
possiamo dimostrare che è un operatore derivato definendolo come composizione di operatori fondamentali, in questo modo:
$$ r \div r_2 = \pi_{X_1}(r) - \pi_{X_1}((\pi_{X_1}(r)\times r_2)-r)$$
Ovvero:
\begin{itemize}
  \item $ \pi_{X_1}(r) \times r_2 $ contiene tutte le n-uple di $\pi_{X_1}(r)$ "estese" con tutti i possibili valori
    di $r_2$.
  \item $(\pi_{X_1}(r)\times r_2)-r$ contiene le "estensioni" di $\pi_{X_1}(r)$ che non compaiono in $r$.
  \item $\pi_{X_1}((\pi_{X_1}(r)\times r_2)-r)$ contiene le n-uple di $\pi_{X_1}(r)$ per le quali un qualche
    completamento con $r_2$ non compare in $r$.
  \item Togliendo queste ultime n-uple a $\pi_{X_1}(r)$  otteniamo tutte le n-uple di $\pi_{X_1}(r)$ che si combinano con tutte
    le n-uple di $r_2$.
\end{itemize}
\section{Chiusura transitiva}
Poniamoci il problema di dover trovare, in un'opportuna tabella supervisione formata da matricole di impiegati e supervisori di tali impiegati,
le matricole di tutti i supervisori di un dato impiegato (ammettendo che i supervisori siano impiegati e possano a loro volta
avere supervisori). Tale richiesta sarebbe perfettamente valida, ma impossibile da esprimere attraverso gli operatori dell'algebra relazionale. \\
Nell'algebra relazionale non esiste la possibilità di esprimere interrogazioni che calcolino la chiusura transitiva di una relazione
arbitraria. Tale operazione potrebbe infatti richiedere un numero infinito di di join (join illimitato). 
\section{Espressioni equivalenti} 
Due espressioni sono equivalenti se producono lo stesso risultato qualunque sia l'istanza fornitagli. In questo caso, sarà opportuno
scegliere espressioni di costo minore, dove il loro "costo" è determinato dalle dimensioni delle istanze intermedie
che la loro esecuzione genera. Vediamo alcune equivalenze fondamentali:
\begin{itemize}
  \item \textbf{Atomizzazione delle selezioni}: una congiunzione di selezioni può essere sostituita da una sequenza
    di selezioni atomiche:
    $$ \sigma_{F_1 \land F_2}(E) = \sigma_{F_1}(\sigma{F_2}(E))$$
  \item \textbf{Idempotenza delle proiezoni}: una proiezione può essere trasformata in una sequenza di proiezioni:
    $$ \pi_{X}(E) = \pi_{X}(\pi_{XY}(E))$$
  \item \textbf{Push selections down}: se una condizione $F$ coinvolge solo attributi dell'espressione $E_2$:
    $$ \sigma_{F}(E_1 \bowtie E_2) = E_1 \bowtie \sigma_{F}(E_2) $$
  \item \textbf{Push projections down}: se un'espressione $E_1$ ha attributi $X_1$, un'espressione $E_2$ ha attributi
    $X_2$, $Y_2 \subseteq X_2$ e gli attributi $X_2 - Y_2$ non sono coinvolti nel join ($X_1 \cap X_2 \subseteq Y_2$):
    $$ \pi_{X_1Y_2}(E_1 \bowtie E_2) = E_1 \bowtie \pi_{Y_2}(E_2) $$
\end{itemize}
\section{Ottimizzazione delle interrogazioni}
Un modulo presente nel DBMS è il \textbf{query processor} (od ottimizzatore). L'ottimizzatore si occupa di scegliere
la strategia realizzativa a partire dall'istruzione in linguaggio dichiarativo di alto livello, tenendo conto
del costo di implementazioni diverse. Le fasi di eseecuzione di una query saranno:
\begin{itemize}
  \item Analisi lessicale, sintattica e semantica della query in linguaggio di alto livello (SQL). Al termine
    di questa analisi, la query verrà tradotta in un'espressione dell'algebra relazionale.
  \item Ottimizzazione algebraica: a questo punto verranno calcolate una o più nuove espressioni algebriche
    equivalenti a quella di partenza, sfruttando le equivalenza sopra descritte.
  \item Ottimizzazoine basata sui costi: fra le alternative calcolate prima, viene selezionata la più efficiente,
    che viene poi utilizzata per effettuare l'interrogazione effettiva sulla base di dati.
\end{itemize}
Nella prima e l'ultima di queste fasi, il DMBS interagisce con un componente detto catalogo, che contiene informazioni
sugli schemi contenuti nel database, la cardinalità delle loro istanze, ecc..
\par\medskip
\textbf{Profili delle relazioni} \\
Tra le informazioni quantitative memorizzate nel catalogo troviamo:
\begin{itemize}
  \item Cardinalità di ciascuna relazione;
  \item Dimensioni delle n-uple;
  \item Dimensioni dei valori;
  \item Numero di valori distinti degli attributi;
  \item Valore minimo e massimo degli attributi.
\end{itemize}
Queste informazioni vengono usate nella fase di ottimizzazione basata sui costi per stimare le dimensioni dei risultati
intermedi di più espressioni algebriche alternative generate dall'ottimizzazione algebrica.
\par\smallskip
\textbf{Ottimizzazione algebrica} \\
In verità, il termine ottimizzazione non è completamente accurato: il processo utilizza infatti delle euristiche per
trovare risultati migliori. Si basa sul concetto di equivalenza per trovare query che restituiscono lo stesso
risultato generando dimensioni d'istanza intermedie minori. Ad esempio, un'ottimizzazione tipica è quella
di eseguire selezioni e proiezioni il più presto possibile, ovvero le cosiddette "push selections down" e "push
projections down".
\section{Grafi}
Un grafo $G=(V,E)$ consiste in un insieme $V$ di vertici (o nodi) e un insieme $E$ di coppie di vertici, detti archi.
Ogni arco ovviamente connetter fra loro due vertici. Possiamo allora distinguere:
\begin{itemize}
  \item \textbf{Grafi orientati}: detti anche grafi diretti, dove ogni arco è orientato è rappresenta relazioni ordinate fra oggetti;
  \item \textbf{Grafi non orientati}: detti anche grafi indiretti, dove ogni arco non è orientato e rappresenta relazioni simmetriche fra oggetti.
\end{itemize}
Sui grafi si possono definire \textbf{cammini} da un vertice $x$ ad un vertice $y$. Formalmente, un cammino è:
$$ (v_0, ... , v_k) \quad \mathrm{di} \quad V, \quad v_0 = x, \quad v_k = y \quad | \quad 1 \leq i \leq k : (v_{i-1}, v_i) \in E$$
Un cammino che torna da dove parte, ovvero dove $(v_0, ... , v_k) : v_0 = v_k$, è detto ciclico. Si dice che un grafo
diretto privo di cicli è \textbf{aciclico}.
\par\smallskip
\textbf{Alberi} \\
Un grafo non orientato si dice connesso se esiste un cammino fra ogni coppia di vertici. Un'albero è un grafo non orientato
nel quale due vertici qualsiasi sono connessi da uno e un solo cammino.
\par\medskip
Un'interrogazione può essere rappresentata da un'albero, dove le foglie (i nodi finali) sono dati (relazioni, file, ecc...),
e i nodi intermedi sono operatori (operatori algebrici, poi effettivi operatori di accesso ai dati). Ad esempio,
l'albero corrispondente all'espressione:
$$ \sigma_{A=10}(R_1 \bowtie R_2) $$
sarà:
\begin{center}
\begin{forest}
  [$\sigma_{A=10}$
    [
      $\bowtie$
      [$R_1$]
      [$R_2$]
    ]
  ]
\end{forest}
\end{center}
La stessa interrogazione dopo il push down della selezione $\sigma_{A=10}$ sarà allora:
$$ R_1 \bowtie \sigma{A = 10}(R_2)$$
con relativo albero:
\begin{center}
\begin{forest}
    [
      $\bowtie$
      [$R_1$]
      [$\sigma_{A=10}$
      [$R_2$]
      ]
    ]
\end{forest}
\end{center}

\par\smallskip
\textbf{Procedura euristica di ottimizzazione} \\
La procedura di ottimizzazione consiste quindi nel:
\begin{itemize}
  \item Decomporre le selezioni congiuntive in successive selezioni atomiche;
  \item Anticipare il più possibile le selezioni;
  \item In una sequenza di selezioni, anticipare le più selettive;
  \item Combinare prodotti cartesiani e selezioni per formare join;
  \item Anticipare il più possibile le proiezioni.
\end{itemize} 

\end{document}

\documentclass[a4paper,12pt]{article}

\usepackage[french,italian]{babel}
\usepackage[T1]{fontenc}
\usepackage[utf8]{inputenc}
\frenchspacing 
\title{Appunti Basi di Dati}
\author{Luca Seggiani}
\date{21 Marzo 2024}

\begin{document}
\maketitle
\section{Relazioni Derivate}
Una \textbf{relazione di base} è una relazione il cui contenuto è autonomo. Una \textbf{relazione derivata} è una relazione
il cui contenuto è funzione di altre relazioni. Si possono cosi avere rappresentazioni diverse per gli stessi dati, definite
mediante interrogazioni, che possono rivolgersi a relazioni di base come ad altre relazioni derivate. Esistono due tipi
di relazioni derivate:
\begin{itemize}
  \item \textbf{Viste materializzate}, dove i risultati sono precalcolati;
  \item \textbf{Viste virtuali}, o più semplicemente viste. Sono le più comuni.
\end{itemize}
\textbf{Viste materializzate} \\
Sono relazioni derivate memorizzate nella base di dati. Il loro vantaggio è che sono precalcolate, e quindi
immediatamente disponibili. I loro svantaggi sono che:
\begin{itemize}
  \item Sono ridondanti;
  \item Appesantiscono gli aggiornamenti;
  \item Sono raramente supportate dai DBMS.
\end{itemize}

\textbf{Viste virtuali} \\
Relazioni derivate non memorizzate nella base di dati, e quindi ricalcolate ad ogni accesso.
Sono supportate da tutti i DBMS.

\par\smallskip
\textbf{Interrogazioni su viste} \\
Le interrogazioni su viste sono eseguite sostituendo alla vista la sua definizione. Le viste possono semplificare
la scrittura di interrogazioni, espressioni complesse, e sotto-espressioni ripetute. L'uso delle viste virtuali
inoltre non influisce sull'efficienza delle interrogazioni. Ad esempio, supponiamo di avere le seguenti relazioni:
$$ R_1(ABC), R_2(DEF), R_3(GH) $$
e di definire una vista $R$:
$$R = \sigma_{A>D}(R_1 \bowtie R_2)$$
Un'interrogazione potrà a questo punto essere definita:
\begin{itemize}
  \item Senza vista:
    $$ \sigma_{B=G} (\sigma_{A>D}(R_1 \bowtie R_2) \bowtie R_3)$$
\\item Con vista:
    $$ \sigma_{B=G} (R \bowtie R_3)$$
\end{itemize}
\par\smallskip
\textbf{Viste e aggiornamenti} \\
Aggiornare una vista significa modificare le relazioni di base in modo che la vista ricalcolata rispecchi
l'aggiornamento. L'aggiornamento sulle relazioni di base dovrebbe essere associato univocamente a quello
specificato sulla vista, ma questo non è sempre vero. Per questo motivo ben pochi aggiornamenti sono ammissibili
sulle viste.
\section{Calcolo relazionale} 
Il calcolo relazionale è una famiglia di linguaggi dichiaratvi basati sul calcolo dei predicati del primo
ordine. Ne esistono diverse versioni:
\begin{itemize}
  \item Calcolo relazionale sui domini, in breve calcolo dei domini;
  \item Calcolo su n-uple con dichiarazione di range, in breve calcolo delle tuple, usato in SQL.
\end{itemize}
\textbf{Assunzioni} \\
I simboli di predicato corrispondono alle relazioni presenti nelle basi di dati, più alcuni predicati standard
come l'uguaglianza e la disuguaglianza. Non compaiono simboli di funzione.
Nel calcolo relazionale vengono utilizzate prevalentemente formule aperte, cioè formule con variabili libere,
il cui valore di verità dipende dai valori assegnati alle variabili libere. Il risultato di un'interrogazione
(formula aperta) è costituito da tutte le tuple di valori che sostituiti alle variabili libere, la rendono vera.
In coerenza con quanto fatto in algebra relazionale, useremo una notazione non posizionale (attributi con nome, etichettati).
\par\smallskip
\textbf{Calcolo dei domini} \\
Vediamo innanzitutto la sintassi: nel calcolo dei domini un'espressone ha forma:
$$ \{ A_1: x_1,....,A_k:x_k\ | f\} $$
dove:
\begin{itemize}
  \item $A_1,...,A_k$ sono attributi distinti (che possono anche non comparire nello schema della base di dati).
  \item $x_1,...,x_K$ sono variabili (che assumiamo essere distinte, non è necessario).
  \item $A_1:x_1,...,A_k:x_k$ è la \textit{target list} (lista degli obiettivi) e descrive il risultato.
  \item $f$ è una formula costruita a partire da formule atomiche utilizzando eventualmente i connettivi booleani
    e i quantificatori logici $\forall x$, $\exists x$, con $x$ variabile (logica del primo ordine).
\end{itemize}
Il risultato di un'espressione nel calcolo dei domini è una relazione su $A_1,...,A_k$, contenente
n-uple in $x_1,...,x_2$ che rendono vera la formula $f$ sull'istanza di basi di dati considerata.
$F$ rispetto all'istanza 
\par\smallskip
\textbf{Formule atomiche del calcolo dei domini} \\
Il predicato $f$ è costituito da formule atomiche. Una formula atomica è definita come una delle tre forme:
\begin{itemize}
  \item \textbf{Schemi di relazione}
    $$ R(A_1 : x_1 ... , A_p : x_p)$$ dove $R(A_1,...,A_P)$ è uno schema di relazione.
  \item \textbf{Operatori fra variabili}
    $$ x_i  \ OP \  x_j $$
    dove $x$ e $y$ sono variabili e $OP$ è un operatore di confronto (uguale, non uguale, maggiore, minore e varianti strette).
  \item \textbf{Operatori fra variabili e costanti}
    $$ x_i \ OP \ c, \quad c \ OP \ x_i $$
    dove $c$ è una costante nel dominio $A_i$ di $x_i$.
\end{itemize}
Definiamo allora alcune proprietà delle formule:
\begin{itemize}
  \item Le formule atomiche sono formule.
  \item Se $f$ è una formula, allora anche $\neg f$ è una formula.
  \item Se $f_1$ e $f_2$ sono formule, allora anche $f_1 \wedge f_2$ e $f_1 \vee f_2$ sono formule.
  \item Se $f$ è una formula e $x$ una variabile, allora anche $\exists x(f)$ e $\forall x(f)$ sono formule,
    dove $\exists$ and $\forall$ sono qualificatori (rispettivamente esistenziale e universale).
  \item Per convenienza, si raggruppano le variabili usate da quantificatori simili: ad esempio,
    $\exists x(\exists y(f))$ si può scrivere come $\exists x,y(f)$.
\end{itemize}
\par\smallskip
\textbf{Valore di verità} \\
L'inclusione di una data n-upla o meno da un'espressione è determinato dal valore di verità della sua formula.
Il valore di verità di una formula è così definito (sulle formule atomiche):
\begin{itemize}
  \item Una formula atomica $R(A_1:x_1,...,A_p:x_p)$ è vera sui valori $x_1,...,x_p$ che costituiscono una n-upla
    valida di $R$.
  \item Una formula atomica $x \ \theta \ y$ o $x \ \theta \ c$ (con c costante) è vera sui valori $x, y$ che soddisfano
    la condizione $\theta$.
  \item Il valore di verità degli operatori logici (congiunzione $\wedge$, disgiunzione $\vee$, negazione $\not$) è definito
    nel modo usuale.
  \item Una formula della forma $\exists x(f)$ è vera se esiste almeno un elemnto del dominio di $x$ che rende vera $f$.
    Analogamente, una formula della forma $\forall x(f)$ è vera se tutti gli elementi del dominio di $x$ rendono $f$.
\end{itemize}

\par\smallskip
\textbf{Interpretazione logica del calcolo dei domini} \\
Un'espressione del calcolo sui domini $\{A_1:x_1,...,A_k:x_k|f\}$ può essere sostanzialmente interpretata come
una formula logica del tipo:
$$ \{ x_1,...,x_k | f(x_1,...,x_l)\}$$
dove $x_1,...,x_k$ sono variabili o costanti, e $f(x_1,...,x_k)$ è un predicato (vero o falso) che determina
l'appartenenza o meno della n-upla di variabili al risultato dell'espressione.
\par\smallskip
\textbf{Vantaggi e svantaggi del calcolo dei domini} \\
Il vantaggio principale del calcolo dei domini è la dichiaritività: si \textit{dichiarano} le proprietà del
risultato desiderato, senza specificare i passaggi necessari a calcolarli. I difetti stanno invece nella
grande verbosità (non è permesso fare proiezioni su stadi intermedi, ergo siamo costretti a portarci dietro tutti
gli attributi nell'enunciato della formula). Inoltre, il calcolo dei domini è dipendente dal dominio
(ovvero non tutte le espressioni sintatticamente valide nel calcolo dei domini sono indipendenti dal dominio):
\par\smallskip
\textbf{Indipendenza dal dominio} \\
Si dice che un'espressione di un linguaggio d'interrogazione è indipendente dal dominio quando il suo
risultato non varia al variare del dominio rispetto alla quale l'espressione su cui viene valutata, salvo
ovviamente i valori presenti nell'istanza e nell'espressione. A questo punto un linguaggio si dice indipendente dal dominio
se tutte le sue espressioni sono tali. L'algebra relazionale è indipendente dal dominio: i risultati vengono 
costruiti sulla base di operazioni svolte sulle relazioni presenti nella base di dati senza mai fare riferimento
ai domini dei valori che compongono le n-uple delle istanze. In altre parole, tutti i valori compaiono
in istanze effettive di relazioni della base di dati o nella formulazione dell'espressione. Il calcolo dei 
domini, al contrario, è dipendente dal dominio. Si può ad esempio scrivere, senza violare nessuna regola sintattica:
$$ \{ A:x, B:y | R(A:x) \wedge y = y \} $$
nel risultato, il valore $x$ dell'attributo $A$ è legato alla relazione $R$, mentre il valore $y$ può essere qualsiasi
valore del dominio dell'attributo $B$, in quanto $\forall B:y | y = y$. Al variare del dominio di $B$, il risultato
dell'espressione cambia (l'espressione è dipendente dal dominio!). Fosse stato $B$ infinito, l'espressione avrebbe
restituito un insieme infinito. Un'altro esempio può essere:
$$ \{ A:x | \neg R(A:x)\} $$
Tale espressione ottiene esattamente il complemento dei valori $x$ sull'attributo $A$ contenuti nella relazione
$R$, ovvero tutti i valori del dominio di $A$ meno quelli in $R$. \\
Per ovviare ai problemi di dipendenza dal dominio presentati dal calcolo dei domini è stato introdotto il calcolo delle tuple.
\par\smallskip
\textbf{Calcolo delle tuple} \\
Il calcolo delle tuple risolve i problemi di verbosità del calcolo dei domini, utilizzando variabili per denotare
tuple anzichè singoli valori legati ad attributi. Ogni relazione coinvolta presenta quindi una variabile,
che rappresenta ogni possibile tupla di quella relazione. Per accedere ai singoli attributi di una tupla
occorre quindi associare una struttura ad ogni variabile. Le espressioni nel calcolo delle tuple hanno forma:
$$ \{ T | L | f \} $$
dove:
\begin{itemize}
  \item $T$ è una \textbf{target list} (obiettivi dell'interrogazione);
  \item $L$ è una \textbf{range list};
  \item $f$ è una formula.
\end{itemize}
\textbf{Target list} \\
Stabiliamo che:
\begin{itemize}
  \item $x$ è una variabile;
  \item $X$ è un insieme di attributi di una relazione;
  \item $Y$ e $Z$ sono sottoinsiemi di attributi di $X$ di pari lunghezza.
\end{itemize}
A questo punto, possiamo dire che $T$ è una lista di elementi del tipo:
\begin{itemize}
  \item $Y:x.Z$, Denotiamo con $Y$ solo gli attributi $Z$ della variabile (tupla) $x$, che assumerà valori
    definiti in $L$ (\textit{range list}).
  \item $x.Z \equiv Z : x.Z$ vogliamo solo gli attributi $Z$ della variabile $x$, che assumerà valori
    definiti in $L$, ma non li ridenominiamo.
  \item $x.* \equiv X : x.X$ Prendiamo tutti gli attributi della variabile $x$, che assumerà valori definiti in $L$.
\end{itemize}
\textbf{Range list} \\
La \textit{range list} $L$ è una lista che contiene, senza ripetizioni, tutte le variabili della \textit{target list},
con associata la relazione da cui viene prelevata ogni variabile:
$$ L \equiv x_1(R_1),...,x_k(R_k) $$
dove $x_i$ è una variabile e $R_i$ è una relazione. $L$ è una dichiarazione di range, ovvero specifica l'insieme
dei valori che possono essere assegnati alle variabili. Non occorrono, come occorrevano nel calcolo dei domini,
condizioni atomiche che vincolano una tupla ad appartenere ad una relazione ($A_1 : x_1 ,..., A_n : x_n$).
\par\smallskip
\textbf{Formule atomiche del calcolo delle tuple} \\
Possiamo definire, come avevamo fatto per il calcolo dei domini, 2 formule atomiche (manca la condizione di vincolo
d'appartenenza alla relazione):
\begin{itemize}
  \item \textbf{Operatori fra variabili}
    $$ x_i.A_1  \ OP \  x_j.A_j $$
    dove $x$ e $y$ sono variabili, $A_1$ e $A_j$ attributi di tali variabili, e $OP$ è un operatore di confronto (uguale, non uguale, maggiore, minore e varianti strette).
  \item \textbf{Operatori fra variabili e costanti}
    $$ x_i.A_i \ OP \ c, \quad c \ OP \ x_i.A_i $$
    dove $c$ è una costante nel dominio dell'attributo $A_i$ di $x_i$.
\end{itemize}
Possiamo quindi definire alcune proprietà delle formule, quasi del tutto analoghe a quelle definite per il calcolo
dei domini:
\begin{itemize}
  \item Le formule atomiche sono formule.
  \item Se $f$ è una formula, allora anche $\neg f$ è una formula.
  \item Se $f_1$ e $f_2$ sono formule, allora anche $f_1 \wedge f_2$ e $f_1 \vee f_2$ sono formule.
  \item Se $f$ è una formula e $x$ una variabile che indica una n-upla sulla relazione $R$, allora anche 
    $\exists xR(f)$ e $\forall xR(f)$ sono formule,
    dove $\exists$ and $\forall$ sono qualificatori (rispettivamente esistenziale e universale). Notiamo
    che anche i quantificatori contengono adesso delle dichiarazioni di range(\textit{range list}). La notazione
    $\exists (R)(f)$ significa "esiste nella relazione $R$ una n-upla $x$ che soddisfa la formula $f$".
\end{itemize}
\textbf{Vantaggi e svantaggi del calcolo delle tuple}
Il vantaggio principale del calcolo delle tuple è la minore verbosità rispetto al calcolo dei domini. Svantaggiosa
è invece l'effettiva impossibilità di esprimere alcune interrogazioni, in particolare l'unione: 
$$ R_1(AB) \cup R_2(AB) $$
Questo perchè ogni variabile nel risultato ha un solo range, mentre l'unione comporta la derivazione di n-uple
da più relazioni distinte. Per questo motivo linguaggi quali l'SQL, definiscono un operatore esplicito di unione (l'UNION).
Intersezione e differenza sono invece esprimibili.
\par\smallskip
\textbf{Calcolo e algebra} \\
Calcolo e algebra sono sostanzialmente equivalenti, ovvero ogni espressione dell'algebra (indipendente dal domino!) ha
un equivalente nel calcolo e viceversa. Esistono però alcune interrogazioni non esprimibili:
\begin{itemize}
  \item \textbf{Calcolo dei valori derivati:} abbiamo la possbilita di estrarre valori, mai di generarne di nuovi.
  \item \textbf{Ricorsività:} le interrogazioni non hanno la possibilità di esprimere ricorsività, come avevamo visto 
    nell'esempio della chiusura transitiva (riguardante riferimenti potenzialmente infiniti fra relazioni).
\end{itemize}
\end{document}


\documentclass[a4paper,12pt]{article}

\usepackage[french,italian]{babel}
\usepackage[T1]{fontenc}
\usepackage[utf8]{inputenc}
\frenchspacing 
\title{Appunti Basi di Dati}
\author{Luca Seggiani}
\date{10 Aprile 2024}

\begin{document}
\maketitle
\section{Modellazione e progettazione concettuale}
La definizione di schemi adeguati per le basi di dati richiede metodologie precise per la modellazione accurata
della realtà che ci interessa. Bisogna tenere a mente che:
\begin{itemize}
  \item Non conviene concetrarsi subito sui dettagli;
  \item Conviene invece stabilire subito interdipendenze fra relazioni;
  \item Il modello relazionale sarà rigido una volta ultimato.
\end{itemize}
In generale si può dire che esiste sempre (e spesso è unico) uno schema che modella accuratamente e nel modo
più semplice possibile una certa realtà di interesse. La progettazione di basi di dati è solo una fase delloo
sviluppo di un sistema informativo (vedere a riguardo le prime lezioni di teoria). Bisogna quindi tenere conto 
di:
\begin{itemize}
  \item \textbf{Ciclo di vita} \textit{(lifecycle)} del sistema informativo, ovvero l'insieme
    delle attivita svolte da analisti, progettisti e utenti nello sviluppo e nell'uso del sistema informativo.
    Questa attività è iterativa, e quindi ciclica.
\end{itemize}
I passi ciclo di vita dovranno essere ben definiti attraverso linguaggi e modelli prestabiliti. Per le basi di
dati, in particolare, conviene adottare modelli di facile utilizzo, che consentano la decomposizione delle attività
in fasi (e/o livelli) distinti, e di utilizzare strategie e criteri di scelta nei vari passaggi.
\par\smallskip
\textbf{Modello a cascata} \\
Il modello a cascata (\textit{waterfall model}) le fasi sono sequenzialmente ordinate, etichettate, e non ripetibili. In ordine,
esse sono le seguenti:
\begin{enumerate}
  \item \textbf{Studio di fattibilità}:
    definizione di costi e priorità della produzione;
  \item \textbf{Raccolta e analisi dei requisiti}:
    studio delle proprietà del sistema che andranno implementate;
  \item \textbf{Progettazione}:
    progettazione di strutture dati e funzioni;
  \item \textbf{Realizzazione}:
    implementazione effettiva del codice;
  \item \textbf{Validazione e collaudo}:
    sperimentazione del prodotto;
  \item \textbf{Funzionamento}:
    il sistema diventa opertaivo in produzione (\textit{shipping}).
\end{enumerate}

Vediamo alcune fasi nel dettaglio.
\section{Raccolta e analisi dei requisti}
Questa fase può essere, a sua volta, divisa in due sotto-fasi:
\begin{itemize}
  \item \textbf{Acquisizione dei requisiti}: il reperimento dei requisiti è un'attività non standardizzata. Esistono
    più modalità:
    \begin{itemize}
      \item Direttamente dagli utenti:
      \begin{itemize}
        \item Interviste, focus group, recensioni, ecc...
        \item Documentazioni apposite;
      \end{itemize}
    \item Attraverso documentazioni preesistenti:
      \begin{itemize}
        \item Normative (legislazioni, regolamenti di settore);
        \item Regolamenti interni, procedure aziendali;
        \item Realizzazioni preesistenti.
      \end{itemize}
    \end{itemize}
    Esistono linguaggi per definire requisiti (\textit{UML}).
  \item \textbf{Analisi dei requisiti}: si analizzano i requisiti raccolti, spesso nella prospettiva di successive
    acquisizioni.
\end{itemize}

\par\smallskip
\textbf{Interazione con gli utenti} \\
Le problematiche dell'interazione con gli utenti possono essere:
\begin{itemize}
  \item Utenti diversi danno risposte diverse;
  \item Utenti a livello più alto hanno spesso una visione più ampia ma meno dettagliata;
  \item Spesso l'acquisizione di requisiti avviene per raffinazione.
\end{itemize}
Conviene quindi:
\begin{itemize}
  \item Effettuare spesso verifiche di comprensione e coerenza;
  \item Verificare anche attraverso esempi (sopratutto nei casi limite);
  \item Richiedere definizioni e classificazioni chiare e specifiche;
  \item Separare gli aspetti essenziali da quelli marginali (\textit{ranking}).
\end{itemize}

\par\smallskip
\textbf{Interazione con gli utenti tramite documentazione} \\
E' opportuno seguire alcune linee guide generali, assicurando:
\begin{itemize}
  \item Standardizzazione della struttura delle frasi;
  \item Separazione delle frasi riguardanti dati da quelle riguardanti funzioni;
  \item Organizzazione di termini e concetti:
    \begin{itemize}
      \item Unificazione di termini (eliminazione di sinonimi);
      \item Esplicitazione del riferimento fra termini;
    \end{itemize}
  \item Organizzazione delle frasi per concetti.
\end{itemize}
\par\smallskip
\section{Progettazione}
La progettazione è una fase del ciclo di vita. Per un sistema software la progettazione si divide effettivamente in:
\begin{itemize}
  \item Progettazione dei dati;
  \item Progettazione delle applicazioni.
\end{itemize}
\textbf{Progettazione per astrazione} \\
Come in tutte le applicazioni informatiche, abbiamo visto che è necessario progettare per livelli successivi di astrazione:
\begin{itemize}
  \item \textbf{Livello concettuale}: esprime i requisiti di un sistema in una descrizione adatta all'analisi
    da punti di vista esterni
  \item \textbf{Livello logico}: evidenzia l'organizzazione dei dati dal punto di vista del loro contenuto informativo,
    descrivendo la struttura dei record e le loro interdipendenze.
  \item \textbf{Livello fisico}: si concentra sulla base di dati vista come un insieme di blocchi fisici sul disco,
    e riguarda quindi l'allocazione dei dati e le modalità di memorizzazione.
\end{itemize}
Con riferimento a quanto detto sugli schemi logici avremo quindi che la progettazione si divide in:
\begin{itemize}
  \item \textbf{Progettazione concettuale}, parte dai requisiti individuati della base e produce uno schema concettuale;
  \item \textbf{Progettazione logica}, parte dallo schema concettuale e produce uno schema logico;
  \item \textbf{Progettazione fisica}, parta dallo schema logico e produce lo schema fisico finale.
\end{itemize}
Come lo era stato il modello a cascata, la progettazione per astrazione è composta da fasi sequenzialmente ordinate che
vanno eseguite strettamente in ordine.
\par\smallskip
\textbf{Modello dei dati} \\
Il modello dei dati è l'insieme dei costrutti utilizzati per organizzare i dati di interesse e definirne
la dinamica. Componente fondamentale del modello sono i meccanismi di strutturazione (costruttori di tipi).
Come per i linguaggi di programmazione comuni, esistono meccanismi che permettono di definire nuovi tipi. Ogni 
modello dei dati prevede alcuni costruttori: ad esempio il modello relazionale prevede un costruttore relazionale che
permette di definire insiemi di record omogenei.
Per riassumere in breve, in ogni base di dati si ha:
\begin{itemize}
  \item Lo \textbf{schema}, invariante nel tempo, che ne descrive la struttura. Si notino inoltre le
    \textbf{intestazioni} delle tabelle (previste nel modello relazionale).
  \item L'\textbf{istanza}, i valori effettivi assunti dalla base di dati in un dato momento. Nel modello
    relazionale rappresenta il \textbf{corpo} di ciascuna tabella.
\end{itemize}
\par\smallskip
\textbf{Modello concettuale Entity-Relationship} \\
Il modello concettuale che utilizzeremo sarà quello \textbf{Entity-Relationship (ER)}. Si noti che la parola \textit{relationship},
sebbene abbia lo stesso significato in lingua inglese, non si riferisce al concetto matematico di relazione che sta alla base del 
modello relazionale. Per questo motivo useremo sempre il termine inglese per descrivere le relazioni del modello
entity-relationship, in modo da distinguerle dalle relazioni del modello relazionale. Il modello entity-relationship
viene sviluppato da P.P. Chen nel 1976, ed è oggi una delle metodologie più affermate nel campo della progettazione dei sistemi informatici,
anche se in un'accezione leggermente diversa da quella in cui era stato concepito inizialmmente. \\ I suoi costrutti base sono:
\begin{itemize}
  \item Entità
  \item Relationship
  \item Attributi
\end{itemize}
\par\smallskip
\textbf{Entità} \\
Un'entità è una classe di oggetti dell'applicazione d'interesse con proprietà comuni e esistenza autonoma.
Un'occorrenza (o istanza) di entità è un elemento della classe (un'elemento, non i dati ad esso legati!). Ogni
entità deve avere un nome che la identifica univocamente nello schema. Graficamente è rappresentato da una scatola.
\par\smallskip
\textbf{Relationship} \\
Una relationship è un legame logico fra due o più entita, rilevante nell'applicazione d'interesse. Può essere chiamata
anche relazione (vedi sopra), correlazione o associazione. Ogni relationship, come per le entità, ha un nome
che la identifica univocamente nello schema. Graficamente è rappresentata da una losanga. \\
Vediamo allora di definire il concetto di occorrenza di relationship, più complesso di quello di occorrenza di entità:
\begin{itemize}
  \item Un'occorrenza di \textbf{relationship binaria} è una coppia di occorrenza di entità, una per ciascuna entità coinvolta.
  \item Una occorrenza di una \textbf{relationship n-aria} è una n-upla di occorenze di entità, una per ciascuna delle $n$ entità coinvolte.
\end{itemize}
Nell'ambito di una relationship non ci possono essere occorrenze (né coppie né n-uple) ripetute.
\par\smallskip
\textbf{Attributo} \\
L'attributo è una proprietà elementare di un'entità o di una relationship, che ci interessa ai fini dell'applicazione
d'interesse. Associa a ogni occorrenza di entità o relationship un valore appartenente ad un dominio, il cosiddetto
dominio dell'attributo.
\par\smallskip
\textbf{Attributo composito} \\
Gli attributi commpositi raggruppano attributi di una medesima entità o relatioship che presentano affinità nel loro
significato o uso (e.g. giorno, mese, anno si compone in data, via, numero civico, CAP in indirizzo, ecc...).
\end{document}

\documentclass[a4paper,12pt]{article}

\usepackage[french,italian]{babel}
\usepackage[T1]{fontenc}
\usepackage[utf8]{inputenc}
\frenchspacing 
\title{Appunti Basi di Dati}
\author{Luca Seggiani}
\date{17 Aprile 2024}

\usepackage{hyperref}

\usepackage{tikz}
\usetikzlibrary{shapes.geometric, shapes.arrows}

\tikzstyle{entity} = [rectangle, text centered, minimum width=3cm, minimum height=1.5cm , draw=black, fill=white]
\tikzstyle{relationship} = [diamond, text centered, draw=black, fill=white]
\tikzstyle{arrow} = [->, >=stealth]
\tikzstyle{connector} = [-, >=stealth]
\tikzstyle{filledarrow} = [single arrow, draw=black, fill=black, rotate=90, minimum height=0.7cm]
\tikzstyle{emptyarrow} = [single arrow, draw=black, fill=white, rotate=90, minimum height=0.7cm]

\begin{document}
\maketitle
\section{Concetti inesprimibili nel modello ER}
Alcuni concetti sono inesprimibili attraverso il modello ER tradizionale, e bisognano quindi di costrutti
particolari:
\begin{itemize}
  \item \textbf{Cardinalità}
    \begin{itemize}
      \item di relationship;
      \item di attributo;
    \end{itemize}
  \item \textbf{Identificatore}
    \begin{itemize}
      \item interno;
      \item esterno;
    \end{itemize}
  \item \textbf{Generalizzazione}
\end{itemize}
\par\smallskip
\textbf{Cardinalità di relationship} \\
Una cardinalità di relationship è rappresentata da una coppia di valori associati ad ogni entità che partecipa alla
relationship. Questi specificano il numero minimo e massimo di occorrenze della relationship a cui ciascuna
occorrenza di entità può partecipare.
I simboli usati saranno:
\begin{itemize}
  \item Per la minima:
  \begin{itemize}
    \item 0 $\rightarrow$ partecipazione opzionale
    \item 1 $\rightarrow$ partecipazione obbligatoria
  \end{itemize}
  \item Per la massima:
    \begin{itemize}
      \item 1 $\rightarrow$ partecipazione opzionale
      \item N $\rightarrow$ non pone alcun limite
    \end{itemize}
\end{itemize}
Con riferimento alle cardinalità massime, abbiamo relationship:
\begin{itemize}
  \item \textbf{Uno a uno}, ad esempio:\\
    \begin{tikzpicture}[node distance=2cm]
      \node (studente) [entity] {Professore};
      \node (esame) [entity, right of=studente, xshift=8cm] {Cattedra};
      \node (presenza) [relationship, right of= studente, xshift=3cm] {Titolarità};
      \draw [arrow] (studente) -- node[anchor=south] {(0, 1)} (presenza);
      \draw [arrow] (esame) -- node[anchor=south] {(0, 1)} (presenza);
    \end{tikzpicture} \par\smallskip
    Un professore può essere tale senza essere titolare di alcuna cattedra, e una cattedra può restare
    vuota. \par\smallskip
    \begin{tikzpicture}[node distance=2cm]
      \node (studente) [entity] {Professore di ruolo};
      \node (esame) [entity, right of=studente, xshift=8cm] {Cattedra};
      \node (presenza) [relationship, right of= studente, xshift=3cm] {Titolarità};
      \draw [arrow] (studente) -- node[anchor=south] {(1, 1)} (presenza);
      \draw [arrow] (esame) -- node[anchor=south] {(0, 1)} (presenza);
    \end{tikzpicture} \par\smallskip
    Un professore di ruolo dovrà ovviamente essere titolare di almeno una cattedra, ma questo non significa
    comunque che tutte le cattedre siano occupate. \par\smallskip
    \begin{tikzpicture}[node distance=2cm]
      \node (studente) [entity] {Professore di ruolo};
      \node (esame) [entity, right of=studente, xshift=8cm] {Cattedra occupata};
      \node (presenza) [relationship, right of= studente, xshift=3cm] {Titolarità};
      \draw [arrow] (studente) -- node[anchor=south] {(1, 1)} (presenza);
      \draw [arrow] (esame) -- node[anchor=south] {(1, 1)} (presenza);
    \end{tikzpicture} \par\smallskip
    E' una forzatura? Sì. Serve a spiegare il concetto? Sempre sì. 
  \item \textbf{Uno a molti}, ad esempio:\\
    \begin{tikzpicture}[node distance=2cm]
      \node (studente) [entity] {Persona};
      \node (esame) [entity, right of=studente, xshift=8cm] {Azienda};
      \node (presenza) [relationship, right of= studente, xshift=3cm] {Impiego};
      \draw [arrow] (studente) -- node[anchor=south] {(0, 1)} (presenza);
      \draw [arrow] (esame) -- node[anchor=south] {(0, N)} (presenza);
    \end{tikzpicture} \par\smallskip
    Una persona può essere o non essere assunta, ma al massimo da una azienda. Al contrario, un'azienda avrà
    probabilmente più di un'impiegato. \par\smallskip
    \begin{tikzpicture}[node distance=2cm]
      \node (studente) [entity] {Ufficiale};
      \node (esame) [entity, right of=studente, xshift=8cm] {Nave};
      \node (presenza) [relationship, right of= studente, xshift=3cm] {Comando};
      \draw [arrow] (studente) -- node[anchor=south] {(1, N)} (presenza);
      \draw [arrow] (esame) -- node[anchor=south] {(0, 1)} (presenza);
    \end{tikzpicture} \par\smallskip
    Un ufficiale per considerarsi tale deve comandare almeno una nave, ma non è detto che tutte le navi abbiano un
    comandante. Alcune sono navi fantasma come nei Pirati dei caraibi. \par\smallskip
    \begin{tikzpicture}[node distance=2cm]
      \node (studente) [entity] {Comune};
      \node (esame) [entity, right of=studente, xshift=8cm] {Provincia};
      \node (presenza) [relationship, right of= studente, xshift=3cm] {Ubicazione};
      \draw [arrow] (studente) -- node[anchor=south] {(1, 1)} (presenza);
      \draw [arrow] (esame) -- node[anchor=south] {(1, N)} (presenza);
    \end{tikzpicture} \par\smallskip
    Ogni comune è ubicato in una e una sola provincia, mentre ogni provincia è l'ubicazione di più comnuni.
  \item \textbf{Molti a molti}, ad esempio:\\
    \begin{tikzpicture}[node distance=2cm]
      \node (studente) [entity] {Studente};
      \node (esame) [entity, right of=studente, xshift=8cm] {Esame};
      \node (presenza) [relationship, right of= studente, xshift=3cm] {Presenza};
      \draw [arrow] (studente) -- node[anchor=south] {(0, N)} (presenza);
      \draw [arrow] (esame) -- node[anchor=south] {(0, N)} (presenza);
    \end{tikzpicture} \par\smallskip
    Non è detto che ogni studente abbia sostenuto un'esame, come non è detto che ogni esame
    sia stato sostenuto da almeno uno studente. \par\smallskip
    \begin{tikzpicture}[node distance=2cm]
      \node (studente) [entity] {Montagna};
      \node (esame) [entity, right of=studente, xshift=8cm] {Alpinista};
      \node (presenza) [relationship, right of= studente, xshift=3cm] {Scalata};
      \draw [arrow] (studente) -- node[anchor=south] {(0, N)} (presenza);
      \draw [arrow] (esame) -- node[anchor=south] {(1, N)} (presenza);
    \end{tikzpicture} \par\smallskip
    Per potersi considerare alpinisti occorre aver scalato almeno una montagna, ma non è detto che tutte le
    montagne siano state scalate. \par\smallskip
    \begin{tikzpicture}[node distance=2cm]
      \node (studente) [entity] {Autore};
      \node (esame) [entity, right of=studente, xshift=8cm] {Libro};
      \node (presenza) [relationship, right of= studente, xshift=3cm] {Scrittura};
      \draw [arrow] (studente) -- node[anchor=south] {(1, N)} (presenza);
      \draw [arrow] (esame) -- node[anchor=south] {(1, N)} (presenza);
    \end{tikzpicture} \par\smallskip
    Ogni autore ha scritto almeno un libro per considerarsi tale, e ogni libro deve essere stato scritto da almeno
    un autore. Questo però non pregiudica che un autore non possa scrivere più libri o un libro non possa essere scritto
    scritto da più autori.
\end{itemize}
\par\smallskip
\textbf{Identificatore di entità} \\
L'identificatore di entità è uno strumento per l'identificazione univoca delle occorrenze di un entità.
\begin{itemize}
  \item Gli attributi dell'entità possono formare l'\textbf{identificatore interno} (o chiave). Per capirsi, un'identificatore interno è una chiave primaria o 
    comunque una chiave candidata. \par\smallskip
    \begin{tikzpicture}
    \node (entita) [entity] {Automobile};
    \fill (2,1) circle (0.1cm);
    \node[right] at (2.2,1) {Targa};
    \draw [connector] (1.9,1) -- (entita);

    \draw (2,0) circle (0.1cm);
    \node[right] at (2.2,0) {Colore};
    \draw [connector] (1.9,0) -- (entita);

    \draw (2,-1) circle (0.1cm);
    \node[right] at (2.2,-1) {Modello};
    \draw [connector] (1.9,-1) -- (entita);
  \end{tikzpicture} \par\smallskip
  Come vediamo dalla figura, targa è la chiave primaria, ovvero l'identificatore. Nel caso un'insieme di attributi sia una chiave,
  adottiamo la rappresentazione seguente: \par\smallskip
    \begin{tikzpicture}
    \node (entita) [entity] {Studente};
    \draw (2,1) circle (0.1cm);
    \node[right] at (2.2,1) {Nome};
    \draw [connector] (1.9,1) -- (entita);

    \draw (2,0) circle (0.1cm);
    \node[right] at (2.2,0) {Cognome};
    \draw [connector] (1.9,0) -- (entita);

    \draw (2,-1) circle (0.1cm);
    \node[right] at (2.2,-1) {Età};
    \draw [connector] (1.9,-1) -- (entita);

    \fill (1.7,1.1) circle (0.1cm);
    \draw[connector] (1.7, 1.1) -- (1.7, -1.1);
  \end{tikzpicture} \par\smallskip
  \item Gli attributi dell'entità e l'identificatore interno di entità esterne raggiunte attraverso relationship formano un'\textbf{identificatore esterno}.
  un'identificazione esterna può essere possibile solo nel caso esista una relationship in cui l'entità da identificare abbia cardinalità $(1,1)$. Per
  capirsi, un identificatore è una chiave contenente una chiave esterna.
    \begin{tikzpicture}[node distance=2cm]
      \node (studente) [entity] {Studente};
      \node (esame) [entity, right of=studente, xshift=8cm] {Facoltà};
      \node (presenza) [relationship, right of= studente, xshift=3cm] {Iscrizione};
      \draw [arrow] (studente) -- node[anchor=south] {(1, 1)} (presenza);
      \draw [arrow] (esame) -- node[anchor=south] {(0, N)} (presenza);

      \draw (2,1) circle (0.1cm);
      \node[right] at (2.2,1) {Nome};
      \draw [connector] (1.9,1) -- (entita);
  
      \fill (1.7,1.1) circle (0.1cm);
      \draw[connector] (1.7, 1.1) -- (1.7, -1.1);
    \end{tikzpicture} \par\smallskip
    Dalla figura si capisce che il nome dello studente e la sua iscrizione a una certa facolà formano il suo identificatore esterno.

\end{itemize}
Ogni entità deve possedere almeno un identificatore, ma può averne in generale più di uno.

\par\smallskip
\textbf{Generalizzazione} \\
La generalizzazione mette in relazione una o più entità $E_1, E_2, ... , E_n$ con una singola entità $E$ che le 
comprende come casi particolari. Si dice che $E$ è una generalizzazione (oppure entità genitrice, madre) di
$E_1, E_2, ... , E_n$ (dette specializzazioni, sottotipi o entità figlie). \par\smallskip
\begin{center}
\begin{tikzpicture}
  \node (dipendente) [entity] {Dipendente};
  \node (funzionario) [entity, right of=dipendente, yshift=-3cm, xshift=-3cm] {Funzionario};
  \node (amministratore) [entity, right of=dipendente, yshift=-3cm, xshift=1cm] {Amministratore};
  \node (filled) [filledarrow, right of=dipendente, xshift=-2.2cm] {};
  \draw [connector] (funzionario) -- (filled);
  \draw [connector] (amministratore) -- (filled);
\end{tikzpicture}
\end{center}
Le caratteristiche delle generalizzazioni sono:
\begin{itemize}
  \item \textbf{Ereditarietà}: tutte le proprietà dell'entità genitore vengono ereditate dalle figlie e non rappresentate
    esplicitamente.
  \item \textbf{Generalizzazione totale}: se ogni occorenza dell'entità genitore è rimpiazzata da almeno un occorrenza delle entità
    figlie, altrimenti si parla di \textbf{generalizzazione parziale}. Di norma le generalizzazioni parziali si indicano
    con una freccia vuota, mentre quelle totali con una freccia piena.
\end{itemize}
  \noindent 
  \begin{tikzpicture}
  \node (dipendente) [entity] {Dipendente};
  \node (funzionario) [entity, right of=dipendente, yshift=-3cm, xshift=-3cm] {Funzionario};
  \node (amministratore) [entity, right of=dipendente, yshift=-3cm, xshift=1cm] {Amministratore};
  \node (filled) [filledarrow, right of=dipendente, xshift=-2.2cm] {};
  \draw [connector] (funzionario) -- (filled);
  \draw [connector] (amministratore) -- (filled);
\end{tikzpicture}
\hspace{0.3cm}
\begin{tikzpicture}
  \node (dipendente) [entity] {Persona};
  \node (funzionario) [entity, right of=dipendente, yshift=-3cm, xshift=-3cm] {Dipendente};
  \node (amministratore) [entity, right of=dipendente, yshift=-3cm, xshift=1cm] {Studente};
  \node (filled) [emptyarrow, right of=dipendente, xshift=-2.2cm] {};
  \draw [connector] (funzionario) -- (filled);
  \draw [connector] (amministratore) -- (filled);
\end{tikzpicture}
\begin{itemize}
  \item \textbf{Generalizzazione esclusiva}: se ogni occorrenza dell'entità genitrice è occorrenza di al massimo una delle
    entità figlie, altrimenti si parla di \textbf{generalizzazione sovrapposta}.
  \item Possono esistere \textbf{gerarchie a più livelli} e multiple generalizzazioni allo stesso livello.
  \item Un'entità può essere inclusa in più gerarchie, sia come genitore che come figlia, se non entrambe.
  \item Se una generalizzazione ha solo un'entita figlia si dice \textbf{sottoinsieme}.
  \item Il genitore di una generalizzazione totale può non avere identificatore (\textbf{anonimato}) purché
    siano identificate le figlie.
\end{itemize}
\section{Progettazione concettuale}
Vediamo adesso la progettazione concettuale nel dettaglio. Il modello ER sarà lo strumento fondamentale nel corso
di questa fase: dovremo innanzitutto decidere, per una qualsiasi specifica fornitaci, quale sia il costrutto ER
più adeguato da utilizzare. Per fare questo ci basiamo sulle definizioni dei costrutti del modello ER:
\begin{itemize}
  \item \textbf{Entità}, se un oggetto ha proprietà significative e descrive oggetti con esistenza autonoma;
  \item \textbf{Attributo}, se un'oggetto è semplice e non ha proprietà specifiche;
  \item \textbf{Relationship}, se un concetto correla più oggetti;
  \item \textbf{Generalizzazione}, se un concetto è caso particolare di un altro.
\end{itemize}
\par\smallskip
\textbf{Design pattern} \\
I design pattern sono soluzioni progettuali a problemi comuni. Sono largamente usati nell'ingegneria del software,
e ne esistono alcuni comuni nella progettazione delle basi di dati.
\begin{itemize}
  \item \textbf{Reificazione di attributo di entità} \\
La reificazione di un attributo di un'entità è la promozione di tale attributo ad un'entità a sé stante, con ovvia creazione
di una relationship che esprima il rapporto fra l'entità di partenza e l'attributo che abbiamo trasformato in entità. Supponiamo
di avere il modello ER:
\begin{tikzpicture}
\node (entita) [entity] {Impiegato};
    \fill (2,1) circle (0.1cm);
    \node[right] at (2.2,1) {Codice};
    \draw [connector] (1.9,1) -- (entita);

    \draw (2,0) circle (0.1cm);
    \node[right] at (2.2,0) {Nome};
    \draw [connector] (1.9,0) -- (entita);

    \draw (2,-1) circle (0.1cm);
    \node[right] at (2.2,-1) {Azienda};
    \draw [connector] (1.9,-1) -- (entita);
\end{tikzpicture} \par\smallskip
L'attributo azienda potrebbe essere trasformato in un entità a sé, per poter includere maggiori informazioni rispetto all'azienda stessa:
\par\smallskip
\begin{tikzpicture}[node distance=2cm]
      \node (studente) [entity] {Impiegato};

    \fill (2,1) circle (0.1cm);
    \node[right] at (2.2,1) {Codice};
    \draw [connector] (1.9,1) -- (entita);

    \draw (2,-1) circle (0.1cm);
    \node[right] at (2.2,-1) {Nome};
    \draw [connector] (1.9,-1) -- (entita);

      \node (esame) [entity, right of=studente, xshift=8cm] {Azienda};
      \node (presenza) [relationship, right of= studente, xshift=3cm] {Impiego};
      \draw [arrow] (studente) -- node[anchor=south] {(1, 1)} (presenza);
      \draw [arrow] (esame) -- node[anchor=south] {(1, N)} (presenza);
    \end{tikzpicture} \par\smallskip
\item \textbf{Reificazione di relationship in entità} \\ 
  La reificazione di una relationship in entità è la promozione di una relationship in un'entità a sé stante, che sarà
  opportunamente collegata alle entità che metteva prima in relazione fra di loro attraverso altre relationship ausiliarie. Nello specifico,
  vediamo come usare la reificazione di relationship in entità per reificare relationship binarie e ternarie:
  \begin{itemize}
    \item \textbf{Reificazione di relationship binarie} \\
     Diciamo di avere un modello abbastanza semplice, contenente due entità con identificatore interno legate da una relationship:
      Magari la relationship potrebbe avere bisogno di più informazioni specifiche: potremo allora usare più approcci alternativi. 
      Il primo sarà di creare un entità corrispondente alla relationship, e di rendere poi gli attributi di quell'entità e le relazioni
      con le entità di partenza identificatore esterno. Un'altro approccio sarà quello di assegnare semplicemente alla nuova entità un identificatore
      univoco.
    \item \textbf{Reificazione di relationship ternarie} \\
      Un problema più sostanziale può essere rappresentato da una relationship ternaria. In questo caso si può adottare un approccio analogo
      al precedente, trasformando la relationship in un entità contenente tutte le informazioni necessario e legata alle entità di partenza con
      3 distinte relationship, che potranno anche formare l'identificatore esterno della nuova entità.
    \item \textbf{Reificazione di attributo di relationship}\\
      Nel modello ER, nessuno ci impedisce di assegnare attributi alle relationship. Questo però andrà molto probabilmente
      tradotto in un entità a sé stante, che sarà legata attraverso le metodologie che abbiamo appena visto alle entità che legava in partenza.
  \end{itemize}
\item \textbf{"Parte di"} \\
    Il pattern parte di permette di definire concetti che sono parte di altri concetti, come ad esempio un cinema con le sue sale.
    Viene rappresentato da una relazione "composizione", uno a molti, dove l'oggetto che è parte dell'altro ha identificatore esterno
    nel primo.
  \item \textbf{"Istanza di"} \\
    Il pattern istanza di permette di definire concetti che sono istanze di altri concetti generali, come ad esempio un torneo e le sue edizioni.
    In questo caso si usa sempre una relazione, "occorrenza", uno a molti, dove l'oggetto occorrenza ha identificatore
    esterno nell'oggetto generale.
  \item \textbf{Caso particolare di entità} \\
    Il pattern caso particolare di entità permette definire casi particolari di determinate entità, attraverso la generalizzazione totale di un entità
    specializzata su un'entita generale. L'entità specializzata sarà il caso particolare.
\end{itemize}
Grafici di modelli ER per i pattern riportati si possono trovare al link:
\url{https://github.com/Guray00/IngegneriaInformatica/blob/master/PRIMO%20ANNO/II%20SEMESTRE/Basi%20di%20dati/Diapositive/Slides%20Tonellotto/A.A.%2022-23/%5B06%5D%20Progettazione%20Concettuale%20e%20Logica.pdf}
\par\smallskip
\textbf{Documentazione associata agli schemi concettuali} \\
Il modello ER non basta mai da solo a descrivere l'interezza della realtà che vogliamo modellare. Esistono infatti, nonostante tutte
le aggiunte che abbiamo apportato al modello ER, vincoli inesprimibili, che dovranno essere espressi dalla documentazione di supporto.
Serviranno quindi:
\begin{itemize}
  \item Un \textbf{dizionario dei dati}, ovvero una tabella che contenga informazioni riguardo alle entità e alle relationship
    che dovranno essere modellizzate dal nostro database.
  \item \textbf{Regole aziendali}: ovvero informazioni riguardo ai vincoli di integrità necessarie, nonché possibili derivazioni
    specifiche ai tipi di dati che staremo trattando (formule particolari per il calcolo di indici, medie...).
\end{itemize}
\end{document}

\documentclass[a4paper,12pt]{article}

\usepackage[french,italian]{babel}
\usepackage[T1]{fontenc}
\usepackage[utf8]{inputenc}
\frenchspacing 
\title{Appunti Basi di Dati}
\author{Luca Seggiani}
\date{24 Aprile 2024}

\begin{document}
\maketitle
\section{Strategie di progetto}
Possiamo definire alcuni tipi di strategie di progetto:
\begin{itemize}
  \item \textbf{Top-down} \\
    Si parte da uno schema iniziale e generale (top) che viene poi definito nei dettagli fino allo schema definitivo (down), muovendosi
    così, figurativamente, dall'alto verso il basso (zip!). La raffinazione dettagli avviene attraverso l'applicazione di primitive:
    \par\smallskip
    \textbf{Primitive di raffinamento}
    \begin{itemize}
      \item Da entità a associazione tra entità;
      \item Da entità a generalizzazione;
      \item Da associazione a insiemi di associazioni;
      \item Da associazione a entità con associazioni;
      \item Introduzione di attributi su entità e associazioni.
    \end{itemize}
    Dove associazione è sinonimo di relationship.
  \item \textbf{Bottom-up} \\
    Si parte dalle specifiche delle componenti minime dello schema: queste specifiche vengono poi integrate fra di loro a formare lo schema completo.
    Come prima, figurativamente, dal basso verso l'alto. Questa integrazione avviene attraverso le primitive di trasformazione (o generalizzazione).
    \par\smallskip
    \textbf{Primitive di trasformazione}
    \begin{itemize}
      \item Generazione di entità;
      \item Generazione di associazione;
      \item Generazione di generalizzazione.
    \end{itemize}
    Come sopra.
  \item \textbf{Inside-out} \\
    La progettazione inside-out è sostanzialmente una generalizzazione della bottom-down, dove si parte da strutture atomiche di complessità minore
    fino a schemi più complessi e sempre più vicini al risultato finale.
\end{itemize}
Nella pratica, una strategia non è mai puramente top-down o bottom-up, ma è più una \textbf{strategia mista}, che può essere sintetizzata come:
\begin{itemize}
  \item Individuazione dei concetti principali e realizzazione di uno schema scheletro (un semplice schema concettuale che racchiude
    i dettagli più importanti della realtà da modellare);
  \item Decomposizione dello schema scheletro;
  \item Raffinazione, espansione e integrazione delle componenti dello schema.
\end{itemize}
Possiamo definire questa metodologia nei dettagli.
\begin{itemize}
  \item \textbf{Analisi dei requisiti}
    \begin{itemize}
      \item Analisi dei requisiti ed eliminazione delle ambiguità;
      \item Costruzione di un glossario di termini;
      \item Raggruppamento dei requisiti in insiemi omogenei.
    \end{itemize}
  \item \textbf{Passo base}
    \begin{itemize}
      \item Definizione uno schema scheletro con i concetti più rilevanti.
    \end{itemize}
  \item \textbf{Passo iterativo}
    \begin{itemize}
      \item Raffinazione dei concetti presenti sulla base delle loro specifiche;
      \item Aggiunta di concetti necessari a descrivere specifiche non implementate.
    \end{itemize}
  \item \textbf{Analisi di qualità}
    \begin{itemize}
      \item Verifica della qualità dello schema e seguente modifica se necessaria. La qualità di uno schema concettuale
        può essere riassunta nella sua:
        \begin{itemize}
          \item \textbf{Correttezza}
          \item \textbf{Completezza}
          \item \textbf{Leggibilità}
          \item \textbf{Minimalità}
        \end{itemize}
    \end{itemize}
\end{itemize}
\section{Progettazione logica}
Giungiamo adesso alla progettazione logica del nostro schema. L'obiettivo della progettazione logica è quello di tradurre lo schema
concettuale in uno schema logico che rappresenti i dati in modo corretto ed efficiente. Prendiamo quindi in ingresso lo schema concettuale,
il modello logico usato e una serie di informazioni sul \textbf{carico applicativo}, ovvero lo sforzo a cui sarà sottoposto il sistema; in uscita
avremo uno schema logico ben definito con relativa documentazione.
\par\smallskip
\textbf{Valutazione delle prestazioni} \\
La valutazione delle prestazioni può essere effettuata attraverso determinati indicatori, ovvero parametri che caratterizzano le prestazioni. I più importanti
sono
\begin{itemize}
  \item \textbf{Spazio}: numero di occorrenze previste;
  \item \textbf{Tempo}: numero di occorrenze (o relationship) visitate per portare a termine un'operazione.
\end{itemize}
Il calcolo degli indicatori di tempo può essere effettuato attraverso una tavola degli accessi: la tavola degli accessi relativa a una data operzione
si ricava dallo schema di navigazione, costruito sul diagramma ER.
\par\smallskip
\textbf{Ristrutturazione dello schema ER} \\
A volte potrebbe essere necessario ristrutturare lo schema ER: ciò permette di semplificarne la traduzione e ottimizzare le prestazioni del sistema.
Notiamo che uno schema ER ristrutturato non è più uno schema concettuale nel senso stretto del termine. La ristrutturazione si effettua attraverso i seguenti passaggi:
\begin{itemize}
  \item \textbf{Analisi delle ridondanze} \\
    Una ridondanza in uno schema ER è un'informazione significativa ma derivabile da altre. Nella fase di progettazione logica si decide se eliminare
    le ridondanze eventualmente presenti nello schema oppure mantenerle (se non addirittura introdurne di nuove). Il vantaggio delle ridondanze è la semplificazione
    delle interrogazioni; gli svantaggi sono il maggiore spazio richiesto e l'appesantimento delle operazioni di aggiornamento.
    Le principali forme di ridondanza in uno schema ER sono:
    \begin{itemize}
      \item \textbf{Attributi derivabili} \\
        Attributi che si possono derivare da attributi o della stessa entità o di entità diverse (o relationship).
      \item \textbf{Relationship derivabili} \\
        Relationship che si possono derivare dalla composizione di altre relationship (e.g. cicli).
    \end{itemize}
  \item \textbf{Eliminazione delle generalizzazioni} \\
    Il modello relazionale non può rappresentare direttamente le generalizzazioni: dispone solamente di entità (relazioni) e relationship
    (dipendenza). Per questo motivo è necessario eliminare le gerarchie, sostituendole a loro volta con entità e relationship. Le metodologie possibili
    sono:
    \begin{itemize}
      \item \textbf{Accorpamento delle figlie} della generalizzazione nel genitore, ovvero l'eliminazione della generalizzazione figlia, che verrà
        trasformata in una relationship ad entità la cui esistenza o non è segnalata da un certo attributo ("tipo"). Si noti che l'entità di partenza mantiene i suoi attributi;
      \item \textbf{Accorpamento del genitore} della generalizzazione sulle figlie, ovvero un processo simile al precedente in cui si trasforma la gerachia in una serie di relationship. In
        questo caso gli attributi vanno a ricadere sulle figlie (si "accorpa" sulle figlie), e il genitore può essere eliminato direttamente;
      \item \textbf{Sostituzione} della generalizzazione con relationship, ovvero la trasformazione diretta in relationship, dove gli attributi dell'entità genitrici e generalizzate restano
        alle loro entità d'appartenenza, e la loro gerarchia viene trasformata in relationship con cardinalità (0,1) $\rightarrow$ (1,1).
    \end{itemize}
    La scelta fra le alternative possibili di eliminazione di generalizzazione può essere fatta sulla base della tabella degli accessi, seguendo alcune linee guida generali:
    \begin{itemize}
      \item L'accorpamento delle figlie conviene se gli accessi al padre e alle figlie sono contestuali;
      \item L'accorpamento del genitore conviene se gli accessi al padre e alle figlie sono differenti;
      \item La sostituzione conviene se gli accessi alle entità figlie sono separati da quelli al padre.
    \end{itemize}
    Notiamo poi che non sono escluse soluzioni ibride, sopratutto nel caso di gerarchie a più livelli.
  \item \textbf{Partizionamento/accorpamento di entità e relationship} \\
    Passiamo adesso alle ristrutturazioni fatte per rendere più efficienti le operazioni in base al principio di riduzione
    degli accessi. Questo si può ottenere:
    \begin{itemize}
      \item \textbf{Separando gli attributi} di un concetto a cui si accede separatamente;
      \item \textbf{Raggruppando attributi} di concetti diversi a cui si accede insieme.
    \end{itemize}
    Considereremo, per semplicità, che ad ogni accesso si legge sempre l'intera informazione. Le casistiche principali saranno:
    \begin{itemize}
      \item \textbf{Partizionamento verticale} di entità: il partizionamento verticale consiste nel dividere una data entità
        in due (sotto)entità, con un'opportuna relationship fra di esse. Ognuna delle due entità è in corrispondenza biunivoca con l'altra (cardinalità (1,1))
        con chiave esterna corrispondente alla relationship per l'entità che non si prende la chiave interna.
      \item \textbf{Partizionamento orizzontale} di relationship: il partizionamento orizzontale consiste nel dividere una data relationship dotata di attributi in più
        relationship corrispondenti a diversi valori di un attributo. Ad esempio, la divisione di una relationship d'appartenenza a due relationship di "appartenza corrente" e 
        "appartenenza passata" in termini temporali.
      \item \textbf{Eliminazione di attributi multivalore}: l'eliminazione di attributi multivalore viene incontro all'impossibilità di rappresentare attributi multivalore nel modello
        relazionale. Si trasformano quindi dati attributi in un'entità a sé stante che sarà collegata all'entità di partenza con una relazione di cardinalità (1,N) $\rightarrow$ (1,1) (o (1,N)?).
      \item \textbf{Accorpamento di entità e relationship}: l'accorpamento di entità e relationship è l'esatto opposto del partizionamento verticale: due entità
        collegate da una relationship di cardinalità (1,1) $\rightarrow$ (0,1) potranno tranquillamente essere accorpate in un'unica entità con attributi a cardinalità (0,1) corrispondenti agli ex attributi
        della seconda entità.
    \end{itemize}
  \item \textbf{Scelta di identificatori primari}
    La scelta degli identificatori primari è indispensabile alla traduzione nel modello relazionale, dove esiste il concetto di chiave. I criteri da adottare sono:
    \begin{itemize}
      \item L'assenza di opzionalità;
      \item La semplicità;
      \item Utilizzo nelle operazioni più frequenti o importanti
    \end{itemize}
  Nel caso nessuno degli identificatori soddisfi i requisiti necessari può essere necessario, sebbene sia poco elegante, introdurre nuovi attributi (codici) generati appositamente,
  magari in maniera incrementale.
\end{itemize}
\par\smallskip
\textbf{Traduzione verso il modello relazionale} \\
Approcciando la traduzione nel modello relazionale dello schema logico, dovremmo pensare di trasformare entità in relazioni sugli stessi attributi,
e relationship in relazioni sugli identificatori delle entità coinvolte.
\end{document}

\documentclass[a4paper,12pt]{article}

\usepackage[french,italian]{babel}
\usepackage[T1]{fontenc}
\usepackage[utf8]{inputenc}
\frenchspacing 
\title{Appunti Basi di Dati}
\author{Luca Seggiani}
\date{3 Maggio 2024}

\usepackage{listings}
\usepackage{xcolor}

\definecolor{codegreen}{rgb}{0,0.6,0}
\definecolor{codegray}{rgb}{0.5,0.5,0.5}
\definecolor{codepurple}{rgb}{0.58,0,0.82}
\definecolor{backcolour}{rgb}{0.95,0.95,0.92}

\lstdefinestyle{code-style}{
    backgroundcolor=\color{backcolour},   
    commentstyle=\color{codegreen},
    keywordstyle=\color{magenta},
    numberstyle=\tiny\color{codegray},
    stringstyle=\color{codepurple},
    basicstyle=\ttfamily\footnotesize,
    breakatwhitespace=false,         
    breaklines=true,                 
    captionpos=b,                    
    keepspaces=true,                 
    numbers=left,                    
    numbersep=5pt,                  
    showspaces=false,                
    showstringspaces=false,
    showtabs=false,                  
    tabsize=2
}

\lstset{style=code-style}

\begin{document}
\maketitle
\section{Dipendenze funzionali}
Definiamo un metodo per valutare formalmente la qualità della progettazione degli schemi relazionali, ovvero misurare se un raggruppamento
di attributi è migliore o peggiore di un altro, senza doversi affidare al buon senso
del progettista dello schema o del modello ER.
Seguiamo l'approccio top-down: iniziamo dall'individuare raggruppamenti più generali di attributi, ed effettuiamo poi decomposizioni
successive per raffinare tali raggruppamenti generali. 
Gli obiettivi che abbiamo durante lo sviluppo del processo logico sono:
\begin{itemize}
  \item \textbf{Conservazione dell'informazione}, ovvero il mantenimento delle informazioni che ci interessano della realtà in analisi.
  \item \textbf{Minimizzazione della ridondanza}, ovvero la riduzinoe al minimo della memorizzazione ripetuta della stessa informazione.
\end{itemize}
Possiamo ricavare da questo alcune linee guida:
\begin{itemize}
  \item \textbf{Linea guida 1: semplice è bello} \\
    Uno schema di relazione deve essere progettato in modo da essere semplice da capire. Non si devono raggruppare attributi
    da più tipi di entità in un'unica relazione. Intuitivamente, anzi, dovremmo far corrispondere ad ogni schema di relazione
    una sola entità o una sola relationship, in modo da evitare \textbf{amibiguità semantiche}.
  \item \textbf{Linea guida 2: no alle anomalie} \\
    Gli schemi di basi di dati vanno progettati in modo che non si possano presentare anomalie in fase di definizione dei dati.
    La mancanza di anomalie va certificata attraverso una descrizione fondamentale della semantica dei fatti descritti della
    realtà di interesse. Se si possono presentare anomalie, esse vanno rilevate e si devi assicurare che i programmi che aggiornano la base
    operino correttamente e in sicurezza.
  \item \textbf{Linea guida 3: evitare valori NULL} \\
    Conviene evitare il più possibile i valori NULL, in quanto riempiono lo schema di informazioni inutili.
    I valori NULL possono rivelarsi necessari solamente nei casi eccezionali rispetto alla cardinalità di una relazione.
\end{itemize}
\par\smallskip
\textbf{Dipendenza Funzionale} \\
Una dipendenza funzionale (\textit{functional dependency, FD}) esprime un legame semantico tra due gruppi di attributi di uno schema di relazione $R$.
Una FD è una proprietà di $R$, non un suo particolare stato valido $r$ di $R$. Una FD non può quindi essere dedotta
da uno stato valido $r$, ma deve essere definita esplicitamente sulla base della semantica degli attributi di $R$.
\par\smallskip
\textbf{Forma normale} \\
Una forma normale è una proprietà di una base di dati relazionale che ne garantisce la qualità, ovvero l'assenza di determinati difetti (quelli di cui si parla nelle
linee guida). Una relazione non normalizzata presenta:
\begin{itemize}
  \item \textbf{Ridondanze};
  \item \textbf{Comportameti anomali} in fase di aggiornamento.
\end{itemize}
La normalizzazione è solitamente definita sui modelli relazionali, ma ha senso in altri contesti, ad esempio il modello ER.
\par\smallskip
\textbf{Normalizzazione} \\
La procedura di normalizzazione (che è un vero è proprio algoritmo) serve a trasformare schemi non normalizzati in schemi che godono della forma normale.
La normalizzazione va utilizzata come tecnica di verifica di una base di dati, e non costituisce una metodologia di progettazione. Cercare di progettare uno schema di relazione
attraverso la normalizzazione sarebbe infatti troppo complesso per rappresentare un'opzione viabile.
\par\smallskip
\textbf{Definizione (pseudo)formale di dipendenza funzionale} \\
Data una relazione $r$ su $R(X)$, e due sottoinsiemi non vuoti $Y$ e $Z$ di $x$.
Esiste una dipendenza funzionale in $r$ da $Y$ a $Z$ se, per ogni coppia di n-uple $t_1$ e $t_2$ di $r$ con gli stessi valori di $Y$,
risulta che $t_1$ e $t_2$ hanno gli stessi valori anche su $Z$. Come avevamo già visto, la notazione è:
$$ Y \rightarrow Z $$
Notiamo poi che $Y \rightarrow Z$ non implica $Z \rightarrow Y$.
Una dipendenza funzionale è detta \textbf{completa} quando $Y\rightarrow Z$ e per ogni $W \subset Y$, non vale $W\rightarrow Z$.
Se $Y$ è una superchiave di $R(X)$, allora $Y$ determina ogni altro attributo della relazione, ergo $Y \rightarrow X$. A questo punto, se $Y$ è una chiave
(ovvero superchiave minimale), si ha che $Y\rightarrow X$ dalla chiave a $R(X)$ è completa.
Una dipendenza funzionale si dice \textbf{banale} se è sempre soddisfatta:
\begin{itemize}
  \item $Y \rightarrow Y$ è banale;
  \item $Y \rightarrow A$ se $A \not\in Y$ non è banale;
  \item $Y \rightarrow Z$ non è banale se nessun attributo di $Z$ appartiene a $Y$ (è una condizione più generale della precedente).
\end{itemize}
\par\smallskip
\textbf{Legami fra dipendenze funzionali e anomalie} \\
Le dipendenze funzionali possono essere usate per verificare l'eventuale presenza di anomalie in un progetto. Tornano utili anche nella normalizzazione
di schemi. Data la loro importanza, indicheremo con $R(X,F)$ uno schema di relazione $R(X)$ che rispetta un'insieme di dipendenze funzionali $F$.
\par\smallskip
\textbf{Implicazione} \\
Sia $F$ un insieme di dipendenze funzionali definite su $R(Z)$, e sia $X \rightarrow Y$. Si dice che $F$ implica (logicamente)
$X \rightarrow Y$ se, per ogni possibile istanza $r$ di $R$ che verifica tutte le dipendenze funzionali di $F$, risulta verificata anche
la dipendenza funzionale $X \rightarrow Y$. In simboli, $F \models X \rightarrow Y$. La definizione di implicazione non è direttamente
utilizzabile nella pratica. Essa prevede una quantificazione universale sulle istanze della base di dati, e non disponiamo di un'algoritmo
per calcolare tutte le dipendeze funzionali implicate da un'insieme $F$.
\par\smallskip
\textbf{Regole di inferenza di Armstrong} \\
Armstrong (1974) fornisce delle regole di inferenza che permettono di derivare costruttivamente tutte le dipendenze funzionali che sono implicate da un insieme di dipendenze iniziale.
Esse sono:
\begin{itemize}
  \item \textbf{Riflessività:} 
    $$ Y \subseteq X \models X \rightarrow Y$$
  \item \textbf{Additività} (o arricchimento):
    $$ X \rightarrow Y \models XZ \rightarrow YZ \  \forall Z $$
  \item \textbf{Transitività:}
    $$ X \rightarrow Y \wedge Y \rightarrow Z \models X \rightarrow Z $$
\end{itemize}
\par\smallskip
\textbf{Derivazione} \\
Le regole di inferenza di Armstrong permettono di definire derivazioni. Dato un insieme di regole di inferenza $RI$, un insieme di dipendenze funzionali $F$,
e una dipendenza funzionale $f$, una derivazione di $f$ da $F$ è una sequenza finita $f_1,...,f_n$ tale che:
\begin{itemize}
  \item $ f_n = f $
  \item ogni $f_i$ è un elemento di $F$ o è ottenuta dalle precedenti dipendenze $f_1,...,f_{n-1}$ attraverso una regola di inferenza $RI$.
\end{itemize}
Indichiamo con $F \vdash X \rightarrow Y$ il fatto che la dipendenza funzionale $X \rightarrow Y$ sia derivabile da $F$ usando $RI$.
Le regole di derivazione comunemente usate sono, a partire dalle regole di inferenza di Armstrong:
\begin{itemize}
  \item \textbf{Unione:} 
    $$ \{ X\rightarrow Y, X \rightarrow Z \} \vdash X\rightarrow YZ $$
    Dimostrazione:
    \begin{enumerate}
      \item $X \rightarrow Z$ da ipotesi;
      \item $X (X) \rightarrow ZX $ da additività di (1);
      \item $X \rightarrow Y$ da ipotesi;
      \item $ZX \rightarrow YZ$ da additività di (3);
      \item $X \rightarrow ZX, \ ZX \rightarrow YZ \Rightarrow X \rightarrow YZ$ dalla transitività di (2) e (4).
    \end{enumerate}
  \item \textbf{Decomposizione:}
    $$ \{ X \rightarrow YZ \} \vdash X \rightarrow Y$$
    Dimostrazione:
    \begin{enumerate}
      \item $X \rightarrow YZ$ da ipotesi;
      \item $YZ \rightarrow Y$ da riflessività;
      \item $X \rightarrow YZ, \ YZ \rightarrow Y \Rightarrow X \rightarrow Y$ dalla transitività di (1) e (2).
    \end{enumerate}
  \item \textbf{Indebolimento:}
    $$ \{ X\rightarrow Y \} \vdash XZ \rightarrow Y $$
    \begin{enumerate}
      \item $XZ \rightarrow X$ per riflessività;
      \item $X \rightarrow Y$ da ipotesi;
      \item $XZ \rightarrow X \ X \rightarrow Y \Rightarrow XZ \rightarrow Y$ dalla transitività di (1) e (2).
    \end{enumerate}
  \item \textbf{Identità:}
    $$ \{\} \vdash X \rightarrow X$$
\end{itemize}
\par\smallskip
\textbf{Chiusura di un insieme di attributi} \\
Dato uno schema $R(T,F)$ con $X \subseteq T$, la chiusura di $X$ rispetto a $F$, indicata col simbolo $R_F^+$, è definita come:
$$ X_F^* = \{A\subset T|F\vdash X\rightarrow A\}$$
Se non vi sono amibiguità si può semplicemente scrivere $X^+$.
\par\smallskip
\textbf{Teorema di chiusura degli attributi} \\
Se è possibile, usando le regole di inferenza, scrivere, partendo da $F$, che $X\rightarrow Y$, allora $Y$ è contenuto nella chiusura di $X$ e viceversa.
In simboli:
$$ F \vdash X\rightarrow Y \Leftrightarrow Y \subseteq X^+ $$
\par\smallskip
\textbf{Correttezza e completezza} \\
Dato un qualche insieme di regole di inferenza $RI$ e un insieme di dipendenze funzionali $F$, $RI$ è corretto se:
$$ F \vdash X\rightarrow Y \Rightarrow F\models X \rightarrow Y$$
Ovvero applicando $RI$ ad un insieme $F$ di dipendenze funzionali, si ottengono solo dipendenze logicamente implilcate da $F$.
Inoltre $RI$ è completo se:
$$ F\models X\rightarrow Y \Rightarrow F \vdash X\rightarrow Y $$
Ovvero applicando $RI$ a un insieme $F$ di dipendenze funzionali si ottengono tutte le dipendenze logicamente implicate da $F$.
A questo punto si può enunciare:
\par\smallskip
\textbf{Teorema fondamentale della correttezza e completezza} \\
Le regole di inferenza di Armstrong sono corrette e complete. Questo teorema ci permette di scambiare i simboli $\models$ e $\vdash$. In particolare questo
si applica alla definizione di chiusura di attributi, cioè:
$$ X_F^+ = \{A\subset T|F\models X\rightarrow A\}$$
Si può dimostrare che le regole di inferenza di Armstrong sono minimali, ovvero che non se ne può ignorare anche soltanto una senza privarle della completezza.
Esistono però altri insieme di regole di inferenza completi che non coincidono con le regole di inferenza di Armstrong.
\par\smallskip
\textbf{Chiusura di un'insieme di dipendenze funzionali} \\
Sia $F$ un insieme di dipendenze funzionali definite su $R(Z)$. Allora la chiusura di $F$ è l'insieme $F^+$ di tutte e sole le dipendenze
funzionali implicate da $F$:
$$ F^+ = \{ X\rightarrow Y| F \models X \rightarrow Y \} $$
Dato un'insieme di dipendeze funzionali $F$ definite su $R(Z)$, un'istanza $r$ di $R$ che soddisfa $F$ soddisfa anche le dipendenze
funzionali di $F^+$.
\par\smallskip
\textbf{Algoritmo per il calcolo di $\mathbf{F^+}$} \\
Riportiamo adesso un'algoritmo per il calcolo della chiusura di un insieme di dipendenze funzionali $F^+$ a partire da $F$, usando le regole di inferenza di Armstrong:
\begin{itemize}
  \item Input: $R(T,F)$;
  \item Output: $F^+$ \textit{chiusura}.
\end{itemize}
\begin{lstlisting}[language=perl]
metti F in F+
while (F+ non cambia) do
  foreach f in F+ do
    #applica riflessivita' e addittivita' a f e aggiungi le dipendenze ottenute a F+
  foreach f_1, f_2 in F+ do
    #se possibile, applica la transitivita' a f_1 e f_2 e aggiungi a F+ la dipendenza ottenuta
return F+
\end{lstlisting}
\end{document}

\documentclass[a4paper,12pt]{article}

\usepackage[french,italian]{babel}
\usepackage[T1]{fontenc}
\usepackage[utf8]{inputenc}
\frenchspacing 
\title{Appunti Basi di Dati}
\author{Luca Seggiani}
\date{8 Maggio 2024}

\usepackage{listings}
\usepackage{xcolor}

\definecolor{codegreen}{rgb}{0,0.6,0}
\definecolor{codegray}{rgb}{0.5,0.5,0.5}
\definecolor{codepurple}{rgb}{0.58,0,0.82}
\definecolor{backcolour}{rgb}{0.95,0.95,0.92}

\lstdefinestyle{code-style}{
    backgroundcolor=\color{backcolour},   
    commentstyle=\color{codegreen},
    keywordstyle=\color{magenta},
    numberstyle=\tiny\color{codegray},
    stringstyle=\color{codepurple},
    basicstyle=\ttfamily\footnotesize,
    breakatwhitespace=false,         
    breaklines=true,                 
    captionpos=b,                    
    keepspaces=true,                 
    numbers=left,                    
    numbersep=5pt,                  
    showspaces=false,                
    showstringspaces=false,
    showtabs=false,                  
    tabsize=2
}

\lstset{style=code-style}

\begin{document}
\maketitle
Calcolare $F^+$ è molto costoso: l'algoritmo è a complessità esponenziale. Spesso però, quello che ci
interessa beramente e verificare se $F^+$ contiene una certa dipendenza e non generare l'intera lista
di chiusura. Per fare ciò basta calcolare $X^+$ per il teorema di chisura degli attributi:
$$ F \vdash X \rightarrow Y \Leftrightarrow Y \subseteq X^+ $$
\par\smallskip
\textbf{Algoritmo per il calcolo di $\mathbf{X}^+$} \\
Vediamo quindi come calcolare la chisurua $X^+$:
\begin{itemize}
  \item Input: $R(T,F), \quad X \subseteq T$;
  \item Output: $X^+$ \textit{chiusura}.
\end{itemize}
\begin{lstlisting}[language=perl]
metti X in X+
while (X+ non cambia) do
  foreach W -> V in F do
    if W in X+ and V not in X+ then
      metti V in X+
return X+
\end{lstlisting}
\par\smallskip
\textbf{Chiavi} \\
Dato uno schema $R(T,F)$, un'insieme di attributi $K \subseteq T$ si dice \textbf{superchiave} di $R$ se la dipendenza funzionale
$K \rightarrow T$ è implicata da $F$, ovvero se $K \rightarrow T \in F^+$. Un'insieme di attributi $K \subseteq T$ si dice a questo punto \textbf{chiave}
di $R$ se $K$ è una superchiave di $R$ e non esiste alcun sottoinsieme proprio di $K$ che sia superchiave di $R$.
Dato che in uno schema possono esserci più chiavi, di solito si identifica una chiave primaria che possa fare da identificatore
per tutte le n-uple dello schema. Tutte le altre chiavi si dicono chiavi candidate.
\par\smallskip
\textbf{Algoritmo per tutte le chavi} \\
Il problema di trovare tutte le chiavi di una relazione $R(Z)$ ha complessità esponenziale nel caso peggiore:
\begin{itemize}
  \item Gli attributi che stanno solo a sinistra sono in tutte le chiavi. Si chiami $N$ questo insieme;
  \item Gli attributi che stanno solo a destra non sono in nessuna chiave. Si chiami $M$ questo insieme;
  \item Si aggiunge a $N$ un'attributo alla volta tra quelli che non sono né in $N$ nè in $M$, poi una coppia di attributi,
    e così via. Chiamiamo $X_i$ questo sottoinsieme di attributi: ad ogni aggiunta controlleremo se la dipendenza $N \cup X_i \rightarrow Z$ esiste.
\end{itemize}
\par\smallskip
\textbf{Verifica di una chiave} \\
Spesso è molto più conveniente verificare se una chiave è tale, piuttosto che trovare ogni possibile chiave. Per fare ciò,
possiamo usare l'algoritmo per il calcolo della chiusura di un'insieme di attributi:
\begin{itemize}
  \item $X\subseteq T$ è superchiave di $R(T,F)$ se e solo se $X \rightarrow T \in F^+$, ovvero $T\subseteq X^+$
  \item $X\subseteq T$ è chiave di $R(T, F)$ se e solo se $T \subseteq X^+$, e non esiste $Y \subset X$ tale che
    $T \subseteq Y^+$.
\end{itemize}
\par\smallskip
\textbf{Equivalenza} \\
Due insiemi di dipendenze funzionali $F$ e $G$ sugli attributi $T$ di una relazione $R(T)$ sono equivalenti, in simboli $T \equiv G$, se e solo
se $F^+ = G^+$. La relazione di equivalenza permette di stabilire se due insiemi di dipendenza rappresentano gli stessi fatti. Per verificare
l'equivalenza è sufficiente che:
\begin{itemize}
  \item Tutte le dipendenze di $F$ appartengano a $G^+$;
  \item Tutte le dipendenze di $G$ appartengano a $F^+$.
\end{itemize}
\par\smallskip
\textbf{Ridondanza} \\
Sia $F$ un insieme di dipendenze funzionali. Data $X\rightarrow Y \in F$, $X$ contiene un \textbf{attributo estraneo} se e solo se:
$$ (F - \{X\rightarrow Y\}) \cup (X - \{A\rightarrow\} Y) \equiv F$$
ovvero $X - \{A\} \rightarrow Y \in F^+$ \\ 
$X\rightarrow Y$ è una \textbf{dipendenza ridondante} se e solo se $(F - \{X \rightarrow Y\}) \equiv F$, ovvero $X\rightarrow Y \in (F - \{X\rightarrow Y\})^+$.
Le dipendenze che non contengono attributi estranei e la cui parte destra è un unico attributo sono dette \textbf{dipendenze elementari}.
\par\smallskip
\textbf{Copertura minimale} \\
Sia $F$ un'insieme di dipendenze funzionali. $F$ è una \textbf{copertura minimale} (detta anche \textit{insieme minimale} e \textit{copertura canonica}) se e solo se:
\begin{itemize}
  \item Ogni parte destra di una dipendenza ha un unico attributo;
  \item Le dipendenze non contengono attributi estranei;
  \item Non esistono dipendenze ridondanti;
\end{itemize}
In soldoni, la copertura minimale rappresenta l'insieme di dipendenze funzionali $F'$ più piccolo che implica tutte le dipendenze di $F$.
\par\smallskip
\textbf{Algoritmo per il calcolo della copertura minimale}
\begin{itemize}
  \item Input: F;
  \item Output G \textit{copertura minimale}:
\end{itemize}
\begin{lstlisting}[language=perl]
metti F in G
for each X -> G do
  metti X in Z
  for each A in X Do
    if Y in (Z - {A})+_F then
      metti Z - {A} in Z
    metti (G - {X -> Y}) U (Z -> Y) in G
  foreach F in G do
    if f in (G-{f})+ then
      metti G - {f} in G
return G
\end{lstlisting}
\par\smallskip
\textbf{Teorema della copertura minimale} \\
Il precedente algoritmo dimostra il fatto che per ogni insieme di dipendenze funzionalil $F$ esiste una copertura minimale.
Si noti che questo non afferma che la copertura minimale è unica: possono tranquillamente esistere coperture minimali equivalenti.
\section{Normalizzazione}
\par\smallskip
\textbf{Eliminare le anomalie} \\
La teoria che abbiamo sviluppato finora viene usata per identificare le anomalie in uno schema mal definito. Definiamo quindi il concetto di \textbf{forma normale}:
una forma normale è una proprietà che deve essere soddisfatta dalla dipendenza fra attibuti di schemi "ben fatti". Noi vedremo due esempi:
la \textbf{forma normale di Boyce-Codd}, e un suo miglioramento, la \textbf{terza forma normale}. 
\par\smallskip
\textbf{Forma normale di Boyce-Codd} \\
Uno schema $R(T,F)$ è in forma normale di Boyce-Codd (BCNF) se e solo se per ogni dipendenza funzionale non banale
$ X \rightarrow Y \in F^+ $, $X$ è superchiave di $R$. Per definizione, il fatto che uno schema sia o meno in BCNF dipende
dalla chiusura $F^+$, non dalla specifica copertura $F$. Calcolare $F^+$, come abbiamo visto, ha complessità esponenziale:
fortunatamente esiste un'algoritmo di complessità polinomiale per valutare se uno schema è in forma BCNF. Si usa il corollario:
uno schema $R(T,F)$ con $F$ copertura minimale è in BCNF se e solo se per ogni dipendenza funzionale elementare $X \rightarrow A \in F$, 
$X$ è superchiave.
\begin{itemize}
  \item Input: $R(T,F)$;
  \item Output true se $R$ è in BCNF, false altrimenti. 
\end{itemize}
\begin{lstlisting}[language=perl]
for each X -> Y in F do
  if Y not in X and T not in X+ then
    return false
return true
\end{lstlisting}
\end{document}

\documentclass[a4paper,12pt]{article}

\usepackage[french,italian]{babel}
\usepackage[T1]{fontenc}
\usepackage[utf8]{inputenc}
\frenchspacing 
\title{Appunti Basi di Dati}
\author{Luca Seggiani}
\date{15 Maggio 2024}

\usepackage{listings}
\usepackage{xcolor}

\definecolor{codegreen}{rgb}{0,0.6,0}
\definecolor{codegray}{rgb}{0.5,0.5,0.5}
\definecolor{codepurple}{rgb}{0.58,0,0.82}
\definecolor{backcolour}{rgb}{0.95,0.95,0.92}

\lstdefinestyle{code-style}{
    backgroundcolor=\color{backcolour},   
    commentstyle=\color{codegreen},
    keywordstyle=\color{magenta},
    numberstyle=\tiny\color{codegray},
    stringstyle=\color{codepurple},
    basicstyle=\ttfamily\footnotesize,
    breakatwhitespace=false,         
    breaklines=true,                 
    captionpos=b,                    
    keepspaces=true,                 
    numbers=left,                    
    numbersep=5pt,                  
    showspaces=false,                
    showstringspaces=false,
    showtabs=false,                  
    tabsize=2
}

\lstset{style=code-style}

\begin{document}
\maketitle
\par\smallskip
\textbf{Decomposizione di schemi} \\
Dato unos schema $R(T)$, l'insieme di schemi $\rho = \{ R_1(T_1), ..., R_k(T_k) \}$ è una \textbf{decomposizione} di $R$
solo se $\cup_i T_i = T$ (l'unione degli schemi dà lo schema di partenza). Si noti che questo non richiede che gli schemi $R_i$ siano disgiunti. Una decomposizione equivale allo schema di
partenza in quanto
\begin{itemize}
  \item Preserva i dati;
  \item Preserva le dipendenze funzionali.
\end{itemize}
\par\smallskip
\textbf{Teorema della perdita di dati} \\
Si ha che se $\rho = \{ R_1(T_1)....,R_k(T_k) \}$ è una decomposizione di $R(T,F)$, allora per ogni relazione $r$ che
soddisfa $R(T, F)$:
$$ r \subseteq \pi_{T_1}(r) \bowtie ... \bowtie \pi_{T_k}(r)$$
Ciò significa che una decomposizione ha perdita di dati
quando, ricostruendo una relazione, otteniamo più n-uple della relazione originaria.
\par\smallskip
\textbf{Decomposizioni che preservano i dati} \\
Dato uno schema $R(T,F)$ e una decomposizione $\rho \{ R_1(T_1), ..., R_k(T_k) \}$, $\rho$ preserva i dati se e solo se, per ogni relazione
$r$ che soddisfa $R(T,F)$ si ha:
$$ r = \pi_{T_1}(r) \bowtie ... \bowtie \pi_{T_k}(r) $$
Ciò significa che, per una decomposizione che preserva i dati, ogni istanza valida $r$ della relazione di partenza deve essere identica
al join naturale delle sue proiezioni sui $T_1$ (cioè ricostruendo la relazione si ottiene la relazione di partenza e nulla di più).
\par\smallskip
\textbf{Teorema di preservazione dei dati} \\
Sia $\rho = \{R_1(T_1),R_2(T_2)\}$ una decomposizione di $R(T,F)$. Essa preserva i dati se e solo se $T_1 \cap T_2 \rightarrow T_1 \in F^+$ oppure
$T_1 \cap T_2 \rightarrow T_2 \in F^+$. In altre parole, gli attributi comuni alle due relazioni devono essere chiave in una delle due tabelle (relazioni).
\par\smallskip
\textbf{Proiezione di un insieme di dipendenze} \\
Dato $R(T, F)$ e $T_1 \subseteq T$, la proiezione dell'insieme di dipendenze $F$ sull'insieme di attributi $T_i$ è:
$$ \pi_{T_i} = \{ X \rightarrow Y \in F^+ | X, Y \subseteq T_i \} $$
Nota bene che la proiezione si costruisce sulle dipendenze di $F^+$, non di $F$ soltanto.
\par\smallskip
\textbf{Algoritmo per il calcolo di $\pi_{T_i}(F)$} \\
Si presenta un'algoritmo per il calcolo della proiezione dell'insieme di dipendenze appena definita.
\begin{itemize}
  \item Input: $R(T, F)$ e $T_i \subseteq T$
  \item Output: $\pi_{T_i}(F)$
  \end{itemize}
\begin{lstlisting}[language=perl]
Z vuoto
for each Y in T do
  metti Y - Y in W
  metti Z unito (Y -> (W intersecato T_i)) in Z
return Z
\end{lstlisting}
Questo algoritmo ha, nel caso pessimo, complessità esponenziale.
\par\smallskip
\textbf{Decomposizioni che preservano le dipendenze} \\
Dato uno schema $R(T,F)$, e una decomposizione $\rho = \{ R_1(T_1),...,R_k(T_k) \}$, $\rho$ è una
decomopsizione di $R(T,F)$ che preserva le dipendenze se e solo se $\cup_i \pi_{T_i}(F) \equiv F$
In altre parole, la decomposizione di $R(T, F)$ in due relazioni con attributi $X$ e $Y$ preserva le dipendenze
se $\pi_X(F) \cup \pi_Y(F) \equiv F$, cioè se $(\pi_X(F)\cup\pi_Y(F)) = F^+$. Dovremo verificare quest'ultima
uguaglianza per dimostrare che una decomposizione di $R(T, F)$ preserva le dipendenze. Fare ciò significa:
\begin{itemize}
  \item Calcolare la proiezione di un insieme di dipendenze funzionali su un insieme di attributi, cosa che
    richiede un algoritmo a complessità esponenziale;
  \item Determinare l'equivalenza di due insiemi di dipendenze funzionali $X$ e $G$, cosa che richiede un'algoritmo a 
    complessità poliniomiale:
    \begin{itemize}
      \item Per ogni $X \rightarrow Y \in F$, calcoliamo $X^+_G$ e verifichiamo se $Y \in X^+_G$;
      \item Per ogni $X \rightarrow Y \in G$, calcoliamo $X^+_F$ e verifichiamo se $Y \in X^+_F$;
    \end{itemize}
\end{itemize}
\par\smallskip
\textbf{Algoritmo per la decomposizione in BCNF} \\
Possiamo adesso vedere un'algoritmo che decompone uno schema $R(T, F)$ nella sua forma normale di Boyce-Codd:
\begin{itemize}
  \item Input: $R(T, F)$ con $F$ in forma $X\rightarrow A$, non normalizzata
  \item Output: $\rho$ che preserva i dati 
\end{itemize}
\begin{lstlisting}[language=perl]
metti R(T,F) in P
while esiste R_i(T_i, F_i) in P che non e in BCNF do
  for each X -> A in F_i do
    if A not in X and T_i not in X+ then
      metti R_i(T_i - A, proiezione su T_1 - A di F_i) in R_1 #questi sono tutti gli attributi di R - A
      metti R_i(X + A, proiezione su X + A di F_i) in R_2 #questi sono tutti gli attributi della dipendenza funzionale in esame
      metti P - (R_i unito a {R_1,R_2}) in P
      break
return P
\end{lstlisting}
Questo algoritmo termina e produce una decomposizione della relazione tale che:
\begin{itemize}
  \item La decomposizione prodotta è in BCNF;
  \item La decomposizione prodotta preserva i dati.
\end{itemize}
Non è detto che la decomposizione preservi le dipendenze!
\par\smallskip
\textbf{Qualità delle decomposizioni} \\
Una decomposizione dovrebbe sermpe garantire:
\begin{itemize}
  \item Di essere in BCNF;
  \item Di non presentare perdite di dati, in modo da permettere la ricostruzione dell'informazione originale attraverso join naturali;
  \item Di conservare le dipendenze funzionali, in modo da mantenere i vincoli di integrità originari.
\end{itemize}
Quando non si riesce a raggiungere una forma normale di Boyce-Codd, probabilmente c'è stato un cattivo processo di progettazione prima della normalizzazione.
In ogni caso, esiste una forma normale alternativa (e meno restrittiva) alla Boyce-Codd:
\par\smallskip
\textbf{Terza forma normale} \\
Una relazione $R(T, F)$ è in terza forma normale (3NF) se e solo se, per ogni dipendenza funzionale non banale $X \rightarrow A \in F^+$,
è verificata almeno una delle due condizioni:
\begin{itemize}
  \item $X$ è una superchiave di $R$ (come nella Boyce-Codd);
  \item $A$ è contenuto in almeno una chiave di $R$ (si dice che $A$ è un attributo primo).
\end{itemize}
Vediamo che la BCNF implica la 3NF: uno schema in BCNF è automaticamente in 3NF, ma non tutti gli schemi in 3NF sono in BCNF.
Il problema della 3NF è la sua \textbf{verifica}: il problema di decisione è NP-completo. Il miglior algoritmo deterministico
di risoluzione ha complessità esponenziale nel caso peggiore. Questo è dato dalla ricerca degli attributi primi, cioè le chiavi, che ha
complessità esponenziale. Tuttavia, si può sempre ottenere una decomposizione in 3NF che preserva dati e dipendenze funzionali.
\par\smallskip
\textbf{Algoritmo per la decomposizione in 3NF} \\
Ci basiamo su un intuizione: dato un insieme di attributi $T$ e una copertura minimale $G$, si divide $G$ in gruppi $G_i$ in modo che tutte le dipendenze
funzionali di ogni gruppo $G_i$ abbiano la stessa parte sinistra. Da ogni gruppo $G_i$ si definisce uno schema di relazione composto
da tutti gli attributi che appaioni in $G_i$, la cui chiave (detta \textbf{chiave sintetizzata}) è la parte sinistra comune. L'algoritmo risulta quindi:
\begin{itemize}
  \item Input: $R(T, F)$
  \item Output: $\rho$ che preserva dati, dipendenze ed è in 3NF 
\end{itemize}
\begin{lstlisting}[language=perl]
metti la copertura minimale di F in G
poni P vuoto
#sostituisci ogni insieme di dipendenze {X -> A_1, ..., X -> A_h} con X -> A_1, ..., A_h
for each X -> Y in G do
  #crea uno schema con attributi XY in P
#elimina da P ogni schema che sia contenuto in un'altro schema di P
#se P non contiene nessuno schema i cui attributi costituiscono una superchiave di R, aggiungere a P uno schema con attributi W, dove W e' una chiave di R
\end{lstlisting}
L'esecuzione dell'algoritmo termina e produce una decomposizione tale che:
\begin{itemize}
  \item La decomposizione prodotta è in 3NF;
  \item La decomposizione prodotta preserva i dati e le dipendenze funzionali.
\end{itemize}
Questo algoritmo ha complessità polinomiale: paradossalmente, trovare una soluzione è polinomiale, ma verificarla è esponenziale
\end{document}

\end{document}