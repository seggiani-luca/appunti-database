\documentclass[a4paper,12pt]{article}

\usepackage[french,italian]{babel}
\usepackage[T1]{fontenc}
\usepackage[utf8]{inputenc}
\frenchspacing 
\title{Appunti Basi di Dati}
\author{Luca Seggiani}
\date{29 Maggio 2024}

\begin{document}
\maketitle
\section{Gestione dell'affidabilità}
Il gestore dell'affidabilità si occupa dell'esecuzione dei comandi transazionali, assicurando le loro caratteristiche
acide. I \textbf{comandi transazionali} sono:
\begin{itemize}
  \item Start transaction (B, \textit{begin});
  \item Commit work (C, \textit{commit});
  \item Rollback work (A, \textit{abort}).
\end{itemize}
Il gestore si occupa anche delle operazioni di \textbf{ripristino} (\textit{recovery}) dopo eventuali guasti:
\begin{itemize}
  \item \textbf{Ripresa a caldo} (\textit{warm restart});
  \item \textbf{Ripresa a freddo} (\textit{cold restart}).
\end{itemize}
Il gestore dell'affidabilità mantiene un \textbf{log} delle transazioni: un'archivio permanente che registra le operazioni svolte.
Il gestore dell'affidabilità lavora insieme al gestore dei metodi d'accesso (che gestisce i metodi di sistema \textit{fix} e \textit{unfix}),
e al gestore delle transazioni.
\par\smallskip
\textbf{Persistenza delle memorie} \\
Ricordiamo brevemente il concetto di persistenza di una memoria. Riguardo alle memorie di interesse nel nostro caso, si ha che:
\begin{itemize}
  \item \textbf{Memoria centrale:} è persistente, l'informazione viene però distrutta da qualsiasi guasto di sistema;
  \item \textbf{Memoria di massa:} è persistente e sopravvive ai guasti di sistema, ma può comunque danneggiarsi;
  \item \textbf{Memoria stabile:} astrazione che rappresenta una memoria persistente non danneggiabile, che viene
    perseguita nella realtà attraverso la ridondanza:
    \begin{itemize}
      \item Dischi replicati;
      \item Nastri magnetici;
      \item ecc...
    \end{itemize}
\end{itemize}
\par\smallskip
\textbf{Log} \\
Il log è un file \textbf{sequenziale} (quindi non ad accesso casuale) gestito dal gestore dell'affidabilità, scritto in 
memoria stabile. Riporta tutta le operazioni svolte sulla base di dati, nell'ordine in cui vengono svolte. E' formato da
\textbf{record}:
\begin{itemize}
  \item \textbf{Operazioni delle transazioni}
    \begin{itemize}
      \item begin, B(T);
      \item insert, I(T, O, AS) (nel record O, T memorizza per la prima volta l\textbf{after state} AS);
      \item delete, D(T, O, BS) (nel record O, T elimina il \textbf{before state} BS);
      \item update, U(T, O, BS, AS) (nel record O, T elimina il before state BS e memorizza l'after state AS);
      \item commit, C(T);
      \item abort, A(T).
    \end{itemize}
  \item \textbf{Record di sistema}:
    \begin{itemize}
      \item dump;
      \item checkpoint.
    \end{itemize}
\end{itemize}
Queste ultime due transazioni servono a fare regisitrazioni dell'intero stato di un database (\textit{dump}) o delle sue transazioni (\textit{checkpoint}).
\par\smallskip
\textbf{Checkpoint e dump} \\
Il log ha la funzione di permettere la ricostruzione delle operazioni. Checkpoint e dump forniscono stadi intermedi di ricostruzione,
da utilizzare in caso di guasti.
\begin{itemize}
  \item \textbf{Checkpoint} \\
    Il checkpoint registra quali transazioni sono attive in un dato istante temporale, cioè "a metà strada". Ha lo scopo
    di confermare che altre transazioni o non sono ancora iniziate, sono già finite. Per tutte le transazioni che hanno effettuate
    il commit i dati possono essere trasferiti in memoria di massa. Esistono più modalità di checkpoint, di cui ne vedremo una:
    \begin{itemize}
      \item Si sospende l'accettazione di operazioni di commit e abort finché non è completata la registrazione;
      \item Si forza la scrittura in memoria di massa delle pagine in memoria modificate da transazioni che hanno
        già fatto commit;
      \item Si forza la scrittura nel log di un record contenente gli identificatori di tutte le transazioni attive;
      \item Si ricomincia ad accettare tutte le operazioni da parte delle transazioni.
    \end{itemize}
    Con questo meccanismo si assicura la persistenza delle transazioni che hanno eseguito il commit.
  \item \textbf{Dump} \\
    Il dump è una copia completa della base di dati (un \textit{backup}). Solitamente viene prodotta mentre il sistema è inattivo,
    salvato in memoria stabile, e registrato in un record di dump nel log che indica il momento in cui il dump è stato effettuato,
    vari dettagli pratici, ecc...
\end{itemize}
\par\smallskip
\textbf{Esito di una transazione} \\
Vediamo cosa succede ad una transazione quando si verifica un guasto. L'esito di una transazione è determinato irrevocabilmente
quando viene scritto il record di commit nel log in modo \textbf{sincrono} e forzato. Un guasto prima di tale istante porta
ad un UNDO di tutte le azioni, e ad una ricostruzione dello stato originario (prima della transazione) della base di dati. Un guasto dopo
tale istante non deve avere consegueze sugli effetti di tale transazioni: lo stato finale deve essere ricostruito, se necessario con
un REDO. I record di abort possono essere invece scritti in modo \textbf{asincrono}.
\par\smallskip
\textbf{Regole di modifica del log} \\
Esistono due regole per la scrittura nel log:
\begin{itemize}
  \item \textbf{Write-Ahead} \\
    Si scrive la parte BS (before state) del log prima di effettuare l'operazione sulla base di dati. Questo
    consente di disfare le azioni già memorizzate (UNDO) di transazioni senza commit avendo in memoria stabile
    un valore corretto.
  \item \textbf{Commit-Precedenza} \\
    Si scrive la parte AS (after state) dei record di log prima di commit. Consente di rifare le azioni (REDO)
    di una transazione che effettuato il commit ma le cui pagine modificate non sono ancora state trascritte in
    memoria di massa.
\end{itemize}
\par\smallskip
\textbf{Operazioni UNDO e REDO} \\
Vediamo nel dettaglio le operazioni UNDO e REDO:
\begin{itemize}
  \item \textbf{UNDO} \\
    L'UNDO (disfacimento) di un azione è, per le operazioni viste su un oggetto $O$:
    \begin{itemize}
      \item update, delte: copiare il valore del BS (before state) in $O$;
      \item insert: eliminare $O$.
    \end{itemize}
  \item \textbf{REDO} \\
    Il REDO (rifacimento) di un'azione è, per le operazioni viste su un oggetto $O$:
    \begin{itemize}
      \item insert, update: copiare il valore dell'AS (after state) in $O$;
      \item delete: eliminare $O$.
     \end{itemize}
\end{itemize}
Valgono le \textbf{idempotenze}:
$$ \mathrm{undo(undo(A))} = \mathrm{undo(A)} $$
$$ \mathrm{redo(redo(A))} = \mathrm{redo(A)} $$
Cioè, l'annullamento o il rifacimento di un'operazione $A$ $n$ volte è uguale all'annullamento o il rifacimento una singola volta.
\par\smallskip
\textbf{Modalità di inserzione in log} \\
Esistono più modalità di trascrizione delle operazioni sul log:
\begin{itemize}
  \item \textbf{Modalità immediata} \\
    La modalità immediata comporta la trascrizione immediata di tutte le operazioni di scrittura provenienti
    da transazioni uncommitted (che non hanno ancora fatto commit). Richiede un UNDO al momento del guasto, ma non richiede
    REDO nel caso di esito con successo.
  \item \textbf{Modalità differita} \\
    Nella modalità differita si eseguono prima tutte le update, e poi le operazioni di scrittura. Non esistono
    valor AS nel log che vengono da transazioni uncommitted. Non c'è bisogno di UNDO, in quanto non ci sono scritture
    prima del commit. Richiede però un REDO nel caso non si sappia l'esito di una transazione (dopo le update potrebbe
    comunque non essere stata eseguita).
  \item \textbf{Modalità mista} \\
    Nella modalità mista la scrittura può avvenire sia in immediata che in differita. Potrebbero
    essere necessarie sia UNDO che REDO.
\end{itemize}
Alcune considerazioni pratiche: la modalità differita non viene molto utilizzata in pratica. Questo perché
tale modalità è molto più efficiente nel \textbf{recupero} (\textit{recovery}), ma è complessivamente meno efficiente
di una modalità in cui il gestore può scegliere liberamente quando scrivere in memoria secondaria. Visto che il caso
naturale di una transizione è quello di un'esito di successo, si preferisce adottare una modalità che ottimizzi questo
tipo di risultato.
\par\smallskip
\textbf{Guasti} \\
Esistono più tipi di guasti:
\begin{itemize}
  \item \textbf{Guasti "soft"} \\
    I guasti soft (\textit{morbidi}) sono errori di programma, crash di sistema, cadute di tensione, ecc...
    Si perdono i contenuti della memoria centrale, ma non si perde né la memoria secondaria, cioè il database, né
    la memoria stabile, cioè il log. Si può fare una ripresa a caldo, (\textit{warm start}).
  \item \textbf{Guasti "hard"} \\
    I guasti hard (\textit{duri}) riguardano la memoria secondaria, ma non la memoria stabile. Se si perde la base di dati, bisogna fare una ripresa a
    freddo (\textit{cold restart}), ricostruendo la base dai log. Nel caso di perdita della memoria stabile, cioè del log,
    ogni tentativo di recupero è impossibile. Si ricomincia da capo.
\end{itemize}
\par\smallskip
\textbf{Modello di funzionamento fail-stop} \\
L'individuazione di un guasto forza l'arresto completo di tutte le transazioni in corso. Quindi si riavvia il sistema,
e si effettua una procedura di \textbf{restart}, al termine della quale, se si ha successo, si riavvia nuovamente, in caso contrario il \textbf{buffer} è vuoto, ma le transazioni
possono ricominciare.
\par\smallskip
\textbf{Procedura di restart} \\
Il processo di restart ha l'obbiettivo di classificare le transazioni in:
\begin{itemize}
  \item \textbf{Completate}: ergo tutte in memoria stabile;
  \item \textbf{In commit ma non necessariamente completate}: può servire un REDO;
  \item \textbf{Senza commit}: vanno annullate, ergo serve un UNDO.
\end{itemize}

Il gestore dell'affidabilità, al restart di sistema:
\begin{itemize}
  \item Legge su un file di RESTART (nel log) l'indirizzo dell'ultimo checkpoint;
  \item Prepara due file: un UNDO list con gli identificatori delle transazioni attive, e un REDO list vuoto;
  \item Si assicura che nessun'utente si attivo.
\end{itemize}
\par\smallskip
\textbf{Ripresa a caldo} \\
La ripresa a caldo avviene in quattro fasi:
\begin{itemize}
  \item Si trova l'ultimo checkpoint (ripercorrendo il log a ritroso);
  \item Si costruisce gli insiemi UNDO (transazioni attive ma non committed prima del guasto, da disfare) e REDO
    (transazioni committed tra il CK e il guasto, da rifare).
  \item Si ripercorre il log all'indietro (\textbf{rollback}) fino alla più vecchia azione delle transazioni in UNDO e REDO, disfacendo
    tutte le azioni delle transazioni in UNDO;
  \item Si ripercorre il log in avanti (\textbf{rollforward}) rifacendo tutte le azioni delle transazioni in redo.
\end{itemize}
\par\smallskip
\textbf{Ripresa a freddo} \\
La ripresa a freddo avviene in due fasi, più la ripresa a caldo:
\begin{itemize}
  \item Ci si riporta al record di dump più recente nel log e si ripristina la parte di dati deteriorata;
  \item Si eseguono le operazioni registrate sul log sulla parte deteriorata fino all'istante del caldo;
  \item Si esegue una ripresa a caldo.
\end{itemize}
\end{document}
