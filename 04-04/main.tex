\documentclass[a4paper,12pt]{article}

\usepackage[french,italian]{babel}
\usepackage[T1]{fontenc}
\usepackage[utf8]{inputenc}
\frenchspacing 
\title{Appunti Basi di Dati}
\author{Luca Seggiani}
\date{4 Aprile 2024}

\usepackage{listings}
\usepackage{xcolor}

\definecolor{codegreen}{rgb}{0,0.6,0}
\definecolor{codegray}{rgb}{0.5,0.5,0.5}
\definecolor{codepurple}{rgb}{0.58,0,0.82}
\definecolor{backcolour}{rgb}{0.95,0.95,0.92}

\lstdefinestyle{code-style}{
    backgroundcolor=\color{backcolour},   
    commentstyle=\color{codegreen},
    keywordstyle=\color{magenta},
    numberstyle=\tiny\color{codegray},
    stringstyle=\color{codepurple},
    basicstyle=\ttfamily\footnotesize,
    breakatwhitespace=false,         
    breaklines=true,                 
    captionpos=b,                    
    keepspaces=true,                 
    numbers=left,                    
    numbersep=5pt,                  
    showspaces=false,                
    showstringspaces=false,
    showtabs=false,                  
    tabsize=2
}

\lstset{style=code-style}

\begin{document}
\maketitle
\section{Raggruppamento}
Il raggruppamento (o aggregazione) divide in gruppi i record risultanti dalle clausole FROM e WHERE (detti record target),
sulla base di un certo attributo particolare in comune, che resta quindi costante in un determinato gruppo.
Poniamo di voler calcolare, nello schema clinica, per ogni specializzazione, la parcella media dei medici:
\begin{lstlisting}[language=SQL]
SELECT Specializzazione, AVG(Parcella) AS ParcellaMedia
FROM Medico
GROUP BY Specializzazione;
\end{lstlisting}
Dove il select di Specializzazione e AVG(Parcella) ha senso solo a seguito del GROUP BY, con Specializzazione
proprio come attributo di raggruppamento (!).
\par\smallskip
\textbf{Condizioni sui gruppi} \\
Le condizioni sui gruppi sono controllate gruppo per gruppo (non record per record) e permettono di scartarne alcuni. 
Si applicano attraverso l'operatore HAVING.
Poniamo di voler indicare le specializzazioni della clinica con più di due medici:
\begin{lstlisting}[language=SQL]
SELECT Specializzazione
FROM Medico 
GROUP BY Specializzazione
HAVING COUNT(*) > 2
\end{lstlisting}
Si nota che il DISTINCT non è necessario: ogni specializzazione viene presa comunque una volta sola.
Vediamo adesso un caso un attimo più complesso: ottenere le specializzazioni con la più alta parcella media. Dovremo
allora calcolare tutte le medie delle parcelle dei medici di ciascuna specializzazione, calcolare tra queste
la più alta, e selezionare quindi le specializzazioni la cui media corrisponde a questa media massima. \newpage
Il massimo si troverà con:
\begin{lstlisting}[language=SQL]
SELECT MAX(D.MediaParcelle)
FROM (
  SELECT M2.Specializzazione, AVG(M2.Parcella) AS MediaParcella
  FROM Medico M2
  GROUP BY M2.Specializzazione
)
AS D;
\end{lstlisting}
e la query completa sarà:
\begin{lstlisting}[language=SQL]
SELECT M.Specializzazione
FROM Medico M
GROUP BY Specializzazione
HAVING AVG(Parcella) = 
(
  SELECT MAX(D.MediaParcelle)
  FROM (
    SELECT M2.Specializzazione, AVG(M2.Parcella) AS MediaParcella
    FROM Medico M2
    GROUP BY M2.Specializzazione
  )
  AS D;
)
\end{lstlisting}
\section{Subquery correlate}
Abbiamo già visto le subquery non correlate (\textit{non-correlated}): il result set di tali subquery viene calcolato
una sola volta ed è indipendente dalle specifiche della query esterna (\textit{outer query}). Nelle subquery correlate
invece il result set dipende da ciascuna tupla della tupla esterna. L'ordine di esecuzione è: prima si esegue
il WHERE della query esterna, e poi la subquery. Poniamo ad esempio di voler indicare matricola
e parcella dei medici che hanno visitato per la prima volta almeno un paziente nel mese di ottobre 2013. Avremo
bisogno di selezionare pazienti che non erano mai stati visitati da un determinato medico prima di ottobre:
\begin{lstlisting}[language=SQL]
SELECT DISTINCT V1.Medico
FROM Visita V1
WHERE YEAR(V1.Data) = 2013
  AND MONTH (V1.Data) = 10
  AND V1.Paziente NOT IN ( 
                            SELECT V2.Paziente
                            FROM Visita V2
                            WHERE V2.Medico = V1.Medico
                              AND V2.Data < V1.Data
                          )
\end{lstlisting}
Notiamo la particolarità della subquery correlata: la relazione V1, appartenente alla query esterna, viene usata
nella clausola WHERE della correlata. Ricordiamo che una subquery correlata viene eseguita nuovamente per ogni tupla,
ed è quindi particolarmente inefficiente dal punto di vista della complessità. Notiamo inoltre che una subquery può
essere usata anche nella clausola SELECT della query, per calcolare velocemnte un valore da inserire nel risultato. Ad esempio,
vogliamo trovare tutti i pazienti di sesso maschile, indicandone nome e numero di visite effettuate. Quest'ultima informazione
non sarà un dato da proiettare (nel WHERE), ma da calcolare nel SELECT.
\begin{lstlisting}[language=SQL]
SELECT Nome, (
  SELECT COUNT(*)
  FROM Visite V
  WHERE V.Paziente = P.CodFiscale
) AS NumeroVisite
FROM Paziente P
WHERE P.Sesso = 'M'
\end{lstlisting}
\end{document}

