\documentclass[a4paper,12pt]{article}

\usepackage[french,italian]{babel}
\usepackage[T1]{fontenc}
\usepackage[utf8]{inputenc}
\frenchspacing 
\title{Appunti Basi di Dati}
\author{Luca Seggiani}
\date{11 Aprile 2024}

\usepackage{listings}
\usepackage{xcolor}

\definecolor{codegreen}{rgb}{0,0.6,0}
\definecolor{codegray}{rgb}{0.5,0.5,0.5}
\definecolor{codepurple}{rgb}{0.58,0,0.82}
\definecolor{backcolour}{rgb}{0.95,0.95,0.92}

\lstdefinestyle{code-style}{
    backgroundcolor=\color{backcolour},   
    commentstyle=\color{codegreen},
    keywordstyle=\color{magenta},
    numberstyle=\tiny\color{codegray},
    stringstyle=\color{codepurple},
    basicstyle=\ttfamily\footnotesize,
    breakatwhitespace=false,         
    breaklines=true,                 
    captionpos=b,                    
    keepspaces=true,                 
    numbers=left,                    
    numbersep=5pt,                  
    showspaces=false,                
    showstringspaces=false,
    showtabs=false,                  
    tabsize=2
}

\lstset{style=code-style}

\begin{document}
\maketitle
\par\smallskip
\textbf{Correlated subquery nel SELECT} \\
L'inserimento di subquery correlate all'interno del SELECT permette di calcolare valori da inserire in un attributo
dell'insieme risultato. Proprio per questo motivo le subquery di questo tipo dovranno obbligatoriamente essere scalari.
Poniamo ad esempio di voler considerare il nome e il numero di visite effettuate da tutti i pazienti di sesso maschile.
Attraverso il raggruppamento, si potrà semplicemente dire:
\begin{lstlisting}[language=SQL]
SELECT P.Nome, COUNT(*) AS NumeroVisite
FROM Visita V INNER JOIN Paziente P ON V.Paziente = P.CodFiscale
WHERE P.Sesso = "M"
GROUP BY P.CodFiscale
\end{lstlisting}
Lo stesso comportamento può essere ottenuto attraverso le subquery correlate, ad esempio con:
\begin{lstlisting}[language=SQL]
SELECT P.Nome, (
  SELECT COUNT(*)
  FROM Visite V
  WHERE V.Paziente = P.CodFiscale
) AS NumeroVisite
FROM Paziente P
WHERE P.Sesso = "M"
\end{lstlisting}
Si può assumere che il modulo del DBMS detto \textit{query optimizer}, che ha il compito di ottimizzare le query,
tradurrà una query simile nell'equivalente che utilizza il raggruppamento.
\end{document}
