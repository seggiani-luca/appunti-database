\documentclass[a4paper,12pt]{article}

\usepackage[french,italian]{babel}
\usepackage[T1]{fontenc}
\usepackage[utf8]{inputenc}
\frenchspacing 
\title{Appunti Basi di Dati}
\author{Luca Seggiani}
\date{15 Maggio 2024}

\usepackage{listings}
\usepackage{xcolor}

\definecolor{codegreen}{rgb}{0,0.6,0}
\definecolor{codegray}{rgb}{0.5,0.5,0.5}
\definecolor{codepurple}{rgb}{0.58,0,0.82}
\definecolor{backcolour}{rgb}{0.95,0.95,0.92}

\lstdefinestyle{code-style}{
    backgroundcolor=\color{backcolour},   
    commentstyle=\color{codegreen},
    keywordstyle=\color{magenta},
    numberstyle=\tiny\color{codegray},
    stringstyle=\color{codepurple},
    basicstyle=\ttfamily\footnotesize,
    breakatwhitespace=false,         
    breaklines=true,                 
    captionpos=b,                    
    keepspaces=true,                 
    numbers=left,                    
    numbersep=5pt,                  
    showspaces=false,                
    showstringspaces=false,
    showtabs=false,                  
    tabsize=2
}

\lstset{style=code-style}

\begin{document}
\maketitle
\par\smallskip
\textbf{Decomposizione di schemi} \\
Dato unos schema $R(T)$, l'insieme di schemi $\rho = \{ R_1(T_1), ..., R_k(T_k) \}$ è una \textbf{decomposizione} di $R$
solo se $\cup_i T_i = T$ (l'unione degli schemi dà lo schema di partenza). Si noti che questo non richiede che gli schemi $R_i$ siano disgiunti. Una decomposizione equivale allo schema di
partenza in quanto
\begin{itemize}
  \item Preserva i dati;
  \item Preserva le dipendenze funzionali.
\end{itemize}
\par\smallskip
\textbf{Teorema della perdita di dati} \\
Si ha che se $\rho = \{ R_1(T_1)....,R_k(T_k) \}$ è una decomposizione di $R(T,F)$, allora per ogni relazione $r$ che
soddisfa $R(T, F)$:
$$ r \subseteq \pi_{T_1}(r) \bowtie ... \bowtie \pi_{T_k}(r)$$
Ciò significa che una decomposizione ha perdita di dati
quando, ricostruendo una relazione, otteniamo più n-uple della relazione originaria.
\par\smallskip
\textbf{Decomposizioni che preservano i dati} \\
Dato uno schema $R(T,F)$ e una decomposizione $\rho \{ R_1(T_1), ..., R_k(T_k) \}$, $\rho$ preserva i dati se e solo se, per ogni relazione
$r$ che soddisfa $R(T,F)$ si ha:
$$ r = \pi_{T_1}(r) \bowtie ... \bowtie \pi_{T_k}(r) $$
Ciò significa che, per una decomposizione che preserva i dati, ogni istanza valida $r$ della relazione di partenza deve essere identica
al join naturale delle sue proiezioni sui $T_1$ (cioè ricostruendo la relazione si ottiene la relazione di partenza e nulla di più).
\par\smallskip
\textbf{Teorema di preservazione dei dati} \\
Sia $\rho = \{R_1(T_1),R_2(T_2)\}$ una decomposizione di $R(T,F)$. Essa preserva i dati se e solo se $T_1 \cap T_2 \rightarrow T_1 \in F^+$ oppure
$T_1 \cap T_2 \rightarrow T_2 \in F^+$. In altre parole, gli attributi comuni alle due relazioni devono essere chiave in una delle due tabelle (relazioni).
\par\smallskip
\textbf{Proiezione di un insieme di dipendenze} \\
Dato $R(T, F)$ e $T_1 \subseteq T$, la proiezione dell'insieme di dipendenze $F$ sull'insieme di attributi $T_i$ è:
$$ \pi_{T_i} = \{ X \rightarrow Y \in F^+ | X, Y \subseteq T_i \} $$
Nota bene che la proiezione si costruisce sulle dipendenze di $F^+$, non di $F$ soltanto.
\par\smallskip
\textbf{Algoritmo per il calcolo di $\pi_{T_i}(F)$} \\
Si presenta un'algoritmo per il calcolo della proiezione dell'insieme di dipendenze appena definita.
\begin{itemize}
  \item Input: $R(T, F)$ e $T_i \subseteq T$
  \item Output: $\pi_{T_i}(F)$
  \end{itemize}
\begin{lstlisting}[language=perl]
Z vuoto
for each Y in T do
  metti Y - Y in W
  metti Z unito (Y -> (W intersecato T_i)) in Z
return Z
\end{lstlisting}
Questo algoritmo ha, nel caso pessimo, complessità esponenziale.
\par\smallskip
\textbf{Decomposizioni che preservano le dipendenze} \\
Dato uno schema $R(T,F)$, e una decomposizione $\rho = \{ R_1(T_1),...,R_k(T_k) \}$, $\rho$ è una
decomopsizione di $R(T,F)$ che preserva le dipendenze se e solo se $\cup_i \pi_{T_i}(F) \equiv F$
In altre parole, la decomposizione di $R(T, F)$ in due relazioni con attributi $X$ e $Y$ preserva le dipendenze
se $\pi_X(F) \cup \pi_Y(F) \equiv F$, cioè se $(\pi_X(F)\cup\pi_Y(F)) = F^+$. Dovremo verificare quest'ultima
uguaglianza per dimostrare che una decomposizione di $R(T, F)$ preserva le dipendenze. Fare ciò significa:
\begin{itemize}
  \item Calcolare la proiezione di un insieme di dipendenze funzionali su un insieme di attributi, cosa che
    richiede un algoritmo a complessità esponenziale;
  \item Determinare l'equivalenza di due insiemi di dipendenze funzionali $X$ e $G$, cosa che richiede un'algoritmo a 
    complessità poliniomiale:
    \begin{itemize}
      \item Per ogni $X \rightarrow Y \in F$, calcoliamo $X^+_G$ e verifichiamo se $Y \in X^+_G$;
      \item Per ogni $X \rightarrow Y \in G$, calcoliamo $X^+_F$ e verifichiamo se $Y \in X^+_F$;
    \end{itemize}
\end{itemize}
\par\smallskip
\textbf{Algoritmo per la decomposizione in BCNF} \\
Possiamo adesso vedere un'algoritmo che decompone uno schema $R(T, F)$ nella sua forma normale di Boyce-Codd:
\begin{itemize}
  \item Input: $R(T, F)$ con $F$ in forma $X\rightarrow A$, non normalizzata
  \item Output: $\rho$ che preserva i dati 
\end{itemize}
\begin{lstlisting}[language=perl]
metti R(T,F) in P
while esiste R_i(T_i, F_i) in P che non e in BCNF do
  for each X -> A in F_i do
    if A not in X and T_i not in X+ then
      metti R_i(T_i - A, proiezione su T_1 - A di F_i) in R_1 #questi sono tutti gli attributi di R - A
      metti R_i(X + A, proiezione su X + A di F_i) in R_2 #questi sono tutti gli attributi della dipendenza funzionale in esame
      metti P - (R_i unito a {R_1,R_2}) in P
      break
return P
\end{lstlisting}
Questo algoritmo termina e produce una decomposizione della relazione tale che:
\begin{itemize}
  \item La decomposizione prodotta è in BCNF;
  \item La decomposizione prodotta preserva i dati.
\end{itemize}
Non è detto che la decomposizione preservi le dipendenze!
\par\smallskip
\textbf{Qualità delle decomposizioni} \\
Una decomposizione dovrebbe sermpe garantire:
\begin{itemize}
  \item Di essere in BCNF;
  \item Di non presentare perdite di dati, in modo da permettere la ricostruzione dell'informazione originale attraverso join naturali;
  \item Di conservare le dipendenze funzionali, in modo da mantenere i vincoli di integrità originari.
\end{itemize}
Quando non si riesce a raggiungere una forma normale di Boyce-Codd, probabilmente c'è stato un cattivo processo di progettazione prima della normalizzazione.
In ogni caso, esiste una forma normale alternativa (e meno restrittiva) alla Boyce-Codd:
\par\smallskip
\textbf{Terza forma normale} \\
Una relazione $R(T, F)$ è in terza forma normale (3NF) se e solo se, per ogni dipendenza funzionale non banale $X \rightarrow A \in F^+$,
è verificata almeno una delle due condizioni:
\begin{itemize}
  \item $X$ è una superchiave di $R$ (come nella Boyce-Codd);
  \item $A$ è contenuto in almeno una chiave di $R$ (si dice che $A$ è un attributo primo).
\end{itemize}
Vediamo che la BCNF implica la 3NF: uno schema in BCNF è automaticamente in 3NF, ma non tutti gli schemi in 3NF sono in BCNF.
Il problema della 3NF è la sua \textbf{verifica}: il problema di decisione è NP-completo. Il miglior algoritmo deterministico
di risoluzione ha complessità esponenziale nel caso peggiore. Questo è dato dalla ricerca degli attributi primi, cioè le chiavi, che ha
complessità esponenziale. Tuttavia, si può sempre ottenere una decomposizione in 3NF che preserva dati e dipendenze funzionali.
\par\smallskip
\textbf{Algoritmo per la decomposizione in 3NF} \\
Ci basiamo su un intuizione: dato un insieme di attributi $T$ e una copertura minimale $G$, si divide $G$ in gruppi $G_i$ in modo che tutte le dipendenze
funzionali di ogni gruppo $G_i$ abbiano la stessa parte sinistra. Da ogni gruppo $G_i$ si definisce uno schema di relazione composto
da tutti gli attributi che appaioni in $G_i$, la cui chiave (detta \textbf{chiave sintetizzata}) è la parte sinistra comune. L'algoritmo risulta quindi:
\begin{itemize}
  \item Input: $R(T, F)$
  \item Output: $\rho$ che preserva dati, dipendenze ed è in 3NF 
\end{itemize}
\begin{lstlisting}[language=perl]
metti la copertura minimale di F in G
poni P vuoto
#sostituisci ogni insieme di dipendenze {X -> A_1, ..., X -> A_h} con X -> A_1, ..., A_h
for each X -> Y in G do
  #crea uno schema con attributi XY in P
#elimina da P ogni schema che sia contenuto in un'altro schema di P
#se P non contiene nessuno schema i cui attributi costituiscono una superchiave di R, aggiungere a P uno schema con attributi W, dove W e' una chiave di R
\end{lstlisting}
L'esecuzione dell'algoritmo termina e produce una decomposizione tale che:
\begin{itemize}
  \item La decomposizione prodotta è in 3NF;
  \item La decomposizione prodotta preserva i dati e le dipendenze funzionali.
\end{itemize}
Questo algoritmo ha complessità polinomiale: paradossalmente, trovare una soluzione è polinomiale, ma verificarla è esponenziale
\end{document}
