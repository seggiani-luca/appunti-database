\documentclass[a4paper,12pt]{article}

\usepackage[french,italian]{babel}
\usepackage[T1]{fontenc}
\usepackage[utf8]{inputenc}
\frenchspacing 
\title{Appunti Basi di Dati}
\author{Luca Seggiani}
\date{27 Marzo 2024}

\usepackage{listings}
\usepackage{xcolor}

\definecolor{codegreen}{rgb}{0,0.6,0}
\definecolor{codegray}{rgb}{0.5,0.5,0.5}
\definecolor{codepurple}{rgb}{0.58,0,0.82}
\definecolor{backcolour}{rgb}{0.95,0.95,0.92}

\lstdefinestyle{code-style}{
    backgroundcolor=\color{backcolour},   
    commentstyle=\color{codegreen},
    keywordstyle=\color{magenta},
    numberstyle=\tiny\color{codegray},
    stringstyle=\color{codepurple},
    basicstyle=\ttfamily\footnotesize,
    breakatwhitespace=false,         
    breaklines=true,                 
    captionpos=b,                    
    keepspaces=true,                 
    numbers=left,                    
    numbersep=5pt,                  
    showspaces=false,                
    showstringspaces=false,
    showtabs=false,                  
    tabsize=2
}

\lstset{style=code-style}

\begin{document}
\maketitle
\section{Subquery}
Le subquery (query annidate) permettono di incapsulare query in altre query, in modo correlato o non correlato.
Presentano un modo alternativo di risolvere i problemi su cui normalmente si userebbe il join: infatti per ogni
problema che si può risolvere usando il join esiste sempre un corrispettivo che usa le subquery. Chiamiamo
outer query la query più esterna, cioè quella che incapsula, e inner query la subquery incapsulata.
\par\smallskip
\textbf{Subquery non correlate} \\
Le subquery non correlate (\textit{noncorrelated}) sono incapsulate nel WHERE, e permettono di ottenere un certo
result set intermedio, che verrà poi usato dall'outer query per calcolare il result set finale. Il result set
della subquery non correlata non dipende dall'outer query. Poniamo ad esempio di voler indicare nome, cognome 
e parcella degli ortopedici che hanno effettuato almeno una visita nell'anno 2013. Definiremo allora la subquery:
\begin{lstlisting}[language=SQL]
SELECT V.Medico
FROM Visita V
WHERE YEAR(V.Data) = 2013
\end{lstlisting}
che innesteremo poi in:
\begin{lstlisting}[language=SQL]
SELECT M.Nome, M.Cognome, M.Parcella
FROM Medico M
WHERE M.Specializzazione = 'Ortopedia'
  AND M.Matricola IN  (
                      SELECT V.Medico
                      FROM VIsita V
                      WHERE YEAR(V.Data) = 2013
                      );
\end{lstlisting}
La parola chiave IN permette di controllare la presenza dell'attributo M.Matricola nel result set della subquery.
La subquery calcolerà quindi un result set composto da tutte le visite fatte nel 2013. L'outer query userà la subquery,
confrontandola con tutti i medici specializzati in ortopedia la cui matricola trova un riscontro nel result set
calcolato dalla subquery. A questo punto si restituiscono gli attributi richiesti. Il meccanismo è del tutto analogo
a quello di un join naturale fra le tabelle "Medico" e "Visita". Notiamo che non è necessario alcuna istruzione DISTINCT, 
in quanto ogni record della tabella Medico viene considerato comunque una volta sola, e cercato nel result set delle visite. \\
Avremmo potuto porre la stessa query usando un comune join, ad esempio come:
\begin{lstlisting}[language=SQL]
SELECT M.Nome, M.Cognome, M.Parcella
FROM Medico M 
  INNER JOIN Visita V ON V.Medico = M.Matricola
WHERE YEAR(V.Data) = 2013 
  AND Medico.Specializzazione = 'Ortopedico';
\end{lstlisting}
\par\medskip
Si può anche usare la negazione di quanto appena fatto, usando la parola chiave NOT IN. Ad esempio, se vogliamo indicare i cognomi
dei pazienti che non appartengono anche a un medico, abbiamo la subquery:
\begin{lstlisting}[language=SQL]
SELECT M.Cognome
FROM Medico M
\end{lstlisting}
innestata in:
\begin{lstlisting}[language=SQL]
SELECT DISTINCT P.Cognome
FROM Paziente P
WHERE P.Cognome NOT IN  (
                        SELECT M.Cognome
                        FROM Medico M
                        );
\end{lstlisting}
\par\smallskip
Analogamente a prima, esisterà una versione che usa il join:
\begin{lstlisting}[language=SQL]
SELECT DISTINCT P.Cognome
FROM Paziente P
  LEFT OUTER JOIN Medico M ON
    M.Cognome = P.Cognome
WHERE M.Cognome IS NULL;
\end{lstlisting}
\textbf{Annidamento multiplo} \\
Non c'è teoricamente limite al numero di subquery che si possono innestare l'una dentro l'altra. A volte può infatti
tornare utile avere innesti multipli (e frontali!). Vediamo come ottenere il numero di tutti i pazienti di Siena mai visitati
da pazienti. Innanzitutto si trovano i medici specializzati in ortopedia:
\begin{lstlisting}[language=SQL]
SELECT M.Matricola
FROM Medico M
WHERE M.Specializzazione = 'Ortopedia'
\end{lstlisting}
\newpage
poi i pazienti visitati da suddetti medici:
\begin{lstlisting}[language=SQL]
SELECT V.Paziente
FROM Visita V
WHERE V.Medico IN (...)
\end{lstlisting}
e si rimuovono da una tabella dei soli pazienti senesi:
\begin{lstlisting}[language=SQL]
SELECT COUNT(*)
FROM Paziente P
WHERE P.Citta = 'Siena'
  AND P.CodFiscale NOT IN (
    SELECT V.Paziente
    FROM Visita V
    WHERE V.Medico IN (
      SELECT M.Matricola
      FROM Medico M
      WHERE M.Specializzazione = 'Ortopedia'
    )
  );
\end{lstlisting}
\par\smallskip
\textbf{Subquery scalari} \\
Una subquery scalare produce, invece che un insieme di record che poi verrà controllato con IN e NOT IN,
un singolo record di un singolo attributo, come ad esempio farebbe la funzione COUNT(). A questo punto, il valore
scalare prodotto può essere controllato con i vari operatori di confronto. Vogliamo ad esempio indicare il numero
degli otorini aventi parcella più alta della media delle parcelle della loro specializzazione. Troviamo innanzitutto
il la parcella media degli otorini (un valore scalare):
\begin{lstlisting}[language=SQL]
SELECT AVG(Medico.Parcella)
FROM Medico M
WHERE M.Specializzazione = 'Otorinolaringoiatria'
\end{lstlisting}
e la useremo in un'outer query:
\begin{lstlisting}[language=SQL]
SELECT COUNT(*)
FROM Medico M1
WHERE M1.Parcella >  (
                      SELECT AVG(Medico.Parcella)
                      FROM Medico M
          WHERE M.Specializzazione = 'Otorinolaringoiatria'
                      );

\end{lstlisting}
\par\medskip
Un caso interessante: mettiamo di voler trovare le entrate complessive generate dai cardiologi della clinica
negli ultimi 2 anni. Ciò risulta semplice con un join:
\begin{lstlisting}[language=SQL]
SELECT SUM(M.Parcella) AS IncassoTotale
FROM Visita V
INNER JOIN Medico M ON V.Medico = M.Matricola
WHERE V.Data > CURRENT_DATE - INTERVAL 12 YEAR
  AND M.Specializzazione = 'Cardiologia'
\end{lstlisting}
La versione con subquery è invece più complessa, richiedendo una subquery nel SELECT:
\begin{lstlisting}[language=SQL]
SELECT SUM((
  SELECT M.Parcella
  FROM Medico M
  WHERE M.Matricola = V.Medico
     AND M.Specializzazione = 'Cardiologia'
)) AS IncassoTotale
FROM Visita V
WHERE V.Data > CURRENT_DATE - INTERVAL 12 YEAR
\end{lstlisting}
Notiamo che la subquery è correlata, come vedremo nella prossima lezione.
\end{document}

